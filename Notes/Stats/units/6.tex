\documentclass[../AP_Statistics.tex]{subfiles}

\begin{document}
	\chapter{Confidence Intervals}
	\chapter{Significance Tests}
		A \textbf{significance test} is a procedure that uses observed data to test between two claims, often made regarding parameters, about hypotheses. \\
		The \textbf{null hypothesis} ($\pmb{H_0}$) claims that the parameter is equal to a \textbf{null value}, what it was previously assumed to be, denoted by a subscript $0$ on the parameter. It is often a statement of no change or difference.
		$$H_0:\rm{parameter} < \text{null value}$$
		The claim that is attempting to be supported is the \textbf{alternative hypothesis} ($\pmb{H_a}$. It can either be \textbf{one-sided}, claiming that the parameter is greater or less than the null value, or \textbf{two-sided}, claiming simply that the parameter is not equal to the null value.
		$$H_a:\rm{parameter} \gtrless \text{null value} \lor \rm{parameter} \ne \text{null value}$$
		A test's $\pmb{P}$\textbf{-value} is the probability of evidence being found for $H_a$ that is at least as strong as it would be assuming that $H_0$ is true.
		$$P\text{-value} = P(\text{statistic supports } H_a \mid H_0)$$ \\
		The smaller the $P$-value, the lower the chances of receiving evidence of the alternative. A small $P$-value therefore supports the $H_a$. \\
		If the $P$-value is less than the \textbf{significance level} $\pmb{\alpha}$, $H_0$ can be rejected and it can be concluded that there is convincing evidence for $H_a$. If the $P$-value is greater than or equal to $\alpha$, $H_0$ cannot be rejected, and it can be concluded that there is not convincing evidence for $H_a$.
		\callout{11.74}{Claims should never be made regarding support of $H_0$, only of $H_a$.}
		When performing significance tests, two types of errors may occur. A \textbf{Type \Roman{1} error} occurs when $H_0$ is rejected despite being true; the data provided convincing evidence for $H_a$ despite $H_0$ being correct. A \textbf{Type \Roman{2} error} occurs when $H_0$ is not rejected despite $H_a$ being true; the did not provide convincing evidence for $H_a$ despite it being correct. \\
		The probability of a Type \Roman{1} error occurring is equal to $\alpha$. \\
		As $\alpha$ increases, the probability of a Type \Roman{1} error increases but that of a Type \Roman{2} error decreases.
		\section{Significance Tests about Proportions}
			In order for a significance test of $H_0:p = p_0$ to be performed, it must be verified that the distribution of $\hat{p}$ is approximately Normal assuming $H_0$, so Large Counts and the 10\% condition must be satisfied and interpreted.
			To perform the significance test, the \textbf{standardized test statistic} $z$ must be calculated. \\
			$$z = \frac{\hat{p} - \mu_{\hat{p}0}}{\sigma_{\hat{p}0}} = \frac{\hat{p} - p_0}{\sqrt{\frac{p_0(1 - p_0)}{n}}}$$
			The $P$-value is the probability of $p$ not satisfying $H_0$ in the way that $H_a$ predicts, which can be calculated using the cumulative probability function.
			$$
				P\text{-value} = P(H_a) = \begin{cases}
 					P(p < p_0) = P(Z < z) & H_a:p < p_0 \\
 					P(p > p_0) = P(Z > z) & H_a:p > p_0 \\
 					P(|p| < |p_0|) = P(Z < -|z|) + P(Z > |z|) & H_a:p \ne p_0
				\end{cases}
			$$
			A confidence interval can be used in tandem with a sample proportion to provide a set of plausible values for $p$, should the alternative hypothesis be convincingly supported. \\
			A two-sided test of of $H_0:p = p_0$ at significance level $\alpha$ usually provides the same conclusion as a confidence level of the complement of $\alpha$. \\
			A test's \textbf{power} is the probability of a Type \Roman{2} error being avoided.
			$$\mathrm{Power} = P(\text{Type \Roman{2} Error})^C = P(\text{statistic convincingly supports } H_a \mid H_a \text{ is true})$$
			Power can be increased by increasing the sample size or significance level or by increasing the \textbf{effect size}, the minimum difference between the null and alternative parameter values. Improving data collection by doing things such as controlling other variables and blocking in experiments or performing a stratified random sample, can also increase the power. \\
			The factors that influence the necessary sample size for a statistical test are the significance level, effect size, and desired power.
			\subsection*{Significance Tests about Differences in Proportions}
				A significance can be performed to compare the proportions for two populations is based on the difference between sample proportions. \\
				The null hypothesis for
		\section{Significance Tests about Means}
		\chapter{Chi-Square Tests}
	\chapter{Slopes}
\end{document}