\documentclass[../AP_Statistics.tex]{subfiles}

\begin{document}
	\chapter{Sampling}
		A \textbf{population} consists of every \textbf{individual} that is in a defined group, while a \textbf{sample} is a subset of a population of interest. \\
		A \textbf{study} refers to a \textbf{sample} or an \textbf{experiment}.
		Studies involving humans must be screened by an \textbf{institutional review board} before they can happen. All participants in said studies must give their \textbf{informed consent} prior to their participation. Any information regarding specific individuals must be kept \textbf{confidential}. \\
		\textbf{Observational studies} observe individuals, measuring \textbf{variables} of interest but not attempting to influence a response. \\
		\textbf{Sampling} attempts to gain information regarding a population by studying subgroups. A \textbf{census}, on the other hand, attempts to gain information regarding all individuals within an area of interest. \\
		\section*{Sample Designs and Bias}
		The \textbf{sample design}, how the sampling is carried out, is crucial to take into account when attempting to collect data that is \textbf{representative} of the population of interest. \\
		\textbf{Bias} is the systematic skewing of results. \\
		\textbf{Voluntary response bias} occurs when data is only collected by those that want to have their data collected. Those with strong opinions are more likely to want to be heard, so they more actively respond. This tends to result in a negative bias. \\
		\textbf{Under-coverage bias} occurs when some portion of the population is left out. \\
		\textbf{Nonresponse bias} occurs when some individuals that are chose elect not to participate. \\
		\textbf{Response bias} occurs when respondents change their results due to the sampling design.\footnote{If questions being asked in a survey are too complicated, responses will likely be non-representative}. \\
		\textbf{Convenience} sampling fails to use randomness. While this makes data easier to obtain, that data will likely be unrepresentative of the population. \\
		The fact that different random samples of the same size from the same population may produce different estimates is called \textbf{sampling variability}. This is reduced by increasing the sample size. \\
		\textbf{Inference} (generalization of results) regarding a population requires that all individuals taking part in a sample be randomly selected from the population. \\
		Evidence of causality requires a strong association that consistently appears across many studies. \\
		Observed results that are too improbable to be explained by chance alone are \textbf{statistically significant}.
		\subsection*{Random zSampling}
			To eliminate some potential bias, chance can be utilized in choosing the sample. The simplest way to do this is to use a \textbf{simple random sample} (\textbf{SRS} or probability design). An SRS of size $n$ consists of $n$ individuals from the population chosen such that each individual has an equal chance of being chosen. \\
			To create an SRS, numerical values can be assigned\footnote{A common method for assigning numbers to each member of a population is to assign them alphabetically.} to each individual in the population such that each number has the same number of digits. The first number should be either 1 or 0 with the appropriate number of leading zeros to account for the total number of individuals. Numbers can then be randomly selected. \\
			\textbf{Systematic random sampling} follows the same rules for numerical assignment as simple random sampling, but rather than randomly selecting $n$ numbers, a single number from the minimum to $k$ (or $k - 1$ if the minimum is 0) is selected, where $k$ is the population size divided by $n$, and each number that is the sum of that number and an integer multiple of $k$ is used. \\
			When the population is large and diversified among many categories, a \textbf{stratified random sample} can be used. The population is divided into non-overlapping subgroups called \textbf{strata}, each of which is subjected to an SRS before they are recombined. When each stratum is \textbf{homogenous}, individuals within the same stratum having similar values, and strata are different from each other, stratified random sampling is preferable to simple random sampling. \\
			\textbf{Cluster sampling} divides the population into groups of people that are geographically close to each other called \textbf{clusters}. When each cluster is \textbf{heterogenous}, individuals within the same cluster not having similar values, and all clusters are similar, cluster sampling can be used, saving time and money. \\
	\chapter{Experimentation}
		\textbf{Experiments} are studies that impose a \textbf{treatment}, the experimental condition applied, upon some group, called \textbf{experimental units} (or \textbf{subjects} if human), to observe the results. They often aim to show that a change in one variable, the \textbf{explanatory variable} or factors, causes a change in another variable, the \textbf{response variable}. \\
		Many experiments combine several factors, so each treatment is made by combining specific values, called \textbf{levels}, of each factor. \\
		Factors attempt to explain the results. \\
		Experiments provide evidence for \textbf{causality}, which cannot be done by samples. They also control \textbf{lurking variables}, external variables that affect the response variable. Additionally, they enable the combination of several factors. \\
		\textbf{Confounding variables} are variables that are tied together such that one's affects on the response variable cannot be distinguished from the other's.
		Well-designed experiments with random assignment enable \textbf{inference} regarding causality.
		\section*{Experimental Design}
			A \textbf{comparative design} is one that compares two or more treatments.
			A \textbf{control group} is a group who's treatment is set up to be compared to the real treatments. They are given a \textbf{placebo}, a dummy treatment.
			If neither the subjects nor those measuring their treatments are aware of who is receiving what treatment, the experiment is \textbf{double-blind}, as is the case with many medical and behavioral experiments. If one group knows, it is \textbf{single-blind}. \\
			When \textbf{randomization} is used to group subjects, the resultant groups should be similar in all respects prior to the application of treatments. \\	
			\textbf{Control}, keeping all variables apart from the treatment the same for all groups, helps to avoid confounding and reduces variation in responses, making it easier to determine a treatment's efficacy.	\\
			Each treatment should be imposed on enough experimental units that the effects of the treatments can be distinguished from chance differences between groups, ensuring \textbf{replicability}. \\
			A \textbf{completely randomized design} assigns treatments to experimental units completely at random. \\
			A \textbf{randomized block design} divides the experimental units into groups, referred to as \textbf{blocks}, that are similar with respect to a variable that is expected to affect the response. Within each block, responses are compared and combined with those of other blocks after accounting for differences between blocks. \\
			A \textbf{matched pairs design} can be used to compare to two treatments. It may involve each subject receiving both in a random order or two similar subjects being paired and the treatments being randomly assigned within each pair.
\end{document}