\documentclass[../AP_Statistics.tex]{subfiles}

\begin{document}
	\chapter{Probability}
		\textbf{Probability}\index{probability} is the long-run relative frequency\index{frequency!relative}. It must be between 0 and 1 (inclusive). \\
		The short-term is unpredictable, but the long-term is predictable. \\
		The \textbf{law of large numbers}\index{law!of large numbers} states that as the number of trials approaches infinity, the \textbf{experimental} (observed) probability\index{probability!experimental} will converge to the \textbf{theoretical} (calculated) probability\index{probability!theoretical}. \\
		A \textbf{sample space}\index{sample!space} $S$ is a list of all possible outcomes. It can be used in calculating theoretical probabilities when each outcome is equally likely.
		A \textbf{probability model}\index{probability!model} is a description of a random process. It is comprised of a sample space and a list of each outcome's corresponding probability. \\
		An \textbfi{event} is any collection of outcomes. \\
		For a probability model to be valid, any individual event's probability must be within $[0, 1]$ and the sum of the probabilities of all outcomes must be equal to zero. \\
		The \textbf{complement rule}\index{rule!complement} states that the probability of some event not occurring, denoted by the superscript $C$ above the event, is equal to 1 minus its probability of occurring. \\
		\[P\left(A^C\right) = 1 - P(A)\]
		It is quite clear that the converse of the complement rule also holds true, which means that a complement's complement is nothing but the original event. 
		\[P\left(\left(A^C\right)^C\right) = P(A)\]
		\callout{17}{The complements of binary relations should be noted:\footnotesize\begin{align*}(A < B)^C &= A \ge B & (A > B)^C &= A \le B & (A = B)^C &= A \ne B & (A < B < C)^C &= (B\le A)\lor (B\ge C)\end{align*}}
		The \textbfi{intersection} of two events, denoted by a $\cap$ between them, occurs when both events occur. It follows the commutative property
		\[P(A\cap B) = P(B\cap A)\]
		The \textbfi{union} of two events occurs when exactly one of the two events occurs. This is also commutative. \\
		\[P(A\cup B) = P(B\cup A)\]
		The \textbf{general addition rule}\index{rule!addition!general} states that the probability of exactly one of two events occurring is equal to their sums of their probabilities of occurring by themselves minus the that of both occurring.
		\[P(A\cup B) = P(A) + P(B) - P(A\cap B)\]
		Complements\index{complement} can be applied to the unions\index{union} and intersections\index{intersection} of two events\index{event}.\footnote{
			The way that complements affect unions and intersections follows De Morgan's law:\begin{align*}\lnot(A\lor B) &= \lnot A\land\lnot B & \lnot(A\land B) &= \lnot A\lor\lnot B\end{align*}This truth is disguised by the fact that \enquote{union} in statistics means \enquote{symmetric difference} in logic.\ssmall
			\begin{align*}
				a &:= x\in A \land b:= x\in B \\
				(A\triangle B)^C &\equiv \lnot(x\in A\oplus x\in B) & (A\cap B)^C &\equiv \lnot(x\in A \land x\in B)\\
				&\equiv \lnot(a \oplus b) &&\equiv \lnot(a \land b)\\ 
				&\equiv \lnot((a\land\lnot b)\lor(b\land\lnot a)) &&\equiv \lnot a\lor \lnot b\\
				&\equiv \lnot(a\land\lnot b)\land\lnot(b\lor\lnot a) &&\equiv (\lnot a\land\lnot b)\lor 0 \\
				&\equiv (\lnot a \lor b)\land(\lnot b\lor a) &&\equiv (\lnot a\lor\lnot b)\lor(p\land (q\land\lnot q)) \\
				&\equiv (\lnot a \land(\lnot b\lor a)) \lor (b\land(\lnot b\lor a)) &&\equiv(\lnot a\lor\lnot b)\lor((a\land a)\land(a\land\lnot b)) \\
				&\equiv ((\lnot a \land \lnot b)\lor (\lnot a \land a)) \lor ((b \land \lnot b) \lor(b\land a)) &&\equiv(\lnot a\lor \lnot b)\lor((a\land b)\land(a\land\lnot b)) \\
				&\equiv (\lnot a \land \lnot b) \lor 0 \lor 0 \lor (b\land a) &&\equiv(\lnot a\lor\lnot b)\land(((a\land b)\lor(a\land\lnot a))\land((\lnot b\land b)\lor(\lnot b\land a))) \\
					&\equiv (x\notin A \land x\notin B)\lor(x\in B \land x\in A) &&\equiv(\lnot a\land\lnot b)\land(a\lor(b\land\lnot a))\land((\lnot b\lor(b\land\lnot a))) \\
				&\equiv (A^C\cap B^C)\cup(A\cap B) &&\equiv(\lnot a\land\lnot b)\lor((a\land\lnot b)\lor(b\land\lnot a)) \\
				&&&\equiv (x\notin A \land x\notin B)\lor((x\in A\land x\notin B)\lor(x\in B\land x \notin A)) \\ 
				&&&\equiv(A\triangle B)\cup(A^C\cap B^C)
			\end{align*}
		}
		\begin{align*}
			P(A\cup B)^C &= P(A\cap B) + P\left(A^C\cap B^C\right)	& P(A\cap B)^C = P(A\cup B) + P\left(A^C \cap B^C\right)
			\end{align*}
		The \pindex{probability} of one event occurring and another not is equal to the difference between the probabilities of the first event occurring and both occurring.
		\[P\left(A\cap B^C\right) = P(A) - P(A\cap B)\]
		For two events to be \textbfi{mutually exclusive} or \textbfi{disjoint}, it must be impossible for both to occur.
		\[P(A\cap B) = 0\]
		Mutually exclusive events follow the \textbf{addition rule for mutually exclusive events}\index{rule!addition!for mutually exclusive events}, which states that the probability of their intersection is equal to their sum.
		\[P(A\cup B) = P(A) + P(B) + P(A\cap B) = P(A) + P(B)\]
		The probability of one event occurring given that another has already occurred is a \textbf{conditional probability}\index{probability!conditional}. It is equal to the probability of both events occurring divided by that of the given event.\footnote{On a two-way table\index{table!two-way}, $P(A|B)$ is the intersection of $A$ and $B$ divided by the total of $B$.}
		\[P(A|B) = \frac{P(A\cap B)}{P(B)}\]
		This allows probability of the intersection of two events to be calculated.
		\[P(A\cap B) = P(A) \times P(B|A) = P(B) \times P(A|B)\]
		The commutative property is not always followed by conditional probabilities.
		\[\lnot\square[P(A|B) = P(B|A)]\]
		Two events are \textbfi{independent} if the one occurring does not affect the probability of the other occurring.
		\[P(A|B) = P(A) \land P(B|A) = P(B)\]
		\callout{17}{\emph{Mutually exclusive} events have no common outcomes while \emph{independent} events are not affected by one of the events occurring.}
		The probability of the intersection of two independent events is simply the product of their probabilities.
		\[P(A\cap B) = P(A|B) \times P(B) = P(A) \times P(B)\]
		If two events are not independent, they are \textbf{dependent}.
		\section*{Simulation}
			%TODO Probability/Simulation
	\chapter{Random Variables}
		\textbf{Random variables}\index{variable!random} take numerical values that describe the outcomes of some \pindex{chance} process. \\
		A \emph{probability distribution}\index{distribution!probability} describes all possible outcomes and their probabilities. \\
		For a probability distribution to be valid, the sum of all probabilities must be one and all individual probabilities be within $[0,1]$. \\
		The probability of $X$ being equal to $x_i$ can be denoted as either $P(x_i)$ or as $p_i$.
		\[P(X = x_i) = P(x_i) = p_i\]
		The \emphi{mean} or \textbfi{expected value} (denoted $E(X)$) is a weighted average of all possible values of $X$.
		\section{Discrete Random Variables}
			A \textbf{discrete random variables}\index{variable!discrete!random} have a fixed finite set of possible values with gaps between them. \\
			The \emph{mean} $\mu$ of a discrete random variable is equal to the sum of the products of each value and its frequency.\footnote{A random variable's sample space can be input into two lists, one corresponding to its values and the other to their frequencies. From there, \texttt{1-Var Stats} can be run the the former input as the \texttt{List} and the latter as the \texttt{FreqList}. This provides the values of $\mu_X$, $\sigma_X$, and $\sigma_X^2$.}
			\[\mu_X = \sum x_ip_i\]
			\callout{13.605}{The mean of a discrete random variable need not be one of its possible values.}
			The \emph{variance} $\sigma^2$ of a discrete random variable is the sum of the products of the square of the distance from the mean multiplied by frequency.\footnote{Multiplying a value by its relative frequency is the same as multiplying it by its frequency and dividing it by the size of the data set.}
			\[\sigma_X^2 = \sum(x_i - \mu_X)^2p_i\]
		\section{Continuous Random Variables}
			A \textbf{continuous random variable} can take on any value within an interval (potentially unbounded). \\
			The probability of $X$ falling within an interval (represented as a binary relation) is equal to the area under the variable's \emph{density curve}\footnote{See section 1.3 for more on density curves} between the bounds of the region.\footnote{The probability of a continuous variable is, of course, an integral.\[P(a < X < b) = \int_a^bf_X(x)\d x\]} \\
			As the number of values that a continuous variable may adopt is uncountably infinite, the probability of $X$ being any particular value is always equal to 0.
			\[P(X = x) = 0\]
			As such, the inclusivity of the bounds of the interval is irrelevant.
			\[P(x_1 \le X \le x_2) = P(X = x_1) + P(x_1 < X < x_2) + P(X = x_2) = P(x_1 < X < x_2)\]
			A density curve's \emph{mean} is the value of at which it would be balanced.\footnote{The mean and variance of a continuous variable are defined just like that of a discrete one, only using integrals in place of sums and $f_X(x)$ rather than $p_i$.\begin{align*}\mu_X = \sum x_ip_i = \int_{-\infty}^\infty xf_X(x)\d x && \sigma_X^2 = \sum(x_i - \mu_X)^2p_i = \int_{-\infty}^\infty (x - \mu_X)^2f_X(x)\d x \end{align*}} \\
		\section{Combinations and Linear Transformations of Random Variables}
			Two \emph{independent} variables can be combined by using all possibilities for the sum or difference of $x_i$ and $y_i$ and the set of possible values for a new variable $Z$. The probability of $Z$ being a particular value $z_i$ is simply the product of the corresponding probabilities of $x_i$ and $y_i$.\footnote{A variable $Z$ and its corresponding sample space can be defined as a combination of $X$ and $Y$ as such:\begin{align*}Z = \{x\pm y\mid(x, y) \in X \times Y\} && S_Z = \{(x \pm y, p(x) \times p(y)) \mid (x, y) \in X \times Y\}\end{align*}} \\
			When adding or subtracting two independent random variables, the means are added or subtracted from each other and the variances are always added.\footnote{Means can be added or subtracted correspondingly with a combination of random variables regardless of independence.} \\
			\begin{align*}
				\mu_{X \pm Y} = \mu_X \pm \mu_Y && \sigma_{X \pm Y}^2 = \sigma_X^2 + \sigma_Y^2
			\end{align*}
			\callout{13.958}{
				It should be noted that \emph{standard deviation} must must be derived from variance.
				\[\sigma_{X \pm Y} \ne \sigma_X \pm \sigma_Y\]
			}
			A \textbf{linear transformation} takes the form of a constant added to the product of a constant and a variable.
			\[Y = a + bX\]
			The mean can simply be plugged in to this formula to find the new mean while the standard deviation is only multiplied.
			\begin{align*}
				\mu_Y &= a + b\mu_X & \sigma_Y &= b\sigma_X
			\end{align*}
	\chapter{Probability Distributions}
		\section{Binomial Distributions}
			\callout{17}{
				For a \emph{discrete} random variable to be \textbf{binomial}, it must meed the following conditions:
				\begin{enumerate}
					\item
						The possible outcomes of each trial must be \textbf{binary}, meaning that each trial can either be a success or a failure.
					\item
						Each trial must be \emph{independent}.
					\item
						The number of trials must be fixed in advance.
					\item
						The probability of success must be constant between trials.
				\end{enumerate}
			}
			A binomial distribution with parameters $n$ trials and probability $p$ of any given trial being successes can be denoted $B(n, p)$. \\
			The probability of $x$ successes in $n$ trials, each with a probability $p$ of being successful, is defined as such:\footnote{The binomial $\binom{n}{x}$ is defined as such:\[\binom{n}{x} = \frac{n!}{(n - x)!}\]}
			The probability of $x$ successes after $n$ trials is also referred to as the \textbf{binomial coefficient}\index{coefficient!binomial} and is denoted as such:
			\[P(x) = f(x,n,p)\]
			\[P(x) = \binom{n}{x}p^x(1 - p)^{n - x}\]
			This is the product of the probability of $x$ successes, that of $n - x$ failures (necessitated by the definition of a binomial), and the total number of possibilities in which there are $x$ successes.\footnote{The probability of exactly and at most $x$ successes can be found using $\binompdf$ and $\binomcdf$ respectively:\begin{align*}P(x) = \binomPDF{n}{p}{x} && P(X < x) = \binomCDF{n}{p}{x}\end{align*}} \\
			The \textbf{sample space} of a binomial variable can be derived as each term of a binomial expansion\footnote{The binomial expansion of $a + b$ raised to power $n$ is the following:\[(a + b)^n = \sum_{n = 0}^n\binom{n}{k}a^{n - k}b^k\]} of the conjugate of $p$ added to $p$ itself.
			\[S_{B(n, p)} = ((1 - p) + p)^n = \sum_{x = 0}^n\binom{n}{x}(1 - p)^{n - x}p^x\]
			A binomial variable's \emph{expected value} (\emph{mean}) is equal to the product of the number of trials and the probability of success while its standard deviation is the product of the probability of success and the number of trials.\footnote{The probability is simply the long-run relative frequency, so the expected number of successes for a binomial distribution is simply the expected relative frequency of successes multiplied by the total number of trials.}
			\[\mu_X = np\]
			The \emph{variance} of a binomial variable is the product of the number of trials and the probability of success and its conjugate.
			\[\sigma_X^2 = np(1 - p)\]
			A binomial distribution\index{distribution!binomial} can be graphed as a \pindex{histogram}.\footnote{A binomial distribution's graph can be thought of as a piecewise.\[f(x, n, p) = \begin{cases}\binom{n}{x}(1 - p)^{n - x}p^x & x\in \mathbb{N} \\ f(\nint(x), n, p) & x\notin \mathbb{N}\end{cases}\]} \\
			Depending on the value of $p$, the skew of the binomial distribution will vary. \\
			\begin{center}
				\begin{tabular}{|c|ccc|}\hline
					$p$ & $<0.5$ & $0.5$ & $>0.5$ \\\hline
					skew & right & symmetrical & none \\\hline
				\end{tabular}
			\end{center}
		\section{Geometric Distributions}
			The \textbf{geometric setting} is identical to the \emph{binomial setting} apart from the fact that the number of trials is not predetermined.
			\callout{17}{
				For a \emphi{discrete} random variable to be \textbfi{geometric}, it must meet the following conditions.
				\begin{enumerate}
					\item
						The possible outcomes of each trial must be \emphi{binary}.
					\item
						Each trial must be \emphi{independent}.
					\item
						The probability of success must be constant between trials.
				\end{enumerate}
			}
			The probability of achieving success after exactly $x$ trials is the probability of $x - 1$ failures and 1 success, and is appropriately defined as such:
			\[P(x) = (p - 1)^{x - 1}p\]
			The \emph{mean} of a geometric variable\index{variable!geometric} is simply the reciprocal of $p$.
			\[\mu_X = \frac{1}{p}\]
			The \emph{standard deviation} of a geometric variable is as follows:
			\[\sigma_X = \frac{\sqrt{1 - p}}{p}\]
\end{document}