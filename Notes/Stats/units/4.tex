\documentclass[../AP_Statistics.tex]{subfiles}

\begin{document}
	\chapter{Probability}
		\textbf{Probability} is the long-run relative frequency. It must be between 0 and 1 (inclusive). \\
		The short-term is unpredictable, but the long-term is predictable. \\
		The \textbf{law of large numbers} states that as the number of trials approaches infinity, the \textbf{experimental} (observed) probability will converge to the \textbf{theoretical} (calculated) probability. \\
		A \textbf{sample space} $S$ is a list of all possible outcomes. It can be used in calculating theoretical probabilities when each outcome is equally likely.
		A \textbf{probability model} is a description of a random process. It is comprised of a sample space and a list of each outcome's corresponding probability. \\
		An \textbf{event} is any collection of outcomes. \\
		For a probability model to be valid, any individual event's probability must be within $[0, 1]$ and the sum of the probabilities of all outcomes must be equal to zero. \\
		The \textbf{complement rule} states that the probability of some event not occurring, denoted by the superscript $C$ above  is equal to 1 minus its probability of occurring. \\
	\chapter{Random Variables and Probability Distributions}
\end{document}