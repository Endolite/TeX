\documentclass[./Electricity and Magnetism.tex]{subfiles}

\begin{document}
	\section{Electrostatics}
		\subsection{Electric Charge}
			An electron is negatively charged. \\
			Two particles of the same polarity repel each other while those of opposite polarity are attracted. \\
			\textbf{Conductors} are materials in which electrons are able to move relatively freely. \textbf{Nonconductors/Insulators} are the opposite, limiting electron movement. \\
			\textbf{Semiconductors} are materials that are between conductors and insulators in terms of conductivity. \\
			\textbf{Superconductors} are perfect conductors. \\
			Atoms are comprised of positively charged protons, negatively charged electrons, and neutral (though very slightly negatively charged) neutrons. In conductors, the outermost electrons are able to move relatively freely. These mobile electrons are called \textbf{conduction electrons}.
			\textbf{Induction} describes the phenomenon of neutral conductors being attracted to charged ones. \\
			\textbf{Coulomb's law} describes the electrostatic force between two particles of charges \(q_1\) and \(q_2\) as
				\[F = \frac{1}{4\pi\varepsilon_0}\frac{|q_1||q_2|}{r^2} \tag{Coulomb's Law}\]
				where \(\varepsilon_0 \approx \SI[per-mode = fraction]{8.85d-12}{C^2/N^2.m^2}\), the \textbf{vacuum permittivity constant}. This is often rewritten as
				\[F = k\frac{|q_1||q_2|}{r^2} \tag{Coulomb's Law}\]
				where \(k = \frac{1}{4\pi\varepsilon_0} \approx \SI{8.99d9}{N.m^2/C^2}\) is the \textbf{electrostatic constant} or the \textbf{Coulomb constant}. \\
				The electrostatic force is pointed either directly towards or away from the other particle. If multiple are acting on the same particle, the net force is the \textit{vector} sum. \\
				Particles that interact through the electrostatic force form a \textit{third-law pair}. \\
			One \textbf{shell theories} hypothesizes that a shell with uniform charge density acts like a single particle at its center from the perspective of a particle outside the shell while another claims that it cancels out, providing no net force to a particle within the shell. \\
			Electric charge is \textbf{quantized}, meaning that it can only take on certain (\textit{discrete}) values. \\
			A particle's charge \(q\) can be written as \(ne\), where \(n\) is a nonzero integer and \(e \approx \SI{1.609d-19}{C}\) is the elementary charge
				\[q = ne\]
				The proton, neutron, and electron, denoted p, n, and e (or e-) respectively, have the corresponding charges \(e\), \(0\), and \(-e\). \\
			The net charge of an isolated system is always conserved.
		\subsection{Electric Fields}
			The \textbf{electric field \(\bm{\vec{E}}\)} is the vector field of the electric charge on every point in a region surrounding a charged object. It is measured in \SI{}{N/C} To measure it, a \textbf{positive} charge \(q_0\), called a \textit{test charge} is placed at a point. The electrostatic force \(F\) is then measured on the test charge. The electric field at this point is defined to be
				\[\vec{E} = \frac{\vec{F}}{q_0}\]
				The magnitude of electric field due to a point charge \(q\) at any point of distance \(r\) from said point charge is
				\[E = \frac{F}{q_0} = k\frac{|q|}{r^2}\]
			The direction vector \(\vec{d}\) of a dipole typically goes from the negative end to the positive. \\
			The \textbf{dipole moment \(\bm{\vec{p}}\)} is defined as
				\[\vec{p} = q\vec{d}\]
				The dipole moment always attempts to align with the direction of the field, making it simple to see the direction of rotation of the dipole.
			The torque \(\vec{\tau}\) on a dipole in an electric field is
				\[\vec{\tau} = \vec{p} \times \vec{E}\]
			\textbf{Linear charge density} is denoted as \(\lambda\) as is found as
				\[\dd{q} = \lambda \dd{s}\]
				for a curved rod of charge \(Q\) and length \(s\).
			A vector into the paper is denoted on one end by \(\otimes\) while one pointing out is denoted by \(\odot\). \\
			Work \(W\) is the integral of force with respect to displacement, making it
				\[W = \int_C \vec{F} \cdot \dd{\vec{r}}\]
				The work done by a conservative force is denoted by \(W_c\), a change in potential energy by \(\Delta U\), and the gravitational force by \(F_g\). It should be noted that gravity is a conservative force and that
				\[W_c = -\Delta U\]
				As such,
				\[\Delta U_g = mg(\Delta h)\]
	\section{Gauss' Law}
		The \textbf{area vector \(\bm{\text{d}{\vec{A}}}\)} for an area element on a surface is a vector with magnitude equal to area \(\dd{A}\) of the element that is perpendicular to the surface pointing outwards. \\
		The \textbf{electric flux \({\bm{\text{d}}{\Phi_E}}\)} is given by
			\[\dd{\Phi_E} = \vec{E} \cdot \dd{\vec{A}} \tag{electric flux}\]
			with units \(\SI{}{N/C.m^2}\).
			The \textbf{total flux} through a surface is found by the surface integral
				\[\Phi_E = \iint_S \vec{E} \cdot \dd{\vec{A}} \tag{total flux}\]
			Through a \textbf{closed surface} (as used in Gauss' law),
				\[\Phi_E = \oiint_S \vec{E} \cdot \dd{\vec{A}} = \oiint_S E\dd{A}\cos\varphi \tag{net flux}\]
				where \(\varphi\) is the angle between the electric field and the surface. \\
				For a uniform electric field,
				\[\Phi_E = \oiint_S E \dd{A} \cos\varphi = E\cos\varphi\oiint_S \dd{A} = EA\cos\varphi \]
			The relationship between the surface and the field can be described by flux as
			\[
				\begin{array}{|c|ccc|}\hline
					\Phi_E & < 0 & 0 & > 0 \\\hline
					\varphi & < 90^\circ & 90^\circ & > 90^\circ \\\hline
				\end{array}
			\]
		\textbf{Gauss' law} relates the net flux \(\Phi_E\) of an electric field through a closed (Gaussian) surface to the net charge \(\subt{q}{enc}\) enclosed by the surface as
			\[\varepsilon_0\Phi_E = \subt{q}{enc} \tag{Gauss' law}\]
		If excess charge is placed on a conductor, the charge will move to the surface. \\
		Everywhere inside a conductor, \(\subt{E}{net} = 0\). \\	
		A \textbf{(uniform) surface charge density \(\bm{\sigma}\)} is equal to
			\[\sigma = \frac{q}{A} \tag{uniform surface charge density}\]
		The magnitude of the electric field outside of a conductor with uniform surface charge density \(\sigma\) is
			\[E = \frac{\sigma}{\varepsilon_0} \tag{conducting surface}\]
			That outside of an insulator is
			\[E = \frac{\sigma}{2\varepsilon_0} \tag{insulator}\]
		The magnitude of the electric field produced by a uniform spherical shell of radius \(R\) is
			\[
				E = \begin{cases}
 					k\dfrac{q}{r^2} & r \ge R \\
 					0 & r < R
 				\end{cases}
			\]
		A \textbf{(uniform) volume charge density \(\bm{\rho}\)} is equal to
			\[\rho = \frac{q}{V} \tag{uniform volume charge density}\]
		Within a sphere of radius \(R\) with uniform volume charge density, the magnitude of the field is radial:
			\[E = \left(k\frac{Q}{R^3}\right)r \tag{uniform charge, field at \(r \le R\)}\]
\end{document}
