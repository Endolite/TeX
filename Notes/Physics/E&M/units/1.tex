\documentclass[./Electricity and Magnetism.tex]{subfiles}

\begin{document}
	An electron is negatively charged. \\
	Two particles of the same polarity repel each other while those of opposite polarity are attracted. \\
	\textbf{Conductors} are materials in which electrons are able to move relatively freely. \textbf{Nonconductors/Insulators} are the opposite, limiting electron movement. \\
	\textbf{Semiconductors} are materials that are between conductors and insulators in terms of conductivity. \\
	\textbf{Superconductors} are perfect conductors. \\
	Atoms are comprised of positively charged protons, negatively charged electrons, and neutral (though very slightly negatively charged) neutrons. In conductors, the outermost electrons are able to move relatively freely. These mobile electrons are called \textbf{conduction electrons}.
	\textbf{Induction} describes the phenomenon of neutral conductors being attracted to charged ones. \\
	\textbf{Coulomb's law} describes the electrostatic force between two particles of charges \(q_1\) and \(q_2\) as
		\[F = \frac{1}{4\pi\varepsilon_0}\frac{|q_1||q_2|}{r^2} \tag{Coulomb's Law}\]
		where \(\varepsilon_0 \approx \SI[per-mode = fraction]{8.85d-12}{C^2/N^2.m^2}\), the \textbf{vacuum permittivity constant}. This is often rewritten as
		\[F = k\frac{|q_1||q_2|}{r^2} \tag{Coulomb's Law}\]
		where \(k = \frac{1}{4\pi\varepsilon_0} \approx \SI{8.99d9}{N.m^2/C^2}\) is the \textbf{electrostatic constant} or the \textbf{Coulomb constant}. \\
		The electrostatic force is pointed either directly towards or away from the other particle. If multiple are acting on the same particle, the net force is the \textit{vector} sum. \\
		Particles that interact through the electrostatic force form a \textit{third-law pair}. \\
	One \textbf{shell theories} hypothesizes that a shell with uniform charge density acts like a single particle at its center from the perspective of a particle outside the shell while another claims that it cancels out, providing no net force to a particle within the shell. \\
	Electric charge is \textbf{quantized}, meaning that it can only take on certain (\textit{discrete}) values. \\
	A particle's charge \(q\) can be written as \(ne\), where \(n\) is a nonzero integer and \(e \approx \SI{1.609d-19}{C}\) is the elementary charge
		\[q = ne\]
		The proton, neutron, and electron, denoted p, n, and e (or e-) respectively, have the corresponding charges \(e\), \(0\), and \(-e\). \\
	The net charge of an isolated system is always conserved.
\end{document}
