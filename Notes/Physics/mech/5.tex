\documentclass[../AP_Physics_C.tex]{subfiles}

\begin{document}
	\section{Rotation}
		A \textbf{rigid body} rotates as a unit. \\
		The axis about which an object rotates is the \textbf{axis of rotation}. The \textbf{angular position} $\pmb{\theta}$ of this line is taken relative to a fixed direction, the \textbf{zero angular position}. \\
		\callout{17}{Although its can be changed (if specified), positive angles are conventionally \textbf{counterclockwise} from the zero angular position.}
		Angular dimension is measured using \textbf{radians} ($\pmb{\rad}$), which are dimensionless.
		\[\theta = \frac{s}{r}\]
		A \textbf{revolution} is equal to $360^\circ$ which is also equal to $2\pi\rad$.
	\section{Rolling, Torque, and Angular Momentum}
		For an object to \textbf{roll} is for it to move rotationally and translationally along a surface. For an object to roll \textbf{smoothly} is for it not to leave the ground while it is rolling. \\
		\callout{17}{Smooth rolling can be thought of as pure rotation and pure translation or as rotation about a moving contact point.}
		The center of mass of a rolling object moves parallel to the surface. The rest of the object rotates about the center of mass. \\
		The \textbf{arc distance} $\pmb{S}$, the distance covered on the surface, and the velocity about the center of mass are defined as linear variables:
		\begin{align*}
			S &= \theta r & \com{v} &= \omega r
		\end{align*}
		As rolling objects move both translationally and rotationally, they have both translational and rotational kinetic energy.
		\[K = \frac{1}{2}\com{I}\omega^2 + \frac{1}{2}M\com{v}^2\]
		If no slipping occurs, then energy is conserved (even with friction).
		The acceleration about the center of mass follows the pattern of position and velocity.
		\[\com{a} = \alpha r\]
		For an object to roll smoothly on a slope, three things are required are required:
		\begin{enumerate}
			\item
				The gravitational force must be vertically down.
			\item
				The normal force must be perpendicular to the slope.
			\item
				The force of friction must point up the slope.
		\end{enumerate}
		The acceleration of a body rolling smoothly down a slope can be found as such:
		\[\comd{a}{,x} = \frac{-g\sin\theta}{I + \frac{\com{I}}{Mr^2}}\]
\end{document}