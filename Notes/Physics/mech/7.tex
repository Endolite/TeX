\documentclass[../AP_Physics_C]{subfiles}

\begin{document}
	\textbf{Gravitation} is the tendency for bodies to attract each other. \\
	The gravitational force is dependent on mass. It is always attractive. \\
	Newton's \textbf{law of gravitation} defines the strength of the attractive force between particles.
	\[F_G = G\frac{m_1m_2}{r^2}\]
	$G$ is the \textbf{gravitational constant}
	\[G \approx 6.67 \times 10^{-11} \,\N\cdot\m^2/\kg^2\]
	Newton's law of gravitation can be rewritten in vector form using $\hat{r}$, which is a vector of magnitude 1 that points from one particle to another.
	\[\vec{F}_G = G\frac{m_1m_2}{r^2}\hat{r}\]
	As is made clear by this vector form, gravitational forces always come in \emph{third-law pairs}. \\
	The \textbf{shell theorem} states that a uniform spherical shell of matter attracts a particle that is outside the shell as though all of the shell's mass was concentrated at its center. \\
	The net gravitational force can be calculated by the \textbf{principle of superposition}, which simply states that the net effect is the sum of the individual effects.
	\[\vec{F}_{\net} = \sum \vec{F}_i\]
	For a real (extended) object, this sum becomes an integral.
	\[\vec{F}_{\net} = \int\d\vec{F}\]
	If the object in question is a uniform sphere or shell, it can be treated as having its mass at its center rather than considering the object as a whole. \\
	Newton's law of gravitation can be combined with Newton's second law to calculate the acceleration due to gravity.
	\[a_{G,1} = \frac{F_g}{m_1} = \frac{Gm_1m_2/r^2}{m_1} = \frac{Gm_2}{r^2}\]
\end{document}