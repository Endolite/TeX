\documentclass[../AP_Physics_C]{subfiles}

\begin{document}
	\textbf{Gravitation} is the tendency for bodies to attract each other. \\
	The gravitational force is dependent on mass. It is always attractive. \\
	Newton's \textbf{law of gravitation} defines the strength of the attractive force between particles.
	\[F_G = \frac{Gm_1m_2}{r^2}\]
	$G$ is the \textbf{gravitational constant}
	\[G \approx 6.67 \times 10^{-11} \,\N\cdot\m^2/\kg^2\]
	Newton's law of gravitation can be rewritten in vector form using $\hat{r}$, which is a vector of magnitude 1 that points from one particle to another.
	\[\vec{F}_G = \frac{Gm_1m_2}{r^2}\hat{r}\]
	As is made clear by this vector form, gravitational forces always come in \emph{third-law pairs}. \\
	The \textbf{shell theorem} states that a uniform spherical shell of matter attracts a particle that is outside the shell as though all of the shell's mass was concentrated at its center. \\
	The net gravitational force can be calculated by the \textbf{principle of superposition}, which simply states that the net effect is the sum of the individual effects.
	\[\vec{F}_{\net} = \sum \vec{F}_i\]
	For a real (extended) object, this sum becomes an integral.
	\[\vec{F}_{\net} = \int\d\vec{F}\]
	If the object in question is a uniform sphere or shell, it can be treated as having its mass at its center rather than considering the object as a whole. \\
	Newton's law of gravitation can be combined with Newton's second law to calculate the acceleration due to gravity.
	\[a_{G,1} = \frac{F_g}{m_1} = \frac{Gm_1m_2/r^2}{m_1} = \frac{Gm_2}{r^2}\]
	An additional implication of the shell theorem is that no net gravitational force is exerted on a particle within the shell. (The forces still exist, but balance out.) \\
	The \textbf{gravitational potential energy} of a two-particle system is as follows:\footnote{The formula for gravitational potential energy is the integral of that of the force of gravity with respect to $r$.\[U_G = \int F_G\,\d r = \int \frac{Gm_1m_2}{r^2}\,\d r = \frac{-Gm_1m_2}{r}\]}
	\[U_G = \frac{-Gm_1m_2}{r}\]
	The gravitational potential energy of a system is the sum of all the gravitational potential energies of every possible pair of particles.
	Gravitational potential energy is a property of a pair of particles. The potential energy cannot be allocated to the individual particles. \\
	When one object is far more massive than the other, the smaller object can be given all of the gravitational potential energy (for practical calculations). \\
	For a projectile to escape a body's gravitational pulls, it must come to rest only at infinity, if at all. This means that $K$ must both be 0, and $U$ is 0 because $r$ is $\infty$. Their sum must therefore be at least 0 at the surface of the body.
	\begin{align*}
		K + U = 0 &=\frac{1}{2}m_1v^2 + \left(\frac{-Gm_1m_2}{r}\right) \\
			v &= \sqrt{\frac{2Gm_2}{r}}
	\end{align*}
	\section{Kepler's Laws of Planetary Motion}
		\subsection{The Law of Orbits}
			All planets in the solar system move in elliptical orbits with the Sun as a focus. \\
			An orbit is defined by its \textbf{semimajor axis} $\bm{a}$ and its \textbf{eccentricity} $\bm{e}$.
		\subsection{The Law of Areas}
			A line connecting a planet to the Sun sweeps out equal areas in the plane of the planet's orbit for an equal time interval; that is, the rate $\frac{\d A}{\d t}$ at which is sweeps out an area $A$ is constant.
		\subsection{The Law of Periods}
			The sqaure of the period of a planet's orbit is proportional to the cube of its semimajor axis.
			\[T^2 \propto a^3\]
			More concretely, the law of periods can be written as such:
			\[T^2 =\frac{4\pi^2a^3}{GM}\]
	\section{Satellites}
		The total mechanical energy of a circular orbit can be described as such:\footnote{
			The gravitational force can be related to the centripetal acceleration of a satellite to derive its kinetic energy.
			\begin{align*}
				a_c &= \frac{v^2}{r} = \frac{F_G}{m} = \frac{GMm}{mr^2} \\
				v^2 &= \frac{GM}{r} \\
				K &= \frac{1}{2}mv^2 = \frac{1}{2}\left(\frac{GMm}{r}\right) = \frac{GMm}{2r} \\
					&= \frac{-U_G}{2}
			\end{align*}
			The net mechanical energy can then be found.
			\[E = K + U = \frac{-U}{2} + U = \frac{U}{2} = \frac{-GMm}{2r} = -K\]
		}
		\[E = -K = \frac{-GMm}{2r}\]
		For an elliptical orbit, the semimajor axis can be used in place of the radius, as the eccentricity of the orbit has of no bearing on the total kinetic energy.
		\[E = \frac{-GMm}{2a}\]
\end{document}