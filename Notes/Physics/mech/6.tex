\documentclass[../AP_Physics_C.tex]{subfiles}

\begin{document}
	\section{Equilibrium and Elasticity}
		An object in \textbf{static equilibrium} have a center of mass with constant linear momentum and have constant angular momentum about any point.
		\[\text{equilibrium} \implies (\vec{P} \land \vec{L}) \text{ are constant}\]
		If a body returns to static equilibrium after a slight displacement, it is in \textbf{stable static equilibrium}. Otherwise, it is \textbf{unstable}. \\
		The requirements for equilibrium can be rewritten using Newton's second law.
		\[\text{equilibrium} \implies \vec{F}_{\net}, \vec{\Tau}_{\net} = 0 \]
		Forces are often only considered in the $xy$ plane, further simplifying the requirements.
		\[\text{equilibrium} \implies F_{\net,x}, F_{\net, y}, \tau_{\net, z} = 0\]
		\subsection{Center of Gravity}
			The gravitational force $\vec{F}_g$ on a body is the sum of all of the gravitational forces acting on its individual particles. This can be simplified by treating it as as single force that acts on the \textbf{center of gravity} (\textbf{cog}).
			If the gravitational force is the same for all of a body's particles, then the center of gravity is simply the center of mass.
\end{document}