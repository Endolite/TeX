\documentclass[../AP_Physics_C/mech]{subfiles}

\begin{document}
		\textbf{Kinematics} is the classification and comparison of motion. \\
	A \textbf{particle} is defined as a point-like object, moving in a way such that every part of it moves at the same rate (without rotation or stretching). \\
	The subscripts $f$ and $i$ (or $0$) below a quantity can be used to denote initial or final values for it, while numbers can be used if more than two variations are being considered. \\
	The subscript $\mathrm{avg}$ is used to denote a quantity's average over some domain. \\
	\textbf{Position} must be measured relative to a reference point (typically the origin). The variables $x$, $y$, and $z$ are used depending on the context to denote its components and the vector quantity is generally denoted $\vec{r}$, while \emph{generalized} position is often denoted with $q$. \\
	A change in position is a \textbf{displacement}. The displacement in a direction is denoted with a preceding $\Delta$ (meaning \enquote{change in}, though depending on the context, the variable $d$ can instead be used in isolation. \\
	\[\Delta q  = q_f - q_i \title\]
	The rate of change of position is \textbf{velocity}. \emph{Average} velocity is equal to displacement divided by the change in time, while \emph{instantaneous} velocity is the time derivative of position. Velocity is denoted using $v$ while generalized velocity is denoted as the time derivative of generalized position, adding a dot above. (The use of dots as time derivatives of generalized position applies to any number of dots.)
	\begin{align*}
		a
	\end{align*}
\end{document}