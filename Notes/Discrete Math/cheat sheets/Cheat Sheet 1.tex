\documentclass[12pt, A4, twocolumn]{article}

% Packages
	% Basics
		\usepackage{amsmath}
		\usepackage{bm}
		\usepackage{cellspace}
		\usepackage{csquotes}
		\usepackage[hang,flushmargin]{footmisc}
		\usepackage[margin=0.75in]{geometry}
		\usepackage{hyperref}
		\usepackage[utf8]{inputenc}
	% Diagrams
		\usepackage{pgfplots}
		\usepackage{tikz}
		\usepackage{tikz-3dplot}
			\usetikzlibrary{arrows.meta, angles, calc, quotes}
	% Symbols
		\usepackage{amssymb} % Miscellaneous
% Configuration
	\title{Test 1 Cheat Sheet}
	\author{Arnav Patri}
% Macros
	% Notation
		% Operators
			\DeclareMathOperator{\divr}{div}
		% Sets
			\newcommand{\N}{\mathbb{N}}
			\newcommand{\R}{\mathbb{R}}
			\newcommand{\Z}{\mathbb{Z}}
		% Other
			\DeclareMathOperator{\avg}{avg}
			\renewcommand{\mod}{\text{mod}}
			\DeclareMathOperator{\return}{\text{return}}
			\renewcommand{\th}{\text{th}}
			\DeclareMathOperator{\while}{\text{while}}
	% Utilities
		\newcommand{\callout}[2]{\begin{center}\fbox{\begin{minipage}{#1cm}#2\end{minipage}}\end{center}}
		\newcommand{\comment}[1]{}
		\newcommand{\subsectionb}[1]{\subsection*{#1}\addcontentsline{toc}{subsection}{#1}}
\begin{document}
	\section*{Chapter 4}
		\subsubsection*{Divisibility Rules}
	\section*{Chapter 6}
		\subsection*{Counting Rules}
			\begin{itemize}
				\item
					\textbf{Product Rule:} If a procedure can be decomposed into a sequence of two tasks, one with \(n_1\) possible ways of being completed and another with \(n_2\) ways, there are \(n_1n_2\) total ways to carry out the procedure.
				\item
					\textbf{Sum Rule:} If a task can be completed either in one of \(n_1\) ways or in one of \(n_12\) ways, where there is no overlap between the sets of \(n_1\) and \(n_2\) ways, then there are \(n_1 + n_2\) ways to complete the task.
				\item
					\textbf{Subtraction Rule:} If a task can be completed in either \(n_1\) or \(n_2\) ways, then the number of ways to do the task is \(n_1 + n_2\) minus the number of ways that are shared between both.
				\item
					\textbf{Division Rule:} If a task can be done using a procedure that can be carried out \(n\) ways and exactly \(d\) of \(n\) ways correspond to every way, there are \(n/d\) ways to complete the task.
			\end{itemize}
		\subsection*{Permutations and Combinations}
			\(n, r \in \Z^+, r \le n\)
			\[P(n, r) = \frac{n!}{(n - r)!} \,\,\,\, C(n, r) = \binom{n}{r} = \frac{n!}{r!(n - r)!}\]
			\[\def\arraystretch{2}\begin{tabular}{|c|c|c|}\hline
				Type & Repetition? & Formula \\\hline
				\(r\)-permutations & N & \(\dfrac{n!}{(n - r)!}\) \\
				\(r\)-combinations & N & \(\dfrac{n!}{r!(n - r)!}\) \\
				\(r\)-permutations & Y & \(n^r\) \\
				\(r\)-combinations & Y & \(\dfrac{(n + r - 1)!}{r!(n - 1)!}\) \\\hline
			\end{tabular}\]
		\subsection*{Boxes}
			\begin{itemize}
				\item
					\textbf{Distinguishable Objects, Distinguishable Boxes}
						\[\frac{n!}{n_1!n_2!\cdots n_k!} = \frac{n!}{\prod\limits_{i = 1}^k n_i!}\]
				\item
					\textbf{Indistinguishable Objects, Distinguishable Boxes}
						\[C(n + r - 1, r) = \frac{(n + r - 1)!}{r!(n - 1)!}\]
				\item
					\textbf{Distinguishable Objects, Indistinguishable Boxes}
						\[\sum_{j = 1}^k \frac{1}{j!} \sum_{i = 0}^{j - 1} (-1)^i\binom{j}{i}(j - i)^n\]
			\end{itemize}
		\subsection*{Binomials}
			\textbf{Binomial Theorem}
			\(n \in \N\)
				\[(x + y)^n = \sum_{i = 0}^n \binom{n}{i} x^{n - i}y^i\]
			\textbf{Pascal's Identity}
				\(n, k \in \Z^+, k \le n\)
				\[\binom{n + 1}{k} = \binom{n}{k - 1} + \binom{n}{k}\] \\
			\textbf{Vandermonde's Identity}
				\(m, n, r \in \N, r \le m, n\)
				\[\binom{m + n}{r} = \sum_{k = 0}^r \binom{m}{r - k}\binom{n}{k}\]
	\section*{Chapter 5}
		\subsection*{Induction}
			\textbf{Principle of Mathematical Induction} (\(\Z^+\))
				\[(P(1) \land \forall k(P(k) \Rightarrow P(k + 1))) \Rightarrow \forall n P(n)\]
			\textbf{Proofs}
				\begin{enumerate}
					\item
						Express the statement to be proven as \enquote{for all \(n \ge b\), \(P(n)\)} for fixed integer \(b\).
					\item
						Show \(P(b)\) is true (basis).
					\item
						Identify inductive hypothesis as \enquote{Assume that \(P(k)\) is true for an arbitrary fixed integer \(k \ge b\)}.
					\item
						State what must be proven under the assumption to prove the hypothesis' validity.
					\item
						Prove that \(P(k + 1)\) is true under the assumption (inductive).
					\item
						Identify the conclusion of the inductive step.
					\item
						State the conclusion that \enquote{by mathematical induction, \(P(n)\) is true for all integers \(n\) with \(n > b\)}.
				\end{enumerate}
\end{document}
