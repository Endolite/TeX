\documentclass[./Discrete Math.tex]{subfiles}

\begin{document}
	\section{Graphs and Graph Models}
		\callout{17}{
			A \textit{graph} \(G = (V, E)\) is comprised of \(V \not\equiv \varnothing\), a set of vertices, and, and a set of edges \(E\). Each edge is associated with either 1 or 2 \textit{endpoints}. An edge is said \textit{connect} to its endpoints.
		}
		It should be noted that \(V\) or \(E\) may be infinite. If both are infinite, the graph is considered an \textbf{infinite graph}. If both are finite, the graph is called a \textbf{finite graph}. \\
		A graph in which each edge connects two different vertices and no two edges connect the same pair of vertices is called a \textbf{simple graph}. \\
		Graphs with \textbf{multiple edges} that connect the same vertices are called \textbf{multigraphs}.
\end{document}
