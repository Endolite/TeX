\documentclass[./Discrete Math.tex]{subfiles}

\begin{document}
	\section{Graphs and Graph Models}
		\callout{17}{
			A \textit{graph} \(G = (V, E)\) is comprised of \(V \not\equiv \varnothing\), a set of vertices, and, and a set of edges \(E\). Each edge is associated with either 1 or 2 \textit{endpoints}. An edge is said \textit{connect} to its endpoints.
		}
		It should be noted that \(V\) or \(E\) may be infinite. If both are infinite, the graph is considered an \textbf{infinite graph}. If both are finite, the graph is called a \textbf{finite graph}. \\
		A graph in which each edge connects two different vertices and no two edges connect the same pair of vertices is called a \textbf{simple graph}. \\
		Graphs with \textbf{multiple edges} that connect the same vertices are called \textbf{multigraphs}. \\
		An unordered pair of vertices \(\{u, v\}\) is said to be of multiplicity \(m\) if there are \(m\) different edges associated with it. \\
		An edge connecting a vertex to itself is called a \textit{loop}.
		Graphs with loops or multiple edges connecting the same pair of vertices is sometimes called a \textbf{psuedograph}. \\
		\textbf{Undirected graphs} have \textbf{undirected} edges.
		\callout{17}{
			A \textit{directed graph} or \textit{digraph} \((V, E)\) is comprised of a set of vertices \(V \not\equiv \varnothing\) and a set of \textit{directed edges (arcs)} \(E\). Each directed edge is associated with an ordered pair of vertices. That associated with \((u, v)\) is said to \textit{start} at \(u\) and \textit{end} at \(v\).
		}
		A directed graph without loops or multiple directed edges is a \textbf{simple directed graph}. \\
		A \textbf{directed multigraph} have \textbf{multiple directed edges} between to vertices (or possibly the same vertex). \\
		An ordered pair of vertices \((u, v)\) is said to be of multiplicity \(m\) if there are \(m\) directed edges associated with it. \\
		A \textbf{mixed graph} has both directed and undirected edges. \\
		Two vertices in an undirected graph are \textit{adjacent} (or \textit{neighbors} if there is an edge connecting them. Such an edge is called \textit{incident with} the vertices and is also said to \textit{connect} them. \\
		The set of all neighbors of a vertex \(v\) is denoted by \(N(v)\) and is called the \textit{neighborhood} of \(v\). If \(A\) is a subset of \(V\), \(N(A)\) is denotes the set of all vertices in \(G\) that are adjacent to at least one vertex in \(A\), so \(N(A) = \bigcup\limits_{v \in A} N(v)\). \\
		The \textit{degree} of a vertex in an undirected graph is the number of edges that are incident with. A loop contributes 2 to a vertex's degree. This is denoted by \(\deg v\).
	\setcounter{section}{2}
	\section{Representing Graphs and Graph Isomorphism}
		\setcounter{subsection}{2}
		\subsection{Adjacency Matrices}
			Let \(G(V, E)\) be a graph. The \textbf{adjacency matrix A} (or \(\textbf{A}_G\)) of \(G\) is the \(|V| \times |V|\) matrix where \(\textbf{A}_{G, i, j}\) is the number of edges connecting vertices \(i\) and \(j\).
\end{document}
