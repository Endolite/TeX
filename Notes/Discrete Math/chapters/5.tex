\documentclass[./Discrete Math.tex]{subfiles}

\begin{document}
	\section{Mathematical Induction}
		\setcounter{subsection}{1}
		\subsection{Mathematical Induction}
			Mathematical induction\footnote{In logic, \textbf{deductive reasoning} uses inference to draw conclusions from premises while \textbf{inductive reasoning} draws conclusions that are supported by not ensured by the evidence. Mathematical proofs, including those that employ induction, are deductive.} can be used to prove statements asserting that a propositional function \(P(n)\) is true for all positive integers \(n\).
			\callout{17}{\paragraph{Principle of Mathematical Induction}
				In order to prove that \(P(n)\) is true for all positive integers \(n\), two steps must be completed: 
				\begin{enumerate}
					\item
						The \textbf{basis step} must verify that \(P(1)\) is true.
					\item
						The \textbf{inductive step} must show that \(P(k) \Rightarrow P(k + 1)\) is true for all positive integers \(k\).
				\end{enumerate}
			}
			To complete the inductive step, it is assumed that \(P(k)\) is true for an arbitrary positive integer \(k\) and that this assumption guarantees that \(P(k + 1)\) is true as well. This assumption is called the \textbf{inductive hypothesis}. \\
			The inductive step shows that \(\forall k(P(k) \Rightarrow P(k + 1))\) is true where the domain is \(\Z^+\). \\
			Expresses as a rule of inference, this proof technique can be written as
				\[(P(1) \land \forall k(P(k) \Rightarrow P(k + 1))) \Rightarrow \forall nP(n)\]
				with the domain \(\Z^+\).
		\setcounter{subsection}{4}
			\subsection{Guidelines for Proofs by Mathematical Induction}
				\callout{17}{\paragraph{Template for Proofs by Mathematical Induction}
					\begin{enumerate}
						\item
							Express the statement to be proven in the form of \enquote{for all \(n \ge b\), \(P(n)\)} for a fixed integer \(b\).
						\item
							Denote the basis step, showing that \(P(b)\) is true.
						\item
							Identify the inductive hypothesis in the form \enquote{Assume that \(P(k)\) is true for an arbitrary fixed integer \(k \ge b\)}.
						\item
							Sate what must be proven under the assumption in order to prove the validity of the inductive hypothesis.
						\item
							Prove the statement \(P(k + 1)\) under the assumption.
						\item
							Identify the conclusion of the inductive step.
						\item
							State the conclusion that \enquote{by mathematical induction, \(P(n)\) is true for all integers \(n\) with \(n \ge b\)}.
					\end{enumerate}
				}
	\setcounter{section}{2}
	\section{Recursive Definitions and Structural Induction}
		\setcounter{subsection}{1}
		\subsection{Recursively Defined Functions}
			A function with the set of nonnegative integers as its domain can be defined by a \textbf{basis step}, setting the value of the function at 0, and a \textbf{recursive step}, providing a rule for finding its value at an integer from its values at smaller integers. This describes a \textbf{recursive/inductive definition}. \\
			Recursively defined functions are \textbf{well-defined}, meaning that for every positive integer, the corresponding function value is unambiguously determined.
		\setcounter{subsection}{2}
		\subsection{Recursively Defined Sets and Structures}
			Recursive definitions may include an \textbf{exclusion rule}, excluding all elements other than those specified by the basis step of those generated by the rule.
			\callout{11.67}{
				The set \(\Sigma^*\) of strings over the alphabet \(\Sigma\) is defined recursively as
				\begin{enumerate}
					\item
						\(\lambda \in \Sigma^*\), where \(\lambda\) is an empty string.
					\item
						If \(w \in \Sigma^*\) and \(x \in \Sigma\), then \(wx \in \Sigma^*\).
				\end{enumerate}
			}
			\callout{17}{
				\textit{Concatenation}, denoted by \(\cdot\) is an operation by which two strings can be combined. It is defined as follows:
				\begin{enumerate}
					\item
						If \(w \in \Sigma^*\), then \(w \cdot \lambda = w\).
					\item
						If \(w_1, w_2 \in \Sigma^*\) and \(x \in \Sigma\), then \(w_1 \cdot w_2x = (w_1 \cdot w_2)x\)
				\end{enumerate}
			}
			\callout{17}{
				A \textit{rooted tree} consists of a set of vertices containing a distinguished vertex known as the \textit{root} and edges connecting the vertices. The set of all rooted trees can be defined as
				\begin{enumerate}
					\item
						A single vertex \(r\) is a rooted tree.
					\item
						Suppose \(T_1, T_2, \ldots, T_n\) are disjoint rooted trees with respective roots \(r_1, r_2, \ldots, r_n\). The graph formed by adding a vertex from the root \(r\), which is not part of any of the trees, to each of the roots is also a rooted tree.
				\end{enumerate}
			}
\end{document}
