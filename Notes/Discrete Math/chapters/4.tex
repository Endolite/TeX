\documentclass[./Discrete Math.tex]{subfiles}

\begin{document}
	
	\setcounter{section}{1}
	\section{Integer Representations and Algorithms}
		\callout{17}{\paragraph{Definition of a Number}
			A number is dependent on a given base and its place value and digits.
		}
		\setcounter{subsection}{1}
		\subsection{Representations of Integers}
			A base \(b\) has \(b - 1\) digits. The first digit from the right is multiplied by \(b^0\), the second by \(b^1\), and so on. The number itself is the sum of each digit multiplied by \(b\) raised to the power of its respective place value. \\
			0 is a member of every base (except sometimes base 1).
			\callout{17}{
				Let \(b\) be an integer greater than 1. If \(b\) is an integer greater than 1 and \(n\) is positive, then \(n\) can be expressed uniquely in the form
					\[n = a_kb^k + a_{k - 1}b^{k - 1} + \cdots + a_1b + a_0\]
			}
			A number in base \(b\) is denoted by \((n)_b\).
			\callout{17}{
				A number is a linear combination of its digits and their place values.
			}
			\callout{17}{\paragraph{Constructing Base \(b\) Expansions}
				Given an integer \(n\) to be represented in base \(b\),
				\begin{align*}
					&q := n \\
					&k := 0 \\
					&\while\, q \ne 0 \\
						 &\qquad a:= a \,\mod\, b \\
						 &\qquad q := q \divr b \\
						 &\qquad k := k + 1 \\
					&\return\, (a_{k - 1}, \ldots, a_1, a_0) \{(a_{k - 1} \ldots a_1a_0)_b \text{ is the base \(b\) expansion of \(n\)\}} 
				\end{align*}
			}
			A number in its own base is always represented as 10. \\
			Addition and multiplication in base \(b\) follows the same conventions as that of base 10. \\
			To add two numbers \(a\) and \(b\) in base 2, their rightmost bits \(a_0\) and \(b_0\) can be added such that \\
				\[a_0 + b_0 = 2c_0 + s_0\]
				where \(s_0\) is \(s_0\) is the rightmost bit of the binary expansion of the sum and \(c_0\) is the \textbf{carry}, being either 0 or 1. This process can be repeated. \\
				\[c_0 = \frac{a_0 + b_0 - s_0}{2}\]
	\section{Primes and Greatest Common Divisors}
		\setcounter{subsection}{1}
		\subsection{Primes}
			\callout{17}{A \textbf{prime number} is a whole number whose only factors are 1 and itself. By definition, it does not appear on the multiplication table. A nonprime positive integer is called \textbf{composite}}
			\callout{17}{\paragraph{The Fundamental Theorem of Arithmetic}
				Every integer greater greater than 1 can be written uniquely as the the product of one or more primes.
			}
			Two numbers are relatively prime or coprime if their greatest common factor (GCF) is 1. \\
			If \(n\) is divisible by \(a\) and \(b\), then it is also divisible by \(a \times b\).
	\setcounter{section}{0}
	\section{Divisibility and Modular Arithmetic}
		\setcounter{subsection}{1}
		\subsection{Division}
			\callout{17}{
				If \(a\) and \(b\) are nonzero integers such that \(\frac{b}{a}\) is an integer, it is said that \(a\) \textit{factor/divisor} of \(b\) and that \(b\) is a multiple of \(a\). This is denoted as \(a \mid b\). If \(a\) is not a factor of \(b\), it is denoted as \(a \not{\mid} \,\, b\).
			}
			\callout{17}{
				Let \(a\), \(b\), and \(c\) be nonzero integers.
				\begin{enumerate}
					\item
						If \(a \mid b\) and \(b \mid c\), then \(a \mid (b + c)\).
					\item
						If \(a \mid b\), then \(a \mid bc\) for any integer \(c\).
					\item
						If \(a \mid b\) and \(b \mid c\), then \(a \mid c\).
				\end{enumerate}
			}
		\subsection{The Division Algorithm}
			\callout{17}{\paragraph{The Division Algorithm}
				Let \(a\) and \(b\) be integers, the latter of which is positive. Then there are unique integers \(q\) and \(r\), with \(0 \le r < d\), such that \(a = dq + r\).
			}
			In this equality, \(d\) is called the \textit{divisor}, \(a\) the \textit{dividend}, \(q\) the \textit{quotient}, and \(r\) the \textit{remainder}. The notation used is
				\[q = a \divr d \qquad r = a \bmod d\]
		\subsection{Modular Arithmetic}
			\callout{17}{
				If \(a\) and \(b\) are integers and \(m\) is a positive integer, then \(a\) is \textit{congruent} to \(b\) \textit{modulo} \(m\) if \(m \mid (a - b)\). The notation \(a \equiv b \pmod m\) to denote this \textbf{congruence} in \textbf{modulo} \(m\), \(m\) being the \textbf{modulus}. An incongruency is denoted \(a \not\equiv b \pmod m\)
			}
			\callout{8.86}{
				\(a \equiv b \pmod m\) if and only if \(a \bmod m = b \bmod m\)
			}
			\callout{17}{
				Let \(m\) be a positive integer. \(a\) is congruent modulo \(m\) to \(b\) if there exists an integer \(k\) such that \(a = b + km\).
			}
			\callout{17}{
				Let \(m\) be a positive integer. If \(a \equiv b\) and \(c \equiv d\) modulo \(m\), \(a + c \equiv b + d\) and \(ac = bd\) modulo \(m\) as well.
			}
		\subsection*{Divisibility Rules}
			\begin{enumerate}
				\setcounter{enumi}{6}
				\item
					If the difference between a 2 times a number's last digit and the rest of the number is divisible by 7 or 0, the number is as well.
				\setcounter{enumi}{17}
					If the difference between a number's last digit multiplied by 5 and the rest of the numbers is divisible by 17 or 0, the number is divisible by 17.
				\setcounter{enumi}{18}
				\item
					If the sum of 2 times the last digit of a number and the rest of the digits is divisible by 19, the number is divisible by 19.
				\setcounter{enumi}{22}
				\item
					If the sum of 7 times the last digit of a number and the rest of the number is divisible by 23, then so is the number.
				\setcounter{enumi}{30}
				\item
					If the difference between 3 times the last digit of a number and the rest of the number is divisible by 31, then so is the number.
			\end{enumerate}
\end{document}
