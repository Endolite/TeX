\documentclass[./Differential Equations.tex]{subfiles}

\begin{document}
	\section{Linear Models: Initial-Value Problems}
		\subsection{Spring/Mass Systems: Free Undampened Motion}
			\subsubsectionb{Hooke's Law}
				Suppose a rigid body of is attached to a flexible spring. The spring force \(F_s\) is proportional to the displacement \(s\) of the body from its equilibrium position and is in the direction of equilibrium; that is
					\[F_s = -ks\]
			\subsubsectionb{Newton's Second Law}
				Newton's second law of motion states that
					\[\subt{F}{net} = ma\]
					where \(m\) is mass and \(a\) is acceleration or \(\dv*[2]{x}{t}\). \\
				If the mass on a spring vibrates without regard to any external forces, having \textbf{free motion}, and the mass is hanging vertically from the spring, then Newton's second law gives
					\[
						m\dv[2]{x}{t} = \overset{\subt{F}{net}}{\overbrace{-(x + s) + mg}}
							= -kx + \overset{0}{\overbrace{mg - ks}} 
							= -kx
					\]
					where \(s\) is the equilibrium position (where \(ks = mg\)) and \(x\) is the displacement from equilibrium.
			\subsubsectionb{DE of Free Undampened Motion}
				Dividing by \(m\) yields the second-order DE
					\[\dv[2]{x}{t} + \omega^2x = 0\]
					where \(\omega^2 = k/m\). This equation is said to describe \textbf{simple harmonic motion (SHM)} or \textbf{free undampened motion}.
			\subsubsectionb{Equation of Motion}
				The auxiliary equation of the SHM DE is
					\[m^2 + \omega^2 = 0\]
					so the solutions are
					\[m = \pm i\omega\]
					making the general solution
					\[x(t) = C_1\cos(\omega t) + C_2\sin(\omega t)\]
				The \textbf{period} of motion \(\bm{T}\) (in \(\SI{}{s}\)), the amount of time it takes for a full oscillation to occur, is
					\[T = \frac{2\pi}{\omega}\]
					The \textbf{frequency} of motion \(\bm{f}\) (in \(\SI{}{s^{-1}}\) or \(\SI{}{hz}\)) is
					\[f = \frac{1}{T} = \frac{\omega}{2\pi}\]
					The number
					\[\omega = \sqrt{\frac{k}{x}}\]
					(in \(\SI{}{rad/s}\)) and \(f\) are both sometimes referred to as the \textbf{natural frequency} of the system. \\
					The equation derived by solving for the constants of the general solution is the \textbf{equation of motion} of the system.
			\subsubsectionb{Alternative Forms of \(\bm{x(t)}\)}
				When \(C_1, C_2 \ne 0\), the \textbf{actual amplitude \(\bm{A}\)} of free vibrations is not immediately obvious, so it is often convenient to convert the equation of SHM to the simpler form
					\[x(t) = A\sin(\omega t + \varphi)\]
					where
					\[A = \sqrt{C_1^2 + C_2^2}\]
					and \(\varphi\) is a \textbf{phase angle} defined by
					\[
						\left.\begin{aligned}
							\sin\varphi &= \frac{C_1}{A} \\
							\cos\varphi &= \frac{C_2}{A}
						\end{aligned}\right\}
						\tan\varphi = \frac{C_1}{C_2}
					\]
				A cosine function is sometimes preferred, making the solution
					\[x(t) = A\cos(\omega t + \varphi)\]
					where \(\varphi\) is defined by
					\[
						\left.\begin{aligned}
							\sin\varphi &= \frac{C_2}{A} \\
							\cos\varphi &= \frac{C_1}{A}
						\end{aligned}\right\}
						\tan\varphi = \frac{C_2}{C_1}
					\]
			\subsubsectionb{Double Spring Systems}
				The \textbf{effective spring constant \(\bm{\subt{k}{eff}}\)} of a system with two \textit{parallel} springs with spring constants \(k_1\) and \(k_2\) is 
					\[\subt{k}{eff} = k_1 + k_2\]
				That of a system with two \textit{series} springs is
					\[\subt{k}{eff} = \frac{k_1k_2}{k_1 + k_2}\]
			\subsubsectionb{Systems with Variable Spring Constants}
				In reality, it is reasonable to expect the spring constant to decay over time. One model for the \textbf{aging spring} replaces the spring constant \(k\) with the decreasing function
					\[K(t) = k\en^{-\alpha t}\]
					where \(k\) and \(\alpha\) are positive constants. The linear DE
					\[mx'' + k\en^{-\alpha t}x = 0\]
					cannot be solved with the methods discussed thus far. \\
				When a spring/mass system is subject to a rapidly decreasing temperature, \(k\) may be replaced with \(K(t) = kt\) where \(k\) is a positive constant, a function that increases with time. The resulting model
				\[mx'' + ktx = 0\]
				is a form of \textbf{Airy's differential equation}.
		\subsection{Spring/Mass Systems: Free Dampened Motion}
			\subsubsection{DE of Free Dampened Motion}
				Damping forces
\end{document}