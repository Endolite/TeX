\documentclass{subfiles}

\begin{document}
	\section{Classical Relativity}
		A \enquote{theory of relativity} is simply a way for observers in different reference frames to compare their observations. \\
		The mathematical basis for converting between reference frames is called a \textit{transformation}. \\
		Consider two observes \(O\) and \(O'\) observing the same event, \(O'\) moving relative \(O\) with constant velocity \(\vec{u}\). In their own reference frames \(O\) and \(O'\) are both at rest, but they move relative to one another with constant velocity \(\vec{u}\). According to \(O\), the space and time coordinates are \(x\), \(y\), \(z\), and \(t\), while according to \(O'\), those of the \textit{same event} are \(x'\), \(y'\), \(z'\), and \(t'\). If the relative velocity is only in the common \(xx'\) direction, \(\vec{u}\) can represent the velocity of \(O'\) as measured by \(O\). \\
		It is assumed that each observer is able to test and verify Newton's laws in their reference frames. A reference frame that follows Newton's first law (the law of inertia) is said to be an \textit{inertial frame}. In order for a frame to be inertial, it must not be accelerating. \\
		The classical \textit{Galilean} transformation relating the coordinates assumes as a postulate of classical physics that \(t = t'\); that is, time is the same for all observers. It is also assumed for simplicity that the coordinate systems are chosen such that their origins overlap at \(t = 0\). Consider an object in \(O'\) at coordinates \((x', y', z')\). According to \(O\), the \(y\) and \(z\) coordinates are the same as those observed in \(O'\). Along the \(x\) direction, though, \(O\) would observe the object at \(x = x' + ut\). This yields the \textit{Galilean coordinate transformation}
			\begin{align*}
				x' &= x - ut &
					y' &= y &
					z' &= z &
					t' &= t
						\tag{Galilean coordinate transformation}	
			\end{align*}
			The velocities as observed by \(O\) can be found simply by differentiating the prior results:
			\begin{align*}
				v_x' &= v_x - u &
					v_y' &= v_y &
					v_z' &= v_z	
						\tag{Galilean velocity transformation}
			\end{align*}
			Again differentiating yields the relationships between the accelerations:
			\begin{align*}
				a_x' &= a_x &
					a_y' &= a_y &	
					a_z' &= a_z
			\end{align*}
			This shows that Newton's laws hold for both observes so long as \(u\) is constant, as the observers measure identical accelerations, meaning that the results of applying \(\vec{F} = m\vec{a}\) are also identical.
	\setcounter{section}{2}
	\section{Einstein's Postulates}
		\begin{enumerate}[\textbf{\arabic*.}]
			\item \textbf{Principle of Relativity} 
				The laws of physics are identical for all inertial reference frames.
			\item 
				The speed of light is constant for all inertial reference frames.
		\end{enumerate}
	\section{Consequences of Einstein's Postulates}
		As a consequence of these two postulates, the motion of the observer determines their measurements. \\
		\subsectionb{The Relativity of Time}
			Consider a beam of light traveling up to a mirror and back down (in the \(y\) direction) in a reference frame moving at speed \(u\) in the \(x\) direction. An observer. An observer \(O\) at rest and an observer \(O'\) moving in the same reference frame as the mirror. \\
				From the perspective of \(O'\), the time interval between the light leaving and returning from its source is
				\[\Delta t_0 = \frac{2L_0}{c}\]
				where \(L_0\) is the distance between the origin of the light and the mirror. \\
				From the perspective of \(O\), the light beam travels a distance \(2L\), where
				\[L = \sqrt{L_0^2 + \left(\frac{u\Delta t}{2}\right)^2}\]
				as derived by Pythagorean theorem. As \(O\) observes the beam of light to travel at \(c\), the time interval measured is
				\[
					\Delta t = \frac{2L}{c} 
						= \frac{2\sqrt{L_0^2 + (u\Delta t/2)^2}}{c}
				\]
				Substituting for \(L_0\) and solving for \(\Delta t\) yields
				\[\Delta t = \frac{\Delta t_0}{\sqrt{1 - u^2/c^2}} \tag{time dilation}\]
				As the \(u \le c\), the denominator of this expression is at most 1, meaning that \(\Delta t \le \Delta t_0\); that is, \(O\) measures a longer time interval than \(O'\). For observer \(O\), the beginning and ending of the time interval occur at the same location, so the interval \(\Delta t_0\) observed by them is known as \textit{proper time}.
		\subsectionb{The Relativity of Length}
			Suppose the beam of light instead travels in the same direction as that of the motion of \(O'\). According to \(O\), the distance traveled is \(L\), while from the perspective of \(O'\) (from which the clock is at rest), it is \(L_0\). \\
				The light takes time interval \(\Delta t_1\) to reach the mirror, in which it travels distance \(c\Delta t_1\), which is equal to the length \(L\) plus the distance \(u\Delta t_1\) traveled by by \(O'\):
				\[c\Delta t_1 = L + u\Delta t_1\]
				The light then takes time interval \(\Delta t_2\) to return to its source, in which it travels distance \(c\Delta t_2\), equal to the length \(L\) minus the distance \(u\Delta t_2\):
				\[c\Delta t_2 = L - u\Delta t_2\]
				Adding these times yields \(\Delta t\):
				\[
					\Delta t = \Delta t_1 + \Delta t_2
						= \frac{L}{c - u} + \frac{L}{c + u}
						= \frac{2L}{c}\frac{1}{1 - u^2/c^2}
				\]
				Applying the formula for time dilation,
				\[
					\Delta t = \frac{\Delta t_0}{\sqrt{1 - u^2/c^2}}
						= \frac{2L_0}{c}\frac{1}{\sqrt{1 - u^2/c^2}}
				\]
				Setting the above two equations equal to each other yields
				\[L = L_0\sqrt{1 - \frac{u^2}{c^2}} \tag{length contraction}\]
				Observer \(O'\), who is at rest relative to the object, measures the \textit{rest length \(L_0\)} (the \textit{proper length}) while all observers relative to whom \(O'\) is in motion measure a shorter length \textit{only in the direction of motion}.
				A moving body appears to be shorter than one at rest. \\
				It should be noted that length contraction occurs only in the direction of movement, meaning that any dimensions perpendicular to that of movement are unaffected. \\
		\subsectionb{Relativistic Velocity Addition}
			A source emits particles traveling at speed \(v'\) according to observer \(O'\), who is at rest with respect with respect to the source. A bulb is triggered to flash when the particle reaches it. This flash makes a return trip to the source. The time interval \(\Delta t_0\) measured by \(O'\) is comprised of the time it takes for the particle to travel distance \(L_0\) at speed \(v'\) and the time it takes the flash to travel the same distance at \(c\):
				\[\Delta t_0 = \frac{L_0}{v'} + \frac{L_0}{c}\]
				\(O'\) is moving at speed \(u\) relative to observer \(O\). For \(O\), the particle traveling at speed \(v\) according to \(O\) reaches the bulb after \(\Delta t_1\), traveling a distance of \(v\Delta t_1\) equal to the (contracted) length \(L\) plus the distance \(u\Delta t_1\) moved by \(O'\):
				\[v\Delta t_1 = L + u\Delta t_1\]
				In the interval \(\Delta t_2\), the light beam traverses distance \(c\Delta t_2\) equal to length \(L\) minus the distance \(u\Delta t_2\) traveled by \(O'\):
				\[c\Delta t_2 = L - u\Delta t_2\]
				Solving the above two equation for \(\Delta t_1\) and \(\Delta t_2\) and adding them yields the total interval \(\Delta t\) between the emission of the particle and the arrival of the beam at the detector according to \(O\). Applying the time dilation formula to relate this result to \(\Delta t_0\) and using the length contraction formula to relate \(L\) to \(L_0\) yields the result
				\[v = \frac{v' + u}{1 + v'/uc^2} \tag{relativistic velocity addition}\]
				Note that when \(v' = c\) (a beam of light being observed),
				\[
					v = \frac{c + u}{1 + cu/c^2}
							= \frac{c + u}{1 + u/c}
							= \frac{c + u}{\dfrac{c + u}{c}}
							= c
				\]
				regardless of the value of \(u\). All observers measure the same speed of light \(c\) regardless of their reference frame. \\
		\subsectionb{The Relativistic Doppler Effect}
			The classical Doppler effect states that an observer \(O\) moving relative to the source of waves detects a frequency \(f'\) different from the true frequency \(f\) being emitted by the source \(S\):
				\[f' = f\frac{v \pm v_O}{v \mp v_S} \tag{classical Doppler effect}\]
				where \(v\) is the speed of the waves through some medium, \(v_S\) is the speed of the source \textit{relative to the medium}, and \(O\) is that of the observer \textit{relative to the medium}. The upper signs are chosen when \(S\) and \(O\) are moving towards each other while the lower ones are chosen when they are moving away from each other. \\
			Consider a source of electromagnetic waves at rest in the reference frame of observer \(O\). An observer \(O'\) moving relative to the source at speed \(u\) will observe a different frequency. Suppose \(O\) observes \(N\) waves emitted at frequency \(f\). It takes an interval \(\Delta t_0 = N/f\) for these \(N\) waves to be emitted from the point of view of \(O\). This is the proper time interval in the frame of reference of \(O\). The corresponding interval to \(O'\) is \(\Delta t'\), during which \(O\) moves a distance \(u\Delta t'\). The wavelength \(\lambda'\) according to \(O'\) is the total length occupied by the waves divided by the number of waves:
				\[
					\lambda' = \frac{c\Delta t' + u\Delta t'}{N}
						= \frac{c\Delta t' + u\Delta t'}{f\Delta t_0}
				\]
				The frequency from the reference frame of \(O'\) is \(f' = c/\lambda'\), so
				\[
					f' = \frac{c}{\lambda'}
						= f\frac{\Delta t_0}{\Delta t'}\frac{1}{1 + u/c}
				\]
				Applying the time dilation formula,
				\[
					f' = f\frac{\sqrt{1 - u^2/c^2}}{1 + u/c}
						= f\sqrt{\frac{1 - u/c}{1 + u/c}} \tag{relativistic Doppler shift}
				\]
				This is the formula for the \textit{relativistic Doppler shift} where the waves are observed parallel to \(\vec{u}\). It should be noted that unlike the classical formula, this formula does \textit{not} distinguish between the source and observer motion; it is only dependent on the relative speed \(u\) between the source and observer. \\
				Restated,
				\begin{align*}
					\lambda' &= \lambda\sqrt{\frac{1 + u/c}{1 - u/c}} &
						\frac{u}{c} &= \frac{(\lambda'/\lambda)^2 - 1}{(\lambda'/\lambda)^2 + 1}	
				\end{align*}

	\section{The Lorentz Transformation}
		While the Galilean transformation holds for the \enquote{common-sense} experience observed at low speeds, it does not agree with experiments performed at high speeds. A new set of transformations is therefore required that is capable of reproducing relativistic effects such as time dilation, length contraction, velocity addition, and the Doppler shift. \\
		A transformation that enables observers \(O\) and \(O'\) moving relative to each other to compare their measurements in space and time coordinates of the same event. Teh transformation equations relate the measurements \(x\), \(y\), \(z\), \(t\) in \(O\) to \(x'\), \(y'\), \(z\), and \(t'\) in \(O'\). This transformation must be linear (depending only on the first powers of each coordinate, following the homogeneity of space and time), consistent with Einstein's postulates, and it must reduce to the Galilean transformation when the relative speed between the observers is small. It is again assumed that the relative speed is only in the positive \(xx'\) direction. \\
		The new transformation is called the \textit{Lorentz transformation}:
		\begin{align*}
			x' &= \frac{x - ut}{1 - u^2/c^2} &
				y' &= y &
				z' &= z &
				t' &= \frac{t - (u/c^2)x}{\sqrt{1 - u^2/c^2}}
					\tag{Lorentz transformation}
		\end{align*}
		This is sometimes rewritten as
		\begin{align*}
			x' &= \gamma(x - \beta ct) &
				y' &= y &
				z' &= z &
				ct' &= \gamma(ct - \beta x)	
		\end{align*}
		or
		\begin{align*}
			x' &= \gamma(x - vt) &
				y' &= y &
				z' &= z &
				t' &= \gamma\left(t - x\frac{v}{c^2}\right)	
		\end{align*}
		or in matrices
		\begin{align*}
			\begin{bmatrix}
				x' \\
				y' \\
				z' \\
				ct'
			\end{bmatrix} &=
					\begin{bmatrix}
						\gamma & 0 & 0 & -\beta\gamma \\
						0 & 1 & 0 & 0 \\
						0 & 0 & 1 & 0 \\
						-\beta\gamma & 0 & 0 & \gamma
					\end{bmatrix}
					\begin{bmatrix}
						x \\
						y \\
						z \\ 
						ct
					\end{bmatrix} &
				\begin{bmatrix}
					x' \\
					ct'
				\end{bmatrix} &=
					\begin{bmatrix}
						\gamma & -\beta\gamma \\
						-\beta\gamma & \gamma
					\end{bmatrix}
					\begin{bmatrix}
						x \\
						ct
					\end{bmatrix} &
				\begin{bmatrix}
					x \\
					ct
				\end{bmatrix} &=
					\begin{bmatrix}
						\gamma & \beta\gamma \\
						\beta\gamma & \gamma
					\end{bmatrix}
					\begin{bmatrix}
						x' \\
						ct''
					\end{bmatrix}
		\end{align*}
		where
		\[
			\beta = \frac{u}{c} \qquad \text{and} \qquad
				\gamma = \frac{1}{\sqrt{1 - u^2/c^2}}
		\]
		It is often useful to write these equation in terms of intervals by replacing each coordinate by their corresponding interval. \\
		The equations for \(y'\) and \(z'\) are identical to their Galilean counterparts while the first and last equations reduce to theirs when \(u \ll c\). \\
		The Lorentz transformation can then be used to derive some predictions of special relativity.
		\subsectionb{Length Contraction}
			A rod of length \(L_0\) is at rest as observed by \(O'\). The rod extends from \(x_1'\) to \(x_2'\); that is, \(L_0 = x_2' - x_1'\). Observer \(O\), relative to whom the rod is in motion, measures the ends of the rod at \(x_1\) and \(x_2\). For the length of the rod to be measured by \(O\), \(x_1\) and \(x_2\) must be \textit{simultaneously} determined. \\
			Suppose \(O'\) flashes a bulb at \(x_1'\) and \(t_1'\) and then flashes a bulb at \(x_2'\) and \(t_2'\). The Lorentz transformation gives
				\begin{align*}
					x_1' &= \frac{x_1 - ut_1}{\sqrt{1 - u^2/c^2}} &
						x_2' &= \frac{x_2 - ut_2}{\sqrt{1 - u^2/c^2}}	
				\end{align*}
				Subtracting the first equation from the second yields
				\[x_2' - x_1' = \frac{x_2 - x_1}{\sqrt{1 - u^2/c^2}} - \frac{u(t_2 - t_2)}{\sqrt{1 - u^2/c^2}}\]
				The bulbs must be set off such that the flashes appear to be simultaneous to \(O\). This enables a simultaneous determination to be made of the rod's endpoints. if the flashes are observed simultaneously, \(t_1 = t_2\), so
				\[x_2' - x_1' = \frac{x_2 - x_1}{\sqrt{1 - u^2/c^2}}\]
				Substituting \(L_0\) for \(x_2' - x_1'\) and \(L\) for \(x_2 - x_1\) yields
				\[L = L_0\sqrt{1 - u^2/c^2}\]
		\subsectionb{Velocity Transformation}
			Suppose \(O\) observes a particle traveling with velocity \(\vec{v}\). The relationship between \(\vec{v}\) and the velocity \(\vec{v}\vps'\) observed by \(O'\) is given by the \textit{Lorentz velocity transformation}
				\begin{align*}
					v_x' &= \frac{v_x - u}{1 - v_xu/c^2} &
						v_y' &= \frac{v_y\sqrt{1 - u^2/c^2}}{1 - v_xu/c^2} &
						v_z' &= \frac{v_z\sqrt{1 - u^2/c^2}}{1 - v_xu/c^2}
							\tag{Lorentz velocity transformation}	
				\end{align*}
				Solving the above equations for \(v_x\), it can be shown that this is identical to the equation for relativistic velocity addition. It should be noted that for \(u \ll c\), the Lorentz velocity transformation reduces to its Galilean counterpart. Note also that despite \(y' = y\), \(v_y' \ne v_y\). This is due to the Lorentz transformation's handling of \(t\). \\
				These transformation can be derived from the coordinate transformation. 
				Differentiating the velocity transformation for \(x'\) yields
				\[\dd{x'} = \frac{\dd{x} - u\dd{t}}{1 - u^2/c^2}\]
				Differentiating the time coordinate transformation yields
				\[\dd{t'} = \frac{\dd{t} - \left(u/c^2\right)\dd{x}}{1 - u^2/c^2}\]
				\(v_x'\) is therefore
				\[
					v_x' = \dv{x'}{t'}
							= \frac{\dfrac{\dd{x} - u\dd{t}}{1 - u^2/c^2}}{\dfrac{\dd{t} - (u/c^2)\dd{x}}{1 - u^2/c^2}}
							= \frac{\dd{x} - u\dd{t}}{\dd{t} - (u/c^2)\dd{x}}
							= \frac{\dv*{x}{t} - u}{1 - (u/c^2)\dv*{x}{t}}
							= \frac{v_x - u}{1 - v_xu/c^2}	
				\]
				Differentiating \(y' = y\) yields \(\dd{y'} = \dd{y}\), so
				\[
					v_y' = \dv{y'}{t'}
							= \frac{\dd{y}}{\dfrac{\dd{t} - (u/c^2)\dd{x}}{\sqrt{1 - u^2/c^2}}}
							= \frac{\dd{y}\sqrt{1 - u^2/c^2}}{\dd{t} - (u/c^2)\dd{x}}
							= \frac{\sqrt{1 - u^2/c^2}\dv*{y}{t}}{1 - (u/c^2)\dv*{x}{t}}
							= \frac{v_y\sqrt{1 - u^2/c^2}}{1 - uv_x/c^2}
				\]
				Differentiating for \(z' = z\) yields \(\dd{z'} = \dd{z}\), so
				\[
					v_z' = \dv{z'}{t'}
							= \frac{\dd{z}}{\dfrac{\dd{t} - (u/c^2)\dd{x}}{\sqrt{1 - u^2/c^2}}}
							= \frac{\dd{z}\sqrt{1 - u^2/c^2}}{\dd{t} - (u/c^2)\dd{x}}
							= \frac{\sqrt{1 - u^2/c^2}\dv*{z}{t}}{1 - (u/c^2)\dv*{x}{t}}
							= \frac{v_z\sqrt{1 - u^2/c^2}}{1 - uv_x/c^2}
				\]
	\section{The Twin Paradox}
		Suppose there are a pair of twins on earth. Twin A remains on Earth while twin B sets off on a rocket ship. A believes that as a consequence of time dilation, B should be younger than them by the time they return. For two observers in relative motion, though, \textit{each} believes the \textit{other's} clock to be slow. From the point of view of B, A make a round-trip away from them and back again. Under such circumstances, it should be A's clock that is running slow, so A they expect A to be younger when they return. \\
			The resolution of this paradox comes from considering the asymmetric roles of the twins. Special relativity applies only to inertial frames. In order to leave and reenter Earth, B must accelerate and correspondingly decelerate. Although this may be over a brief time interval, B's return journey occurs in a different inertial frame to their departure. This change in B's inertial frame is what results in the asymmetry in the twins' ages. Only B is required to change inertial frames to return, so \textit{all observers will agree} that B is the one that is \enquote{really} in motion, so it is therefore their clock that is \enquote{really} running slow, meaning that B will indeed be the younger twin upon their return.
		\subsectionb{Spacetime Diagrams}
			A \textit{spacetime} diagram plots time on the \(y\)-axis and space on the \(x\)-axis. The slope of a straight line is velocity as a proportion of \(c\).
	\section{Relativistic Dynamics}
		Dynamical quantities such as momentum and kinetic energy are dependent on  length, time, and velocity. As such, they are also affected by Einstein's postulates. \\
		Consider a collision of two particles that is elastic from the reference frame of \(O'\). Particle 1 of mass \(m_1 = 2m\) is initially at rest while particle 2 of mass \(m_2 = m\) is moving in the negative \(x\) direction with initial velocity \(\subt{v}{2,i}' = -0.75c\). The final velocities of particles 1 and 2 are respectively \(\subt{v}{1,f}' = -0.5c\) and \(\subt{v}{2,f}' = 0.25c\). Analyzing this collision classically,
			\begin{align*}
				\subt{p}{i}' &= m_1\subt{v}{1,i}' + m_2\subt{v}{2,i}'
						= -0.75mc &
					\subt{p}{i}' &= m_1\subt{v}{2,f}' + m_2\subt{v}{2,f}'
						= -0.75mc
						= \subt{p}{i}'
			\end{align*}
			As the initial and final momenta are equal according to \(O'\), energy is conserved. \\
		Suppose the reference frame of \(O'\) is moving in the positive \(x\) direction at a speed of \(u = 0.55c\) relative to \(O\). The velocity transformation yields the velocities of \(\subt{v}{1,i} = 0.55c\), \(\subt{v}{2,i} = -0.34c\), \(\subt{v}{1,f} = 0.069c\), and \(\subt{v}{2,f} = 0.703c\). Analyzing the collision,
			\begin{align*}
				\subt{p}{i} &= m_1\subt{v}{1,i} + m_2\subt{v}{2,i}
						= 0.76mc &
					\subt{p}{f} &= m_1\subt{v}{1,f} + m_2\subt{v}{2,f}
						= 0.841mc
						\ne \subt{p}{i}
			\end{align*}
			As \(\subt{p}{i} \ne \subt{p}{f}\), momentum is not conserved according to \(O\). This means that the law of conservation of linear momentum does not satisfy Einstein's first postulate. To retain the conservation of momentum, then, a new definition of momentum is required. This definition must yield a conservation law that holds under relativity (momentum being conserved regardless of the reference frame, so long as it is inertial) and reduces to \(\vec{p} = m\vec{v}\) at low speeds. \\
			These requirements are satisfied by defining relativistic momentum to be
			\[\vec{p} = \frac{m\vec{v}}{\sqrt{1 - v^2/c^2}} \tag{relativistic momentum}\]
			The velocity \(v\) is the velocity of the particle as measured from a particular inertial frame; it is not the velocity of said frame. \\
			Using this relativistic definition of momentum to analyze the above collision,
			\begin{align*}
				\subt{p}{i}' &= \frac{m_1\subt{v}{1,i}'}{\sqrt{1 - \subt{v}{1,i}'^2/c^2}} + \frac{m\subt{v}{2,i}'}{\sqrt{1 - \subt{v}{2,i}'^2/c^2}}
						\approx 1.134mc &
					\subt{p}{f}' &= \frac{m_2\subt{v}{1,f}'}{\sqrt{1 - \subt{v}{1,f}'^2/c^2}} + \frac{m_2\subt{v}{2,f}'}{\sqrt{1 - \subt{v}{2,f}'^2/c^2}}
						\approx 1.134mc
						= \subt{p}{i}'
			\end{align*}
			As \(\subt{p}{i}' = \subt{p}{f}'\), \(O'\) is able to conclude that momentum is conserved. To \(O\),
			\begin{align*}
				\subt{p}{i} &= \frac{m_1\subt{v}{1,i}}{\sqrt{1 - \subt{v}{1,i}^2/c^2}} + \frac{m\subt{v}{2,i}}{\sqrt{1 - \subt{v}{2,i}^2/c^2}}
						\approx 0.956mc &
					\subt{p}{f} &= \frac{m_1\subt{v}{1,f}}{\sqrt{1 - \subt{v}{1,f}^2/c^2}} + \frac{m_2\subt{v}{2,f}}{\sqrt{1 - \subt{v}{2,f}^2/c^2}}
						\approx 0.946mc
						= \subt{p}{i}
			\end{align*}
			so \(O'\) can also conclude that momentum is conserved.
		\subsectionb{Relativistic Kinetic Energy}
			Much like classical momentum, classical kinetic energy also fails under relativity. According to \(O'\), the aforementioned collision has kinetic energies
				\begin{align*}
					\subt{K}{i}' &= \frac{1}{2}m_1\subt{v}{1,i}'^2 + \frac{1}{2}m_2\subt{v}{2,i}'^2 
						\approx 0.281mc^2 &
						\subt{K}{f}' &= \frac{1}{2}m_1\subt{v}{1,f}'^2 + \frac{1}{2}m_2\subt{v}{2,f}'^2 
							\approx 0.281mc^2
							= \subt{K}{i}'
				\end{align*}
				but according to \(O\) is has
				\begin{align*}
					\subt{K}{i} &= \frac{1}{2}m_1\subt{v}{1,i}^2 + \frac{1}{2}m_2\subt{v}{2,i}^2 
						\approx 0.36mc^2 &
						\subt{K}{f} &= \frac{1}{2}m_1\subt{v}{1,f}^2 + \frac{1}{2}m_2\subt{v}{2,f}^2 
							\approx 0.252mc^2
							\ne \subt{K}{i}
				\end{align*}
				so \(O\) concludes that energy is not conserved. \\
			The relativistic definition of kinetic energy is
				\[K = \frac{mc^2}{\sqrt{1 - v^2/c^2}} - mc^2 \tag{relativistic kinetic energy}\]
			There is no limit to the amount of kinetic energy that can be given to a particle. Under the classical definition of kinetic energy, this means that speed must also lack a limit, which violates Einstein's second postulate.
		\subsectionb{Relativistic Total Energy and Rest Energy}
			Relativistic kinetic energy can also be expressed as
				\[K = E - E_0 \tag{relativistic kinetic energy}\]
				where the \textit{relativistic total energy \(E\)} is defined as
				\[E = \frac{mc^2}{\sqrt{1 - v^2/c^2}} \tag{relativistic total energy}\]
				and the \textit{rest energy \(E_0\)} is defined is
				\[E_0 = mc^2 \tag{rest energy}\]
				This suggests that mass can be expressed in units of \(\mathrm{MeV}/c^2\). The \(m\) in this equation is sometimes called the \textit{rest mass \(m_0\)}, which is distinguished from the \enquote{relativistic mass} \(m'\), defined as
				\[m' = \frac{m_0}{\sqrt{1 - v^2/c^2}} \tag{relativistic mass}\]
				The total energy can be found as the sum of the kinetic and rest energies:
				\[E = E_0 + K\]
			Manipulating the equations for relativistic momentum and total energy yields the relationship
				\[E = \sqrt{(pc)^2 + (mc^2)^2}\]
				This can be remembered by
				\[\begin{tikzpicture}[scale = 1.5]
					\node (a) at (0, 0) {};
					\node (b) at (4, 0) {};
					\node (c) at (4, 3) {};
					\draw (a.center) -- node[below]{\(E_0 = mc^2\)} (b.center);
					\draw (b.center) -- node[right]{\(pc\)} (c.center);
					\draw[name path = line] (c.center) -- (a.center);
					\path pic[draw, <->, angle radius = 1cm, "\(\arcsin(v/c)\)", angle eccentricity = 1.5, anchor = west, inner sep = 1 pt] {angle = b--a--c};
					\draw[name path = arc, dashed] (4, 0) arc (0:{atan(3/4)}:4);
					\node[name intersections = {of = line and arc}] (d) {};
					\draw[|<->|] (-0.12, 0.16) to (3.88, 3.16);
				\end{tikzpicture}\]
			A particle traveling close to the speed of light (\(v > 0.99c\)), \(K \gg E_0\). IN this case,
				\[E \cong pc\]
				This is the \textit{extreme relativistic approximation} and can often simplify calculations. For massless particles (such as photons),
				\[E = pc\]
				All massless particles travel at \(c\); otherwise, their kinetic and total energies would be zero.
	\section{Conservation Laws in Relativistic Decays and Collisions}
		The law of conservation of momentum can be applied to all collisions. The only difference for high speeds is that the relativistic definition of momentum must be used. It can be stated the same:
			\callout{14}{\textit{In an isolated system of particles, the total linear momentum remains constant.}}
		Classically, the only form of energy present in an elastic collision is kinetic, so the conservation of energy is simply that of kinetic energy. In inelastic collisions or decay processes, kinetic energy is not conserved. The total energy does indeed remain constant, though; the other forms of energy are simply unaccounted for. This missing energy is usually stored in particles as atomic or nuclear energy. \\
		Relativistically, the internal stored energy is accounted for by the rest energy. In atomic and nuclear processes, the only forms of energy generally accounted for are kinetic and rest (though radiation will be considered later). A loss of kinetic energy then implies an increase in rest energy, as the total relativistic energy does not change. \\
			Consider a reaction in which new particles are produced. The loss of kinetic energy in the original particles is what provides the rest energy of the new particle. In a nuclear decay process, on the other hand, the initial nucleus loses some rest energy in the form of the kinetic energy given to the decay products. \\
			The relativistic law of conservation of energy is as follows:
			\callout{10.2}{\textit{In an isolated system of particles, the relativistic total energy (kinetic energy plus rest energy) remains constant.}}
			When applying this law to relativistic collisions, it does not matter whether a collision is elastic or inelastic, as the inclusion of rest energy accounts for any potential loss of kinetic energy.
	\section{Extra}
		The \textbf{Epstein Cosmic Speedometer} is a plot of proper time (on the vertical axis) against space (on the horizontal axis). A body in motion will move through both space and time, making it an angle line. A stationary body will move only in time, making it  a vertical line. The only thing able to move through space but not time is light, making it a horizontal line. \\
		A spacetime diagram maintains the length of all vectors, plotting space on the horizontal axis and time on the vertical axis. The vertical component of a vector is the \enquote{proper} while its horizontal component is the speed through space. \\
		Moving through space means moving slower through time; that is, moving objects effectively have slower clocks. \\
		Graphs can be made in Lorentz/Minkowski Space. The axes are \(x\) and \(ct\). The new axes can be found as lines with direct variation of slopes \(x'\) and \(ct'\). Note that \(ct'\) and \(x'\) are not perpendicular. To graph with them, then, one must first go along one axis and then parallel along the other. The greater \(\beta\), is, the closer together the \(ct'\) and \(x'\) axes; that is, the slower a reference frame is traveling, the less it skews spacetime. The graph can be used to plot a point in one reference frame and see its coordinates in the other. \\
		It should be noted that
			\[(ct)^2 - x^2 = (ct')^2 - x'^2 \tag{invariant}\]
\end{document}