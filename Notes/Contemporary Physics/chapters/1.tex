\documentclass{subfiles}

\begin{document}
	Under classical physics,
		\begin{align*}
			\vec{v} &= \frac{\Delta\vec{r}}{t} &	
				K &= \frac{1}{2}mv^2 &
				\vec{p} &= m\vec{v} & 
				\vec{L} &= \vec{r} \times \vec{p}
					= I\vec{\omega}
		\end{align*}
		where \(\vec{v}\) is velocity, \(\vec{r}\) is position, \(m\) is mass, \(\vec{p}\) is momentum, \(\vec{L}\) is angular momentum, and \(I\) is rotational inertia. \\
	Working on very small scales, the unit of charge used is often the magnitude of that of an electron (or proton)
		\[\SI{1}{e} \approx \SI{1.602e-19}{C}\]
		and that of energy is the \textbf{electron-volt (eV)}, which is the energy of an electron subjected to a \SI{1}{V} potential difference:
		\[\SI{1}{eV} \approx \SI{1.602e-19}{J}\]
		An \textbf{atomic mass unit (amu)} is defined such that
		\[\SI{12}{amu} = \text{carbon atom}\]
		The speed of light \(c\) and Planck's constant \(h\) are
		\begin{align*}
			c &\approx \SI[per-mode = fraction]{3e8}{\m\per\s} &
				h &\approx \SI{6.63e-34}{\J\s}
					\approx \SI{4.14e-15}{\eV\s}
		\end{align*}
		The product of these two constants is
		\[hc \approx \SI{1240}{\eV\nm}\]
	Failures of space and time require relativity while those of particle theories require quantum mechanics. \\
	A \textbf{theory} is an organized body of facts. It provides a model of explanation. \\
	A \textbf{hypothesis} is a prediction as to how something works. (With data to back up a hypothesis, it can become a theory.) \\
	A \textbf{law} is a correlation between variables.
\end{document}
