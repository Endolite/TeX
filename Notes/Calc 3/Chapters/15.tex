\documentclass[../Calculus \Roman{3}.tex]{subfiles}

\begin{document}
	\section{Double Integrals over Rectangles}
		Just as the area problem leads to the definite integral, the definition of the double integral can be found by attempting to find the volume of a solid.
		\subsection*{Review of the Definite Integral}
			\addcontentsline{toc}{subsection}{Volumes and Double Integrals}
			To find the area of a curve $f$ over the interval $[a, b]$, the interval can be split into $n$ subintervals $[x_{i - 1}, x_i]$ of equal width $\Delta x = (b - a)/n$. Sample points $x_i^*$ can then be chosen in each subinterval. The Riemann sum
				\[\sum_{i = 1}^n f(x_i^*)\Delta x\]
				can be evaluated as $n$ approaches infinity to obtain the definite integral of $f$ from $a$ to $b$.
				\[\int_a^b f(x)\,\d x = \lim_{n \to \infty}\sum_{i = 1}^n f(x_i^*)\Delta x\]
		\subsection*{Volumes and Double Integrals}
			\addcontentsline{toc}{subsection}{Volumes and Double Integrals}
			Consider a function $f$ of two variables defined on a closed rectangle
				\[R = [a, b] \times [c, d] = \left\{(x, y) \in \R^2 \mid a \le x \le b \land c \le y \le d\right\}\]
				Suppose that $f(x, y) \ge 0$. The graph of $f$ is the surface $z = f(x, y)$. Let $S$ be a solid that lies above $R$ and under the graph of $f$.
					\[S = \left\{(x, y, z) \in \R^3 \mid 0 \le z \le f(x, y) \land (x, y) \in R\right\}\]
				To find the volume of $S$, $R$ can first be divided into sub-rectangles, dividing the interval $[a, b]$ into $m$ subintervals $[x_{i - 1}, x_i]$ of equal width $\Delta x = (b - a)/m$ and the interval $[c, d]$ into $n$ subintervals $[y_{i - 1}, y_i]$ of equal width $\Delta y = (d - c)/n$. Drawing lines parallel to the axes through the endpoints of these subintervals forms the sub-rectangles
					\[R_{i,j} = [x_{i - 1}, x_i] \times [y_{i - 1}, y_i] = \{(x, y) \mid x_{i - 1} \le x \le x_i \land y_{j - 1} \le y \le y_j\}\]
					of equal area
						\[\Delta A = \Delta x\Delta y\]
				Selecting a \textbf{sample point} $\bigl(x_{i, j}^*, y_{i, j}^*\bigr)$ in each $R_{i, j}$ enables the approximation of the part of $S$ that lies above $R_{i, j}$, creating boxes of volume
					\[f\bigl(x_{i, j}^*, y_{i, j}^*\bigr)\Delta A\]
				The sum of these boxes can be used to approximate the total volume of $S$.
					\[V \approx \sum_{i = 1}^m\sum_{j = 1}^n f\bigl(x_{i, j}^*, y_{i, j}^*\bigr)\Delta A\]
					This is referred to as the \textbf{double Riemann sum}.
			\callout{17}{
				The \textbf{double integral} of $f$ over the rectangle $R$ is
					\[\iint\limits_R f(x, y)\,\d A = \lim_{m, n \to \infty}\sum_{i = 1}^m\sum_{j = 1}^n f\bigl(x_{i, j}^*, y_{i, j}^*\bigr)\Delta A\]
					if this limit exists.
			}
			This definition of the double integral means that for every number $\varepsilon > 0$, there is an integer $N$ such that
				\[\Biggl|\iint\limits_R f(x, y)\,\d A - \sum_{i = 1}^m\sum_{j = 1}^n f\bigl(x_{i, j}^*, y_{i, j}^*\bigr)\Delta A\Biggr| < \varepsilon\]
				for all integers $m$ and $n$ greater than $N$ and for any choice of sample points $\bigl(x_{i, j}^*, y_{i, j}^*\bigr)$ in $R_{i, j}$. \\
			A function is \textbf{integrable} if the limit definition of its definite integral converges. If $f$ is bounded on $R$ and $f$ is continuous over $R$ (except possibly on a finite number of smooth curves), then $f$ is integrable over $R$. \\
			The sample point $\bigl(x_{i, j}^*, y_{i, j}^*\bigr)$ can be chosen to be any point in sub-rectangle $R_{i, j}$, but if the upper-right corner is chosen, the expression for a double integral can be simplified.
				\[\iint f(x, y)\,\d a = \sum_{i = 1}^m\sum_{j = 1}^n f(x_i, y_j) \Delta A\]
			\callout{17}{
				If $f(x, y) \ge 0$, the volume $V$ of a solid that lies above rectangle $R$ and below surface $z = f(x, y)$ is
					\[V = \iint\limits_R f(x, y)\,\d A\]
			}
		\subsection*{The Midpoint Rule}
			\addcontentsline{toc}{subsection}{The Midpoint Rule}
			\callout{17}{\paragraph{Midpoint Rule for Double Integrals}
				\[\iint\limits_R f(x, y)\,\d A \approx \sum_{i = 1}^m\sum_{j = 1}^n f(\bar{x}_i, \bar{y}_i)\Delta A\]
				where $\bar{x}_i$ is the midpoint of $[x_{i - 1}, x_i]$ and $\bar{y}$ is that of $[y_{i - 1}, y_i]$.
			}
		\subsection*{Iterated Integrals}
			\addcontentsline{toc}{subsection}{Iterated Integrals}
			Single definite integrals are generally far easier to solve using the fundamental theorem of calculus rather than their definition. The same is true for double integrals, simply using two single integrals. \\
			Suppose $f$ is a function of two variables that is integrable over rectangle $R = [a, b] \times [c, d]$. The notation
				\[\int_c^d f(x, y)\,\d y\]
				is used to denote that $x$ is held constant and $f(x, y)$ is integrated with respect to $y$. This is called \textit{partial integration with respect to $y$}. \\
				As the value of $\int_c^d f(x, y)\,\d y$ is dependent on $x$, it defines a function in terms of $x$. Integrating this function with respect to $x$ over $[a, b]$ results in an \textbf{iterated integral}
				\[\int_a^b\int_c^d f(x, y)\,\d x\,\d y\]
				This integral is evaluated \textit{from the inside out}.
			\callout{17}{\paragraph{Fubini's Theorem}
				If $f$ is continuous over rectangle
					\[R = \{(x, y) \mid a \le x \le b \land c \le y \le d\}\]
					then
					\[
						\iint\limits_R f(x, y)\,\d A = \int_a^b\int_c^d f(x, y)\,\d x\,\d y 
								= \int_c^d\int_a^b f(x, y)\,\d y\,\d x
					\]
					so long as $f$ is bounded on $R$, $f$ is discontinuous only on a finite number of smooth curves, and the iterated integrals exist.
			}
			In the special case that $f(x, y)$ is factorable into the product of functions $g$ and $h$ of only $x$ and only $y$,
				\[\iint\limits_R g(x)h(y)\,\d A = \int_a^bg(x)\,\d x\int_c^dh(y)\,\d y\]
				where $R = [a, b] \times [c, d]$.
		\subsection*{Average Value}
			\addcontentsline{toc}{subsection}{Average Value}
			The \textbf{average value} of a single-variable function $f$ on the interval $a, b$ is
				\[f_{\avg} = \frac{1}{b - a}\int_a^b f(x)\,\d x\]
				That of a function of two variables defined on rectangle $R$ is similarly
				\[f_{\avg} = \frac{1}{A(R)}\iint\limits_R f(x, y)\,\d A\]
				where $A(R)$ is the area of $R$.
	\section{Double Integrals over General Regions}
		\subsection*{General Regions}
			\addcontentsline{toc}{subsection}{General Regions}
			Suppose a general region $D$ is bounded, meaning that it can be enclosed by a rectangular region $R$. In order to integrate a function $f$ over $D$, a new function $F$ is defined with domain $R$ as
				\[
					F(x, y) = 
						\begin{cases}
 							f(x, y) & 
 								(x, y) \in D \\
 							0 &
 								(x, y) \notin D \land (x, y) \in R
				 		\end{cases}
				\]
				If $F$ is integrable over $R$, the \textbf{double integral of $\bm{f}$ over $\bm{d}$} is defined as
					\[\iint\limits_D f(x, y)\,\d A = \iint\limits_R F(x, y)\,\d A\]
			A plane region $D$ is \textbf{type \Roman{1}} if it lies between the graphs of two continuous functions of $x$.
				\[D = \{(x, y) \mid a \le x \le b \land g_1(x) \le y \le g_2(x)\}\]
				where $g_1$ and $g_2$ are continuous on $[a, b]$. (It should be noted that $g_1$ and $g_2$ need only be continuous, meaning that they may be piecewise.)
			\callout{17}{
				If $D$ is a continuous type \Roman{1} region described by
					\[D = \{(x, y) \mid a \le x \le b \land g_1(x) \le y \le g_2(x)\}\]
					then
					\[\iint\limits_D f(x, y)\,\d A \int_a^b\int_{g_1(x)}^{g_2(x)} f(x, y)\,\d y\,\d x\]
			}
			A plane region of \textbf{type \Roman{2}} can be expressed as
				\[D = \{(x, y) \mid c \le y \le d \land h_1(y) \le x \le h_2(y)\}\]
				where $h_1$ and $h_2$ are continuous.
			\callout{17}{
				If $D$ is a continuous type \Roman{2} region described by
					\[D = \{(x, y) \mid c \le y \le d \land h_1(y) \le x \le h_2(y)\]
					then
					\[\iint\limits_R f(x, y)\,\d A = \int_c^d\int_{h_1(y)}^{h_2(y)}f(x, y)\,\d x\,\d y\]
			} 
		\subsection*{Changing the Order of Integration}
			\addcontentsline{toc}{subsection}{Changing the Order of Integration}
			The order of integration can be changed to make an integral easier to evaluate.
	\section{Double Integrals in Polar Coordinates}
		\subsection*{Review of Polar Coordinates}
			\addcontentsline{toc}{subsection}{Review of Polar Coordinates}
			The polar coordinates $(r, \theta)$ are related to their corresponding rectangular coordinates $(x, y)$ by the following formulas:
			\begin{align*}
				r^2 &= x^2 + y^2 &
						x &= r\cos\theta &
						y &= r\sin\theta
			\end{align*}
		\subsection*{Double Integrals in Polar Coordinates}
			\addcontentsline{toc}{subsection}{Double Integrals in Polar Coordinates}
				A \textbf{polar rectangle} $R$ is described by
					\[R = \{(r, \theta) \mid a \le r \le b \land \alpha \le \theta \le \beta\}\]
				\callout{17}{
					If $f$ is continuous on polar rectangle $R$ given by
						\[R = \{(r, \theta) \mid a \le r \le b \land \alpha \le \theta \le \beta\}\]
						where
						\[0 \le \beta - \alpha \le 2\pi\]
						then
						\[\iint\limits_R f(x, y)\,\d A = \int_\alpha^\theta\int_a^b f(r\cos\theta, r\sin\theta)r\,\d r\d\theta\]
				}
				\callout{17}{
					If $f$ is continuous on a polar region $D$ of the form
						\[D = \{(r, \theta) \mid \alpha \le \theta \le \beta \land h_1(\theta) \le r \le h_2(\theta)\]
						then
						\[\iint\limits_D f(x, y)\,\d A = \int_\alpha^\theta\int_{h_1(\theta)}^{h_2(\theta)} f(r\cos\theta, r\sin\theta)r\,\d r\,\d\theta\]
				}
	\section{Applications of Double Integrals}
		\subsection*{Density and Mass}
			\addcontentsline{toc}{subsection}{Density and Mass}
			If a lamina occupies a region $D$ in the $xy$-plane and its \textbf{density} at a point $(x, y)$ in $D$ is given by $\rho(x, y)$, a continuous function on $D$, 
				\[\rho(x, y) = \lim\frac{\Delta m}{\Delta A}\]
				where $\Delta m$ and $\Delta A$ are the mass and area of small rectangles and the limit is taken as the dimensions of the rectangle approach 0. The total mass of the lamina is determined by
				\[
					m = \lim_{m, n \to \infty}\sum_{i = 1}^m\sum_{j = 1}^n\rho\bigl(x_{i,j}^*, y_{i, j}^*\bigr)\Delta A
							= \iint\limits_D\rho(x, y)\,\d A
				\]
			Other types of density are similar. If an electric charge is distributed over region $D$ and the charge density is given by $\sigma(x, y)$ at point $(x, y)$ in $D$, then the total \textbf{electric charge} $Q$ is given by
				\[Q = \iint\limits_D\sigma(x, y)\,\d A\]
		\subsection*{Density and Mass}
			\addcontentsline{toc}{subsection}{Density and Mass}
			The \textbf{moment} of a lamina \textbf{about the $\bm{x}$-axis} is given by
				\[
					M_x = \lim_{m, n \to \infty}\sum_{i = 1}^m\sum_{j = 1}^n y_{i, j}^*\rho\bigl(x_{i, j}^*, y_{i, j}^*\bigr)\Delta A
							= \iint\limits_D y\rho(x, y)\,\d A
				\]
				The \textbf{moment about the $\bm{y}$-axis} is similarly
					\[
						M_y = \lim_{m, n \to \infty}\sum_{i = 1}^m\sum_{j = 1}^n x_{i, j}^*\rho\bigl(x_{i, j}^*, y_{i, j}^*\bigr)\Delta A
								= \iint\limits_D x\rho(x, y)\,\d A
					\]
			The center of mass $(\bar{x}, \bar{y})$ is defined such that $m\bar{x} = M_y$ and $m\bar{y} = M_x$. The lamina behaves as though its entire mass is concentrated at this point.
			\callout{17}{
				The coordinates $(\bar{x}, \bar{y})$ of the center of mass of a lamina occupying region $D$ with density function $\rho(x, y)$ are
					\begin{align*}
						\bar{x} &= \frac{M_y}{m} = \frac{1}{m}\iint\limits_D x\rho(x, y)\,\d A &
								\bar{y} &= \frac{M_x}{m} = \frac{1}{m}\iint\limits_D y\rho(x, y)\,\d A
					\end{align*}
					where mass $m$ is given by
						\[m = \iint\limits_D \rho(x, y)\,\d A\]
			}
		\subsection*{Moment of Inertia}
			The \textbf{moment of inertia} or \textbf{second moment} of a particle of mass $m$ about an axis is defined as $mr^2$, where $m$ is the particle's mass and $r$ is its distance from the axis. Extending this concept to lamina, the \textbf{moment of inertia} of a lamina \textbf{about the $\bm{x}$-axis} is
				\[
					I_x = \lim_{m, n \to \infty}\sum_{i = 1}^m\sum_{j = 1}^n\bigl(y_{i, j}^*\bigr)^2\rho\bigl(x_{i, j}^*, y_{i, j}^*\bigr)\Delta A
							= \iint\limits_D y^2\rho(x, y)\,\d A
				\]
				and the \textbf{moment of inertia about the $\bm{y}$-axis} is
					\[
						I_y = \lim_{m, n \to \infty}\sum_{i = 1}^m\sum_{j = 1}^n\bigl(x_{i, j}^*\bigr)^2\rho\bigl(x_{i, j}^*, y_{i, j}^*\bigr)\Delta A
								= \iint\limits_D x^*\rho(x, y)\,\d A
					\]
			The \textbf{moment of inertia about the origin} or \textbf{polar moment of inertia} is
				\[
					I_0 = \lim_{m, n \to \infty}\sum_{i = 1}^m\sum_{j = 1}^n\Bigl(\bigl(x_{i, j}^*\bigr)^2 + \bigl(y_{i, j}^*\bigr)^2\Bigr)\rho\bigl(x_{i, j}^*, y_{i, j}^*\bigr)\Delta A
							= \iint\limits_D \bigl(x^2 + y^2\bigr)\rho(x, y)\,\d A
							= I_x + I_y
				\]
			The \textbf{radius of gyration of a lamina about an axis} is the number $R$ such that
				\[mR^2 = I\]
				The radii of gyration with respect to the $x$- and $y$-axes are given by
					\begin{align*}
						m\bbar{y}^2 &= I_x &
								m\bbar{x}^2 &= I_y
					\end{align*}
					$(\bbar{x}, \bbar{y})$ is the point at which tyhe mass could be concentrated without affecting the moment of inertia with respect to the coordinate axes.
		\subsection*{Probability}
			\addcontentsline{toc}{subsection}{Probability}
			The \textbf{probability density function} $f$ of a continuous random variable $X$ defines the probability of $X$ falling between $a$ and $b$ as
				\[P(a \le X \le b) = \int_a^b f(x)\,\d x\]
				The \textbf{joint density function} of two continuous random variables $X$ and $Y$ is a function $f$ of two variables such that the probability that $(X, Y)$ lies in a region $D$ is given by
					\[P((X, Y) \in D) = \iint\limits_D f(x, y)\,\d A\]
				If this region is a rectangle,
					\[P((x_1 \le X \le x_2) \land (y_1 \le Y \le y_2)) = \int_{x_1}^{x_2}\int_{y_1}^{y_2} f(x, y)\,\d y\,\d x\]
				As probabilities can't be negative and are measured from 0 to 1
					\begin{align*}
						f(x, y) &\ge 0 &
							\iint\limits_{\R^2} f(x, y)\,\d A = 1
					\end{align*}
			If $X$ and $Y$ are random variables with respective probability density functions $f_X$ and $f_Y$, $X$ and $Y$ are \textbf{independent random variables} if their joint density function is the product of their individual density functions.
				\[f(x, y) = f_X(x)f_Y(y) \implies \text{$X$ and $Y$ are independent}\]
		\subsection*{Expected Values}
			\addcontentsline{toc}{subsection}{Expected Values}
			The \textbf{mean} of a continuous random variable $X$ with probability density function $f$ is
				\[\mu = \int_{-\infty}^\infty xf(x)\,\d x\]
			If $X$ and $Y$ are independent random variables with joint density function $f$, then the \textbf{$\bm{X}$-} and \textbf{$\bm{Y}$-means} or \textbf{expected values of $\bm{X}$ and $\bm{Y}$} are
				\begin{align*}
					\mu_X &= \iint\limits_{\R^2} xf(x, y)\,\d A &
							\mu_Y &= \iint\limits_{\R^2} yf(x, y)\,\d A
				\end{align*}
			These are the coordinates of the center of mass of the probability density function.
	\section{Surface Area}
		\callout{17}{
			The surface area of a surface with equation $z = f(x, y)$ with $(x, y) \in D$ where $f_x$ and $f_y$ are continuous is
				\[A(S) = \iint\limits_D\left[\sqrt{1 + \left(\pder{z}{x}\right)^2 + \left(\pder{z}{y}\right)^2}\right]\d A\]
		}
	\section{Triple Integrals}
		\subsection*{Triple Integrals over Rectangular Boxes}
			\addcontentsline{toc}{subsection}{Triple Integrals over Rectangular Boxes}
			A rectangular box $B$ is defined as
				\[B = \{(x, y, z) \mid a \le x \le b \land c \le y \le d \land r \le z \le s\}\]
			The \textbf{triple Riemann sum} can be used to define the \textbf{triple integral} of $f$ over $B$.
				\[\iiint\limits_B f(x, y, z)\,\d V = \lim_{l, m, n \to \infty}\sum_{i = 1}^l\sum_{j = 1}^m\sum_{k = 1}^n f\bigl(x_{i, j, k}^*, y_{i, j, k}^*, z_{i, j, k}^*\bigr)\Delta V\]
			\callout{17}{\paragraph{Fubini's Theorem for Triple Integrals}
				If $f$ is continuous over the rectangular box $R = [a, b] \times [c, d] \times [r, s]$, then
					\[\iiint\limits_D f(x, y, z)\,\d V = \int_a^b\int_c^d\int_r^s f(x, y, z)\,\d z\,\d y\,\d x\]
			}
		\subsection*{Triple Integrals over General Regions}
			\addcontentsline{toc}{subsection}{Triple Integrals over General Regions}
			A solid region $E$ is \textbf{type 1} if it lies between the graphs of two continuous functions of $x$ and $y$; that is to say,
				\[E = \{(x, y, z) \mid (x, y) \in D, u_1(x, y) \le z \le u_2(x, y)\}\]
				where $D$ is the projection of $E$ onto the $xy$-plane. \\
				The integral over a type 1 solid region $E$ with $xy$-projection $D$ is
					\[\iiint\limits_E f(x, y, z)\,\d V = \iint\limits_D\int_{u_1(x, y)}^{u_2(x, y)} f(x, y, z)\,\d z\,\d A\]
				If $D$ is itself a type \Roman{1} plane region, then
					\[E = \{(x, y, z) \mid a \le x \le b \land g_1(x) \le y \le g_2(x) \land u_1(x, y) \le z \le u_2(x, y)\}\]
					and
					\[\iiint\limits_E f(x, y, z)\,\d V = \int_a^b\int_{g_1(x)}^{g_2(x)}\int_{u_1(x, y)}^{u_2(x, y)} f(x, y, z)\,\d z\,\d y\,\d x\]
				If it is a type \Roman{2} plane region, on the other hand, then
					\[E = \{(x, y, z) \mid c \le y \le d \land h_1(y) \le x \le h_2(y) \land u_1(x, y) \le z \le u_2(x, y)\}\]
					and
					\[\iiint\limits_E f(x, y, z)\,\d V = \int_c^d\int_{h_1(y)}^{h_2(y)}\int_{u_1(x, y)}^{u_2(x, y)} f(x, y, z)\,\d z\,\d x\,\d y\]
			A solid region $E$ is \textbf{type 2} if it lies between the graphs of two continuous functions of $y$ and $z$; that is,
				\[E = \{(x, y, z) \mid (y, z) \in D \land u_1(y, z) \le x \le u_2(y, z)\}\]
				where $D$ is the projection of $E$ onto the $yz$-plane. \\
				The triple integral over a type 1 solid region $E$ with $yz$-projection $D$ is
					\[\iiint\limits_E f(x, y, z)\,\d V = \iint\limits_D\int_{u_1(y, z)}^{u_2(y, z)} f(x, y, z)\,\d x\,\d A\]
			A solid region $E$ is \textbf{type 3} if it lies between the graphs of two continuous functions of $x$ and $z$; that is,
				\[E = \{(x, y, z) \mid (x, z) \in D \land u_1(x, z) \le y \le u_2(x, z)\}\]
				where $D$ is the projection of $E$ onto the $xz$-plane.
				The integral over a type 3 solid region is $D$ with $xz$-projection $D$ is
					\[\iiint\limits_E f(x, y, z)\,\d V = \iint\limits_D\int_{u_1(x, z)}^{u_2(x, z)} f(x, y, z)\,\d y\,\d A\]
		\subsection*{Applications of Triple Integrals}
			\addcontentsline{toc}{subsection}{Applications of Triple Integrals}
			If $f(x, y, z) = 1$ for all $(x, y, z) \in E$, then the triple integral represents the volume of $E$.
				\[V(E) = \iiint\limits_E f(x, y, z)\,\d V\]
			If a solid region $E$ has density function $\rho(x, y, z)$, the mass of the region can be found as a triple integral.
				\[m = \iiint\limits_E \rho(x, y, z)\,\d V\]
				The three \textbf{moments} of $E$ about each coordinate plane are
					\begin{align*}
						M_{yz} &= \iiint\limits_E x\rho(x, y, z)\,\d V &
								M_{xz} &= \iiint\limits_E y\rho(x, y, z)\,\d V &
								M_{yz} &= \iiint\limits_E z\rho(x, y, z)\,\d V
					\end{align*}
				The \textbf{center of mass} is at the point $(\bar{x}, \bar{y}, \bar{z})$ where
					\begin{align*}
						\bar{x} &= \frac{M_{yz}}{x} &
								\bar{y} &= \frac{M_{xz}}{y} &
								\bar{z} &= \frac{M_{xy}}{z}
					\end{align*}
					If the density is constant, the center of mass is called the \textbf{centroid} of $E$. \\
				The \textbf{moments of inertia} about the coordinate axes are
					\begin{align*}
						I_x &= \iiint\limits_E \bigl(y^2 + z^2\bigr)\rho(x, y., z)\,\d V &
								I_y &= \iiint\limits_E \bigl(x^2 + z^2\bigr)\rho(x, y, z)\,\d V &
								I_z &= \iiint\limits_E \bigl(x^2 + y^2\bigr)\rho(x, y, z)\,\d V
					\end{align*}
				The total \textbf{electric charge} on a solid occupying region $E$ with charge density $\sigma(x, y, z)$.
					\[Q = \iiint\limits_E \sigma(x, y, z)\,\d V\]
				The \textbf{joint density function} of 3 random variables $X$, $Y$, and $Z$ is a function of 3 variables such that the probability that $(X, Y, Z)$ lies in $E$ is
					\[P((X, Y, Z) \in E) = \iiint\limits_E f(x, y, z)\,\d V\]
					In particular,
					\[P(x_1 \le X \le x_2 \land y_1 \le Y \le y_2 \land z_1 \le Z \le z_2) = \int_{x_1}^{x_2}\int_{y_1}^{y_2}\int_{z_1}^{z_2} f(x, y, z)\,\d z\,\d y\,\d x\]
					The joint density functions satisfies
						\begin{align*}
							f(x, y, z) &\ge 0 &
									\iiint\limits_{\R^3} f(x, y, z)\,\d V &= 1
						\end{align*}
	\section{Triple Integrals in Cylindrical Coordinates}
		\subsection*{Cylindrical Coordinates}
			\addcontentsline{toc}{subsection}{Cylindrical Coordinates}
			The \textbf{cylindrical coordinate system} represents a point as an ordered triple $(r, \theta, z)$, where $(r, \theta)$ is the projection of the point onto the polar plane and $z$ is the displacement perpendicular to the plane from that projection.
				\[\begin{tikzpicture}[xscale = 5, yscale = 5]
							\draw[thick, ->] (0,0,0) coordinate(O) node[anchor = east]{$O$} -- (1, 0, 0) coordinate(Y) node[anchor = north east]{$y$};
							\draw[thick, ->] (0, 0, 0) -- (0, 1, 0) coordinate(Z) node[anchor = north west]{$z$};
							\draw[thick, ->] (0, 0, 0) -- (0, 0, 1) coordinate(X) node[anchor = south east]{$x$};
							\filldraw (1, 1.3, 1) coordinate(P) circle (0.05mm);
							\node[anchor = east] at (1, 1.3, 1) {$(r, \theta, z)$};
							\draw[dashed] (1, 1.3, 1) -- (1, 0, 1) coordinate(PJ);
							\draw[dashed] (1, 0, 1) -- (0, 0, 0);
							\draw[decoration = {brace, raise = 5pt}, decorate] (0, 0, 0) -- node[anchor = south west, inner sep = 6pt]{$r$} (1, 0, 1);
							\path pic[draw, angle radius = 5mm, "$\theta$", angle eccentricity = 1.5, anchor = north, inner sep = -5pt]{angle = X--O--PJ};
							\draw[decoration = {brace, mirror, raise = 5pt}, decorate] (1, 0, 1) -- node[anchor = west, inner sep = 8pt]{$z$} (1, 1.3, 1);
					\end{tikzpicture}\]
			Converting from cylindrical to rectangular coordinates, 
				\begin{align*}
					x &= r\cos\theta &
							y &= r\sin\theta &
							z &= z
				\end{align*}
				Performing the converse operation,
				\begin{align*}
					\theta &= \arctan\Bigl(\frac{y}{x}\Bigr) &
							r &= \sqrt{x^2 + y^2} &
							z &= z
				\end{align*}
			\subsection*{Triple Integrals in Cylindrical Coordinates}
				If $f$ is continuous and $E$ is a type 1 region with projection in the $xy$-plane $D$ defined as
					\[E = \{(x, y, z) \mid (x, y) \in D \land u_1(x, y) \le z \le u_2(x, y)\}\]
					where $D$ is given in polar coordinates by
					\[D = \{(r, \theta) \mid \alpha \le \theta \le \beta \land h_1(\theta) \le r \le h_2(\theta)\}\]
					then
					\[\iiint\limits_E f(x, y, z)\,\d V = \int_\alpha^\beta\int_{h_1(\theta)}^{h_2(\theta)}\int_{u_1(r\cos\theta, r\sin\theta)}^{u_2(r\cos\theta, r\sin\theta)} f(r\cos\theta, r\sin\theta, z)r\,\d z\,\d r\,\d\theta\]
	\section{Triple Integrals in Spherical Coordinates}
		\subsection*{Spherical Coordinates}
			\addcontentsline{toc}{subsection}{Spherical Coordinates}
			The \textbf{spherical coordinate system} represents a point as an ordered triple $(\rho, \theta, \varphi)$ where $\rho$ is the distance to the origin, $\theta$ is the angle in the projection of the point onto the $xy$-plane, and $\varphi$ is the angle from the positive $z$-axis to the line segment connecting the point to the origin.
					\[\begin{tikzpicture}[xscale = 5, yscale = 5]
							\draw[thick, ->] (0,0,0) coordinate(O) node[anchor = east]{$O$} -- (1, 0, 0) coordinate(Y) node[anchor = north east]{$y$};
							\draw[thick, ->] (0, 0, 0) -- (0, 1, 0) coordinate(Z) node[anchor = north west]{$z$};
							\draw[thick, ->] (0, 0, 0) -- (0, 0, 1) coordinate(X) node[anchor = south east]{$x$};
							\draw[-] (0, 0, 0) -- (1, 1.3, 1) coordinate(P) node[anchor = west] {$(\rho, \theta, \phi)$};
							\filldraw (1, 1.3, 1) circle (0.05mm);
								\draw[decoration = {brace, raise = 5pt}, decorate] (0, 0, 0) -- node[anchor = south east, inner sep = 5pt]{$\rho$}(1, 1.3, 1);
							\draw[dashed] (1, 1.3, 1) -- (1, 0, 1) coordinate(PJ);
							\draw[dashed] (1, 0, 1) -- (0, 0, 0);
							\path pic[draw, angle radius = 5mm, "$\theta$", angle eccentricity = 1.5, anchor = south, inner sep = -5pt]{angle = X--O--PJ};
							\path pic[draw, angle radius = 13mm, "$\varphi$", angle eccentricity = 1.5, anchor = north, inner sep = 5pt]{angle = P--O--Z};
					\end{tikzpicture}\]
				Converting from spherical to rectangular coordinates,
					\begin{align*}
						x &= \rho\sin\phi\cos\theta &
								y &= \rho\sin\phi\sin\theta &
								z &= \rho\cos\phi
					\end{align*}
					Performing the converse operation, 
					\begin{align*}
						\rho &= \sqrt{x^2 + y^2 + z^2} &
								\theta &= \arctan\Bigl(\frac{y}{x}\Bigr) &
								\phi &= \arccos\Bigl(\frac{z}{\rho}\Bigr)
					\end{align*}
		\subsection*{Triple Integrals in Spherical Coordinates}
			\addcontentsline{toc}{subsection}{Triple Integrals in Spherical Coordinates}
				A \textbf{spherical wedge} $E$ is defined as
					\[E = \{(\rho, \theta, \phi) \mid a \le \rho \le b \land \alpha \le \theta \le \beta \land \delta \le \phi \le \gamma\}\]
				\callout{17}{
					For a spherical wedge $E$ given by
						\[E = \{(\rho, \theta, \phi) \mid a \le \rho \le b \land \alpha \le \theta \le \beta \land \delta \le \phi \le \gamma\}\]
						the integral of $f(x, y, z)$ over $E$ is
						\[\iiint\limits_E f(x, y, z)\,\d V = \int_\delta^\gamma\int_\alpha^\beta\int_a^b f(\rho\sin\phi\cos\theta, \rho\sin\phi\sin\theta, \rho\cos\phi)\rho^2\sin\phi\,\d\rho\,\d\theta\,\d\phi\]
				}
	\section{Change of Variables in Multiple Integrals}
		In one-dimensional calculus, the substitution rule can be written as 
			\[\int f(x)\,\d x = \int f(x(u))\der{x}{u}\,\d x\]
			\subsection*{Change of Variables in Double Integrals}
				\addcontentsline{toc}{subsection}{Change of Variables in Multiple Integrals}
				A change of variables can generally be described by a \textbf{transformation} $T$ to from the $uv$-plane to the $xy$-plane.
					\[T(u, v) = (x, y)\]
					$x$ and $y$ are related to $u$ and $v$ by
					\begin{align*}
						x &= g(u, v) &
								y &= h(u, v)
					\end{align*}
				It is typically assumed that $T$ is a \textbf{$\bm{C^1}$ transformation}, meaning that $T$ is a two-variable function with real inputs and outputs. \\
				The output of an input is its \textbf{image}. If no two points share an image, the transformation is \textbf{one-to-one}. \\
				If $T$ is a one-to-one transformation, it has an inverse \textbf{inverse transformation} $T^{-1}$ from the $xy$-plane to the $uv$-plane.
					\begin{align*}
						u &= G(x, y) &
								v &= H(x, y)
					\end{align*}
				\callout{17}{
					The \textbf{Jacobian} of transformation $T$ given by $x = g(u, v)$ and $y = h(u, v)$ is
						\[
							\def\arraystretch{1.4}
							\pder{(x, y)}{(u, v)} =
									\begin{vmatrix}
										\pder{x}{u} & \pder{x}{v} \\
										\pder{y}{u} & \pder{y}{v}
									\end{vmatrix} =
									\pder{x}{u}\pder{y}{v} - \pder{x}{v}\pder{y}{u}
						\]
				}
				\callout{17}{\paragraph{Change of Variables in a Double Integral}
					If $T$ is a one-to-one $C^1$ transformation with nonzero Jacobian that maps region $S$ in the $uv$-plane to region $R$ in the $xy$-plane, $f$ is continuous on $R$ , and $R$ and $S$ are type \Roman{1} or \Roman{2} plane regions, then
						\[\iint\limits_R f(x, y)\,\d A = \iint\limits_D f(x(u, v), y(u, v))\left|\pder{(x, y)}{(u, v)}\right|\d u\,\d v\]
				}
			\subsection*{Change of Variables in Triple Integrals}
				\addcontentsline{toc}{subsection}{Change of Variables in Triple Integrals}
				Let $T$ be a one-to-one transformation that maps region $S$ in  $uvw$-space to region $R$ in the $xyz$-space.
					\begin{align*}
						x &= g(u, v, w) &
								y &= h(u, v, w) &
								z &= k(u, v, w)
					\end{align*}
					The \textbf{Jacobian} of $T$ is
						\[
							\def\arraystretch{1.4}
							\pder{(x, y, z)}{(u, v, w)} = 
								\begin{vmatrix}
									\pder{x}{u} & \pder{x}{v} & \pder{x}{w} \\
									\pder{y}{u} & \pder{y}{v} & \pder{y}{w} \\
									\pder{z}{u} & \pder{z}{v} & \pder{z}{w}
								\end{vmatrix}
						\]
					\callout{17}{
						\[\iiint\limits_R f(x, y, z) = \iiint\limits_S f(x(u, v, w), y(u, v, w), z(u, v, w))\left|\pder{(x, y, z)}{u, v, w}\right|\d u\,\d v\,\d w\]
					}
\end{document}