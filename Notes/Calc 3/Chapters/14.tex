\documentclass[../Calculus_\Roman{3}]{subfiles}

\begin{document}
	\section{Functions of Several Variables}
		\subsection*{Functions of Two Variables}
			\addcontentsline{toc}{subsection}{Functions of Two Variables}
			\callout{17}{
				A \textbf{function of two variables} is a rule that assigns to each ordered pair of real numbers in its domain to a real number in its range.
			}
			For a function $z = f(x, y)$, $x$ and $y$ are the \textbf{independent variables} and $z$ is the \textbf{dependent variable}. \\
			A function of two variables is a function with its domain as a subset of $\R^2$ and its range is a subset of $\R$.
		\subsection*{Graphs}
			\addcontentsline{toc}{subsection}{Graphs}
			\callout{17}{
				The \textbf{graph} of two-variable function $f$ is the set of all points $(x, y, z)$ in $\R^3$ such that $z = f(x, y)$ and $(x, y)$ is in a subset of $\R^2$.
			}
		\subsection*{Level Curves and Contour Maps}
			\callout{17}{
				The \textbf{level curves} of a function $F$ of two variables are the curves with equations $f(x, y) = k$ where $k$ is a constant within the range of $f$.
			}
			A collection of level curves is a \textbf{contour map}.
	\section{Limits and Continuity}
		\subsection*{Limits of Functions of Two Variables}
		\addcontentsline{toc}{subsection}{Limits of Functions of Two Variables}
			The notation
				\[\lim_{(x, y) \to (a, b)} f(x, y) = L\]
				is used to denote that the values of $f(x, y)$ approaches $L$ as $x$ approaches $a$ and $y$ approaches $b$.
			\callout{17}{
				For a function of two variables $f$, the \textbf{limit of $\bm{f(x, y)}$ as $\bm{(x, y)}$ approaches $\bm{a, b}$} is denoted as
				\[\lim_{(x, y) \to (a, b)} f(x, y)\].
				This is equal to $L$ if, for every $\epsilon > 0$ there is a corresponding number $\delta > 0$ such that 
					\[(x, y) \in D \land 0 < \sqrt{(x - a)^2 + (y - b)^2} < \delta \implies |f(x, y) - L| < \epsilon\]
			}
		\subsection*{Showing that a Limit Does Not Exist}
			\addcontentsline{toc}{subsection}{Showing that a Limit Does Not Exist}
				The limit does not exist at $(a, b)$ if $f$ approaches two different values when approaching from different paths.
		\subsection*{Properties of Limits}
			\addcontentsline{toc}{subsection}{Properties of Limits}
			The following are true of limits:
				\begin{align*}
					\lim_{x \to a}[f(x) \pm g(x)] &= \lim_{x\ to a}f(x) \pm \lim_{x\ to a}g(x) &
						\lim_{x \to a} cf(x) &= c\lim_{x\to a}f(x) \\
					\lim_{x \to a}[f(x) \times g(x)] &= \lim_{x \to a} f(x) \times \lim_{x \to a} g(x) &
						\lim_{x \to a}\left[\frac{f(x)}{g(x)}\right] &= \frac{\lim_{x \to a}f(x)}{\lim_{x \to a}g(x)}
				\end{align*}
			A \textbf{polynomial} function is a sum of terms in the form $cx^my^n$ where $c$ is a constant and $m$ and $n$ are nonnegative integers. A \textbf{rational function} is a ratio of two polynomials.
		\subsection*{Continuity}
			\addcontentsline{toc}{subsection}{Continuity}
				\callout{17}{
					A function $f(x, y)$ is \textbf{continuous} at $(a, b)$ if
						\[\lim_{(x, y) \to (a, b)}f(x, y) = f(a, b)\]
					It is continuous on an interval if this it is continuous at every point within.
				}
				Polynomials and rational functions are continuous on their entire domains.
		\subsection*{Functions of Three or More Variables}
			\addcontentsline{toc}{subsection}{Functions of Three or More Variables}
				Everything thus far can be extended to functions of more than two variables.
	\section{Partial Derivatives}
		\subsection*{Partial Derivatives of Functions of Two Variables}
			A \textbf{partial derivative} of a two-variable function assumes treats all but the variable being differentiated with respect to as constants. It is the instantaneous rate of change in the direction of the variable. \\
			\callout{17}{
				The partial derivatives of a function $f$ are the functions $f_x$ and $f_y$, defined as
					\begin{align*}
						f_x(x, y) &= \lim_{h \to 0}\left[\frac{f(x + h, y) - f(x, y)}{h}\right] &
						f_y(x, y) &= \lim_{h \to 0}\left[\frac{f(x, y + h) - f(x, y)}{h}\right]
					\end{align*}
			}
			The partial derivative of a function $z = f(x, y)$ with respect to variable $x$ can be denoted
				\[
					f_x(x, y) = f_x
							= \pder{f}{x}
							= \pder{}{x}f(x, y)
							= \pder{z}{x}
							= f_1
							= D_1f
							= D_xf
				\]
			When evaluating a partial derivative with respect to one variable, all other dependent variables can be regarded as constant.
		\subsection*{Interpretations of Partial Derivatives}
			\addcontentsline{toc}{subsection}
			The partial derivative of $f(x, y)$ with respect to $x$ is the slope of the tangent line parallel to the $x$-axis while that with respect to $y$ is the slope of the tangent line parallel to the $y$-axis.
		\subsection*{Higher Derivatives}
			\addcontentsline{toc}{subsection}{Higher Derivatives}
			The partial derivatives of a function of two variables $f$ are themselves functions of two variables with can be differentiated. These are the \textbf{second partial derivatives} of $f$. If $z = f(x, y)$, two of these can be denoted
			\[
				\def\arraystretch{1.4}
				\begin{array}{cccccc}
					(f_x)_x &
						f_{xx} &
						f_{11} &
						\pder{}{x}\left(\pder{f}{x}\right) &
						\pder{^2f}{x^2} &
						\pder{^2z}{x^2} \\
					(f_x)_y &
						f_{xy} &
						f_{12} &
						\pder{}{y}\left(\pder{f}{x}\right) &
						\pder{^2f}{x \partial y} &
						\pder{^2z}{x \partial y}
				\end{array}
			\]
			Using Clairaut's theorem, it can be shown that $f_{xyy} = f_{yxy} = f_{yyx}$ if all are continuous.
	\section{Tangent Planes and Linear Approximations}
		\subsection*{Tangent Planes}
			\addcontentsline{toc}{subsection}{Tangent Planes}
			\callout{17}{
				\paragraph{Equation of a Tangent Plane}
					If function $f$ has continuous partial derivatives, an equation to the tangent plane to the surface $z = f(x, y)$ at $P(x_0, y_0, z_0)$ is
						\[z - z_0 = f_x(x_0, y_0)(x - x_0) + f_y(x_0, y_0)(y - y_0)\]
			}
		\subsection*{Linear Approximations}
			\addcontentsline{toc}{subsection}{Linear Approximations}
			The \textbf{linearization} of $f$ at $(a, b)$ 
\end{document}