\documentclass[./Calculus \Roman{3}.tex]{subfiles}

\begin{document}
	\section{Vector Fields}
		\subsection*{Vector Fields in $\bm{\R^2}$ and $\bm{\R^3}$}
			\addcontentsline{toc}{subsection}{Vector Fields in $\R^2$ and $\R^3$}
			\callout{17}{
				A \textbf{vector field} on $\R^2$ is a function $\vec{F}$ that assigns each point $(x, y)$ in its domain $D$ (some subset of $\R^2$) a two-dimensional vector $\vec{F}(x, y)$. \\
				That on $\R^3$ follows the same definition, simply with 3 dimensions.
			}
		\subsection*{Gradient Fields}
			\addcontentsline{toc}{subsection}{Gradient Fields}
			The gradient of a two-variable function $f$ of $(x, y)$ is
				\[\nabla f(x, y) = \pder{f}{x}\vi + \pder{f}{y}\vj\]
				This is a vector field on $\R^2$.
			A vector field $\vec{F}$ is \textbf{conservative} if there is some scalar function $f$ for which $\vec{F} = \nabla f$. If this is the case, $f$ is a \textbf{potential function} for $\vec{F}$.
	\section{Line Integrals}
		\subsection*{Line Integrals in the Plane}
			\addcontentsline{toc}{subsection}{Line Integrals in the Plane}
			\callout{17}{
				If $f$ is defined along a smooth curve $C$ given by
					\[\vec{r}(t) = x(t)\vi + y(t)\vj\]
					then the \textbf{line integral of $\bm{f}$ along $\bm{C}$} is
					\[\int_C f(x, y)\,\d s = \lim_{n \to \infty}\sum_{i = 1}^n f(x_i^*, y_i^*)\Delta s_i\]
					if this limit exists.
			}
			The formula for line integral of $f(x, y)$ along curve $C$ can be rewritten as 
				\[\int_C f(x, y)\,\d s = \int_a^b\left[f(x(t), y(t))\sqrt{\left(\der{x}{t}\right)^2 + \left(\der{y}{t}\right)^2}\right]\d t\]
			The value of the line integral is independent of the parametrization of the curve so long as the curve is only traversed once from $a$ to $b$.
		\subsection*{Line Integrals with Respect to $\bm{x}$ or $\bm{y}$}
			\addcontentsline{toc}{subsection}{Line Integrals with Respect to $x$ or $y$}
			The \textbf{line integrals of $\bm{f}$ along $\bm{C}$ with respect to $\bm{x}$ and $\bm{y}$} are 
				\begin{align*}
					\int_C f(x, y)\,\d x &= \lim_{n \to \infty}\sum_{i = 1}^n f(x_i^*, y_i^*)\Delta x_i &
							\int_C f(x, y)\,\d y &= \lim_{n \to \infty}\sum_{i = 1}^n f(x_i^*, y_i^*)\Delta y_i
				\end{align*}
				To distinguish, the line integral with respect to $s$ is referred to as the \textbf{line integral with respect to arc length}.
			The parametrization of a curve determines its \textbf{orientation}, the direction of the path traces as $t$ increases.
			If $-C$ denotes a curve with the same points as $C$ but the opposite orientation,
				\begin{align*}
					\int_{-C} f(x, y)\,\d x &= -\int_C f(x, y)\,\d x &
							\int_{-C} f(x, y)\,\d y &= -\int_C f(x, y)\,\d y
				\end{align*}
				Integrating with respect to arc length, though, orientation has no impact on the value, as $s$ is always positive.
				\[\int_{-C} f(x, y)\,\d s = \int_C f(x, y)\,\d s\]
		\subsection*{Line Integrals in Space}
			\addcontentsline{toc}{subsection}{Line Integrals in Space}
			Let $C$ be a smooth space curve given by
				\[\vec{r}(t) = x(t)\vi + y(t)\vj + z(t)\vk\]
				If $f$ is a function of 3 variables that is continuous on some region containing $C$, the \textbf{line integral of $\bm{f}$ along C} (with respect to arc length) is defined as
					\[\int_C f(x, y, z)\,\d s = \lim_{n \to \infty}\sum_{i = 1}^n f(x_i, y_i, z_i)\Delta s_i\]
				This can be evaluated by expanding $s$.
				\[\int_C f(x, y, z)\,\d s = \int_a^b\left[f(x, y, z)\sqrt{\left(\der{x}{t}\right)^2 + \left(\der{y}{t}\right)^2 + \left(\der{z}{t}\right)^2}\right]\d t\]
				This can in turn be more compactly rewritten as
				\[\int_C f(x, y, z)\,\d s = \int_a^b f(\vec{r}(t))|\vec{r}\vps'|\,\d t\]
			Line integrals with respect to $x$, $y$, and $z$ are defined as
				\begin{align*}
					\int_C f(x, y, z)\,\d x &= \lim_{n \to \infty}\sum_{i = 1}^n f(x_i^*, y_i^*, z_i^*)\Delta x_i
							= \int_a^b f(x(t), y(t), z(t))x'(t)\,\d t \\
					\int_C f(x, y, z)\,\d y &= \lim_{n \to \infty}\sum_{i = 1}^n f(x_i^*, y_i^*, z_i^*)\Delta y_i
							= \int_a^b f(x(t), y(t), z(t))y'(t)\,\d t \\
					\int_C f(x, y, z)\,\d z &= \lim_{n \to \infty}\sum_{i = 1}^n f(x_i^*, y_i^*, z_i^*)\Delta y_i
							= \int_a^b f(x(t), y(t), z(t))z'(t)\,\d t
				\end{align*}
		\subsection*{Line Integrals of Vector Fields: Work}
			\addcontentsline{toc}{subsection}{Line Integrals of Vector Fields: Work}
			Work done as a particle moves along $C$ experiencing variable force $F$ is
				\[W = \int_C \left[\vec{F} \cdot \vec{T}\right]\d s\]
				often abbreviated as
				\[\int_C \vec{F} \cdot \d\vec{r}\]
				This can be rewritten using the definition of $\vec{T}$.
				\[W = \int_a^b\left[\vec{F}(\vec{r}(t)) \cdot \vec{r}\vps'(t)\right]\d t\]
			\callout{17}{
				If $\vec{F}$ is a continuous vector field given by vector function defined on smooth curve $C$ given by $\vec{r}(t)$ for $a \le t \le b$, the \textbf{line integral of $\bm{\vec{F}}$ along $\bm{C}$} is
					\[\int_C \vec{F} \cdot \d\vec{r} = \int_a^b\left[\vec{F}(\vec{r}(t)) \cdot \vec{r}\vps'(t)\right]\d t 
							= \int_C \vec{F} \cdot \vec{T}\,\d s\]
			}
			A continuous vector field's integral along curve $C$ given by $\vec{r}(t)$ for $a \le t \le b$ can be split into its components as
				\[\int_C F \cdot \d\vec{r} = \int_C[P\,\d x + Q\,\d y + R\,\d z]\]
				where
				\[\vec{F} = \langle P, Q, R \rangle\]
	\section{The Fundamental Theorem for Line Integrals}
		The Fundamental Theorem of Calculus states that
			\[\int_a^b f'(x)\,\d x = f(b) - f(a)\]
			where $f'$ is continuous on $[a, b]$.
		\subsection*{The Fundamental Theorem for Line Integrals}
			\addcontentsline{toc}{subsection}{The Fundamental Theorem for Line Integrals}
			\callout{17}{
				If $C$ is a smooth curve given by $\vec{r}(t)$ for $a \le t \le b$ and $f$ is a differentiable function of two or three variables whose gradient vector is continuous on $C$, then
					\[\int_C \nabla f(x) \cdot \,\d\vec{r} = f(\vec{r}(b)) = f(\vec{r}(a))\]
			}
		\subsection*{Independence of Path}
			\addcontentsline{toc}{subsection}{Independence of Path}
			If $C_1$ and $C_2$ are two smooth curves, called \textbf{paths}, with the same initial and terminal points $a$ and $b$, then in general,
				\[\int_{C_1} \vec{F} \cdot \d\vec{r} \ne \int_{C_2} \vec{F} \cdot \d\vec{r}\]
				but
				\[\int_{C_1} \nabla f \cdot \d\vec{r} = \int_{C_2} \nabla f \cdot \d\vec{r}\]
				whenever $\nabla f$ is continuous. \\
				Line integrals of conservative vector fields are \textbf{independent of path}. \\
			A path is \textbf{closed} if its initial and terminal points are the same.
			\callout{17}{
				\[\int_C \vec{F} \cdot \d\vec{r}\]
					is independent of path in $D$ if and only if
					\[\int_C \vec{F} \cdot \d\vec{r} = 0\]
					for every closed path $C$ in $D$.
			}
			\callout{17}{
				If $F$ is a vector field that is continuous on open region $D$, then if
					\[\int_C \vec{F} \cdot \d\vec{r}\]
					is independent of path in $D$, then $\vec{F}$ is a conservative vector field on $D$.
			}
		\subsection*{Conservative Vector Fields and Potential Functions}
			\addcontentsline{toc}{subsection}{Conservative Vector Fields and Potential Functions}
			\callout{17}{
				If $\vec{F}(x, y)$ is a conservative vector field given by
					\[\vec{F}(x, y) = \langle P(x, y), Q(x, y) \rangle\]
					where $P$ and $Q$ have continuous first-order partial derivatives on domain $D$, then throughout $D$,
					\[\pder{P}{y} = \pder{Q}{x}\]
			}
			A \textbf{simple curve} is one that does not intersect itself between its endpoints. \\
			A \textbf{simply connected region} is a contiguous region without any holes.
			\callout{17}{
				If
					\[\vec{F} = \langle P, Q \rangle\]
					is a vector field on a simply-connected region $D$ and $P$ and $Q$ have continuous first-order partial derivatives and
					\[\pder{P}{y} = \pder{Q}{x}\]
					throughout $D$, then, $\vec{F}$ is conservative.
			}
	\section{Green's Theorem}
		\subsection*{Green's Theorem}
			\addcontentsline{toc}{subsection}{Green's Theorem}
			Let $C$ be a simple closed curve and $D$ be the region that it bounds. The \textbf{positive orientation} of $C$ refers to a single \textit{counterclockwise} traversal of $C$.
			\callout{17}{
				Let $C$ be a positively-oriented, piecewise-smooth, simple closed curve in the plane and let $D$ be the region bounded by $C$. If $P$ and $Q$ have continuous partial derivatives on an open region containing $D$, then
					\[\int_C[P\,\d x + Q\,\d y] = \iint\limits_D\left[\pder{Q}{x} - \pder{P}{y}\right]\d A\]
			}
			The notation
				\begin{align*}
					\oint_C[P\,\d x + Q\,\d y] &&\text{or}&& \varointctrclockwise_C[P\,\d x + Q\,\d y]
				\end{align*}
				is sometimes used to denote a positively-oriented closed curve $C$. \\
			The positively oriented boundary curve of region $D$ is sometimes denoted $\partial D$, so Green's Theorem can be rewritten as
				\[\iint\limits_D \left[\pder{Q}{x} - \pder{P}{y}\right]\d A = \int_{\partial D}[P\,\d x + Q\,\d y]\]
		\subsection*{Finding Areas with Green's Theorem}
			\addcontentsline{toc}{subsection}{Finding Areas with Green's Theorem}
			As the area of $D$ is 
				\[\iint\limits_D 1\,\d A\]
				$P$ and $Q$ should be chosen such that
				\[\pder{Q}{x} - \pder{P}{y} = 1\]
				Green's Theorem can then be used to determine the area of $D$ as
				\[A = \oint_C x\,\d y = -\oint_C y\,\d x = \frac{1}{2}\oint_C[x\,\d y - y\,\d x]\]
		\subsection*{Extended Versions of Green's Theorem}
			\addcontentsline{toc}{subsection}{Extended Versions of Green's Theorem}
			Green's Theorem can be extended to the case where $D$ is a finite union of simple regions.
			\[\int_{C_1 \cup C_2}[P\,\d x + Q\,\d y] = \iint\limits_{D_1 \cup D_2}\left[\pder{Q}{x} - \pder{P}{y}\right]\d A\]
	\section{Curl and Divergence}
		\section*{Curl}
			\addcontentsline{toc}{subsection}{Curl}
			The vector differential operator $\nabla$ is defined as
				\[\nabla = \left\langle \pder{}{x}, \pder{}{y}, \pder{}{z} \right\rangle\]
			If $\vec{F} = \langle P, Q, R \rangle$ is a vector field on $\R^3$ and the partial derivatives of each component exist, then the \textbf{curl} of $\vec{F}$ is a vector field on $\R^3$ defined by the cross product of $\nabla$ and $\vec{F}$.
				\[
					\def\arraystretch{1.2}
					\curl\vec{F} = \nabla \times \vec{F} =
						\begin{vmatrix}
							\vi & \vj & \vk \\
							\pder{}{x} & \pder{}{y} & \pder{}{z} \\
							P & Q & R
						\end{vmatrix}
					 	= \left\langle \pder{R}{y} - \pder{Q}{z}, \pder{P}{z} - \pder{R}{x}, \pder{Q}{x} - \pder{P}{y}\right\rangle
				\]
			\callout{17}{
				The curl of a conservative vector field is $\vec{0}$.
					\begin{align*}
						\def\arraystretch{1.2}
						\curl(\nabla f) &= \nabla \times (\nabla f) =
								\begin{vmatrix}
									\vi & \vj & \vk \\
									\pder{}{x} & \pder{}{y} & \pder{}{z} \\
									\pder{f}{x} & \pder{f}{y} & \pder{f}{z}
								\end{vmatrix} \\
							&= \left\langle \pder{^2f}{y\partial z} - \pder{^2f}{z\partial y}, \pder{^2f}{z\partial x} - \pder{^2f}{x\partial z} , \pder{^2f}{x\partial y} - \pder{^2f}{y\partial x}\right\rangle \\
							&= \langle 0, 0, 0 \rangle
									= \vec{0}
					\end{align*}
			}
			\callout{17}{
				If $\vec{F}$ is a vector field defined on all of $\R^3$ with component functions that have continuous partial derivatives and $\curl \vec{F} = \vec{0}$, then $\vec{F}$ is a conservative vector field.
			}
		\section*{Divergence}
			\addcontentsline{toc}{subsection}{Divergence}
			If $\vec{F} = \langle P, Q, R \rangle$ is a vector field on $\R^3$, then the \textbf{divergence} of $\vec{F}$ is the function defined by
				\[\divg \vec{F} = \nabla \cdot \vec{F} = \pder{P}{x} + \pder{Q}{y} + \pder{R}{z}\]
				provided $\partial P/\partial x$, $\partial Q/\partial y$, and $\partial R/ \partial z$ exist.
			\callout{17}{
				If $\vec{F} = \langle P, Q, R \rangle$ is a vector field on $\R^3$ and each component is twice-differentiable, then
					\begin{align*}
						\divg\curl \vec{F} &= \nabla \cdot (\nabla \times \vec{F})
								= \pder{}{x}\left(\pder{R}{y} - \pder{Q}{z}\right) + \pder{}{y}\left(\pder{P}{z} - \pder{R}{x}\right) + \pder{}{z}\left(\pder{Q}{x} - \pder{P}{y}\right) \\
							&= \pder{^2R}{x\partial y} - \pder{^2Q}{x\partial y} + \pder{^2P}{y\partial z} - \pder{^2R}{x\partial y} + \pder{^2Q}{x\partial z} - \pder{^2P}{z\partial y}
								= 0
					\end{align*}
			}
			The \textbf{Laplace operator} is the dot product of the vector differential and itself.
				\[\nabla^2 = \nabla \cdot \nabla\]
		\section*{Vector Forms of Green's Theorem}
			\addcontentsline{toc}{subsection}{Vector Forms of Green's Theorem}
			The line integral of the vector field $\vec{F} = \langle P, Q \rangle$ over boundary curve $C$ of plane region $D$ is
				\[\oint_C \vec{F} \cdot \d\vec{r} = \oint_C[P\,\d x + Q\,\d y]\]
				Regarding $\vec{F}$ as a vector field $\vec{F} = \langle P, Q, 0 \rangle$ in $\R^3$,
				\[
					\curl \vec{F} =
						\begin{vmatrix}
							\vi & \vj & \vk \\
							\pder{}{x} & \pder{0}{y} & \pder{}{z} \\
							P & Q & 0
						\end{vmatrix}
						= \left(\pder{Q}{x} - \pder{P}{y}\right)\vk
				\]
				The integrand of Green's Theorem can therefore be rewritten as
				\[(\curl\vec{F}) \cdot \vk = \left(\pder{Q}{x} - \pder{P}{y}\right)\vk \cdot \vk = \pder{Q}{x} - \pder{P}{y}\]
				Green's Theorem can be rewritten in vector form as
				\[\oint_C \vec{F} \cdot \d\vec{r} = \oint_C \vec{F} \cdot \vec{R}\,\d s = \iint\limits_D (\curl\vec{F}) \cdot \vk \,\d A\]
				This can also be rewritten as
				\[\oint_C \vec{F} \cdot \vec{n} \,\d s = \iint\limits_D \divg\vec{F}(x, y)\,\d A\]
\end{document}