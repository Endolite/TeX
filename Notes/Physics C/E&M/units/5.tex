\documentclass{subfiles}

\begin{document}
	\section{Inductance and Induction}
		\textbf{Faraday's law of induction} states that an emf and a current can be induced in a loop by changing the amount of magnetic field that passes through the loop. This \enquote{amount of magnetic field} can be visualized as the magnetic field lines passing through the loop. \\
		The magnetic flux \(\Phi_B\) through a region \(R\) subject to a magnetic field \(\vec{B}\) is
			\[\Phi_B = \iint\limits_R \vec{B} \cdot \dd{\vec{A}} \tag{magnetic flux}\]
			If the \(\vec{B}\) is perpendicular to \(R\), then this simply becomes
				\[\Phi_B = BA \tag{\(\vec{B}\) perpendicular to \(A\)}\]
		\textbf{Faraday's law} can be more quantitatively stated as
			\[\emf = -\dv{\Phi_B}{t} \tag{Faraday's law}\]
			The negation of the rate of change of the magnetic flux is courtesy of \textbf{Lenz's law}.
			
\end{document}