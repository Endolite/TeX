\documentclass{subfiles}

\begin{document}
	A magnetic field \(\vec{B}\) is defined as a vector quantity that exists when it exerts a force \(\vec{F}_B\) on a charge moving with velocity \(\vec{v}\). The magnitude of this force can be measured when \(\vec{v}\) is perpendicular to it. The magnitude of \(\vec{B}\) can then be defined as
		\[B = \frac{F_B}{|q|v} \tag{magnitude of magnetic field}\]
		where \(q\) is the charge of the particle. It is measured in units of Teslas (\SI{}{T}).
		These can be summarized by the vector equation
		\[\vec{F}_B = q\vec{v} \times \vec{B} \tag{magnetic force}\]
		It should be noted that a motionless charge does not have any force on it due to a magnetic field. \\
		\(\vec{v}\) and \(\vec{B}\) must not be perpendicular in order for there to be a nonzero resultant force. \\
	As with electric fields, magnetic fields can be represented by field lines, the direction of the tangent at a given point giving the direction of \(\vec{B}\) at that point and the spacing of the lines representing its magnitude. \\
	Field lines enter a magnet at one end (the south pole) and exit from the opposite side (the north pole). As such, magnets are called \textbf{magnetic dipoles}. \\
		Opposite poles attract while the like poles repel. \\
	Electric field lines begin and terminate at a charge while magnetic field lines are closed loops. \\
	A compass needle aligns itself with magnetic field lines. \\
	A charged particle moving through a region subject to both electric and magnetic fields is subject to the forces exerted by both. \\
		Two perpendicular fields are said to be \textbf{crossed fields}. \\
		If the forces exerted by the fields oppose each other, a particular speed will result in no deflection of the particle. \\
	The drifting conduction electrons in a copper wire can still be deflected by a magnetic field. This phenomenon is called the \textbf{Hall effect}, which enables the polarity of the charge carriers in a conductor to be found. The number of such charge carriers per unit volume can also be found. \\
	Suppose a uniform magnetic field \(\vec{B}\) is applied perpendicular to the direction of current \(i\). A Hall-effect potential difference \(V\) is set up. The forces exerted by the electric and magnetic fields are then balanced. The number density \(n\) of the charge carriers can be determined as
		\[n = \frac{Bi}{v\ell e}\]
		where
		\[\ell = \frac{A}{d}\]
		is the thickness of the strip. \\
	A conductor moving through a uniform magnetic field \(\vec{B}\) at speed \(v\) has a Hall-effect potential difference is
		\[V = vBD \tag{Hall-effect potential difference}\]
		where \(d\) is the width orthogonal to both \(\vec{v}\) and \(\vec{B}\). 
\end{document}