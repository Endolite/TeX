\documentclass{subfiles}

\begin{document}
	A magnetic field \(\vec{B}\) is defined as a vector quantity that exists when it exerts a force \(\vec{F}_B\) on a charge moving with velocity \(\vec{v}\). The magnitude of this force can be measured when \(\vec{v}\) is perpendicular to it. The magnitude of \(\vec{B}\) can then be defined as
		\[B = \frac{F_B}{|q|v}\]
		where \(q\) is the charge of the particle. It is measured in units of Teslas (\SI{}{T}).
		These can be summarized by the vector equation
		\[\vec{F}_B = q\vec{v} \times \vec{B}\]
		It should be noted that a motionless charge does not have any force on it due to a magnetic field. \\
		\(\vec{v}\) and \(\vec{B}\) must not be perpendicular in order for there to be a nonzero resultant force.
\end{document}