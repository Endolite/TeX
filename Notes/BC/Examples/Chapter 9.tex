\documentclass[12pt, A4]{article}
\usepackage[utf8]{inputenc}
\usepackage{amsmath}
\usepackage{amssymb}
\usepackage[margin=0.5in]{geometry}
\usepackage[shortlabels]{enumitem}

\title{Chapter 9 Examples}

\begin{document}
	\setcounter{section}{9}
	\setcounter{subsection}{2}
	\subsection{}
	\subsubsection*{Integral Test}
		\begin{align*}
			\sum_{n = 1}^\infty\left[\frac{1}{n^2 + 1}\right] \tag{is always positive, continuous, and decreases as \(n\) grows} \\
			\int_1^\infty\left[\frac{1}{x^2 = 1}\right]dx &= \lim_{a\to\infty}[\arctan{x}]_1^a = \lim_{a\to\infty}[\arctan{a} - \arctan{1}] = \frac{\pi}{2} - \frac{\pi}{4} = \frac{\pi}{4} \therefore \sum_{n = 1}^\infty\left[\frac{1}{n^2 + 1}\right]
		\end{align*}
	\subsubsection*{$p$-series}
		\begin{align*}
			\sum_{n = 1}^\infty\frac{1}{\sqrt{n}} &\text{ diverges} \tag{\(p = \frac{1}{2}\ \le 1 \therefore \) diverges} \\
			\sum_{n = 1}^\infty\frac{1}{n} &\text{ diverges} \tag{\(p = 1 \le 1 \therefore\) diverges}
		\end{align*}
	\subsection{Comparison Tests}
		\paragraph{Direct Comparison Test}
			\begin{align*}
				\sum_{n = 1}^\infty\left[\frac{1}{2 + 3^n}\right] && \sum_{n = 1}^\infty\left[\frac{1}{3^n}\right] = \sum_{n = 1}^\infty\left[\frac{1}{3}\right]^n \tag{converges} \\ 
				\frac{1}{2 + 3^n} \le \frac{1}{3^n} \tag{is always true \(\land\) larger series diverges \(\therefore\) original converges} \\
				\sum_{n = 1}^\infty\left[\frac{1}{10 + \sqrt{n}}\right] && \sum_{n = 1}^\infty\left[\frac{1}{\sqrt{n}}\right] \tag{diverges} \\
				\frac{1}{\sqrt{n}} \le \frac{1}{10 + \sqrt{n}} \tag{false} \\
				\sum_{n = 1}^\infty\left[\frac{1}{n}\right] \tag{diverges} \\
				\frac{1}{n} \le \frac{1}{10 + \sqrt{n}} &\hspace{3cm}
					\begin{array}{ccccc} n & 1 & 9 & 16 & 25 \\ \frac{1}{n} & 1 & \frac{1}{9} & \frac{1}{16} & \frac{1}{25}  \\ \frac{1}{10 + \sqrt{n}} & \frac{1}{11} & \frac{1}{13} & \frac{1}{14} & \frac{1}{15} \\ \frac{1}{n} \le \frac{1}{10 + \sqrt{n}} & \rm{False} & \rm{False} & \rm{True} & \rm{True} \end{array} \\
				&\text{\(\frac{1}{n} \le \frac{1}{10 + \sqrt{n}}\) as \(n\) grows larger \(\land\, \frac{1}{n}\) diverges \(\therefore \frac{1}{10 + \sqrt{n}}\) diverges}
			\end{align*}
	\subsection{}
		\paragraph{Alternating Series Test}
			\begin{align*}
				\sum_{n = 1}^\infty\left[\frac{n}{(-2)^{n - 1}}\right] &= \sum_{n = 1}^\infty\left[\frac{n}{(-1 \times 2)^{n - 1}}\right] \\
					&= \sum_{n = 1}^\infty\left[\frac{1}{(-1)^{n - 1}} \times \frac{n}{2^{n - 1}}\right] \\
				\lim_{n\to\infty}\left[\frac{n}{2^{n - 1}}\right] &= \mathrm{\frac{slow}{fast}} = 0 \\
				a_{n + 1} &\le a_n \\ \frac{n + 1}{2^n} &\le \frac{n}{2^{n - 1}} \tag{larger denominator \(\therefore\) true \(\therefore\) converges}
			\end{align*}
	\subsection{}
		\paragraph{Ratio Test}
			\begin{align*}
				\sum_{n = 1}^\infty\left[\frac{2^n}{n!}\right] \\
				\lim_{n\to\infty}\left|\frac{2^{n+1}}{(n+1)!} \times \frac{n!}{2^n}\right| &= \lim_{n\to\infty}\left|\frac{2}{n + 1}\right| = 0 < 1 \therefore \text{converges}
			\end{align*}
		\subparagraph{Factorials}
			\begin{align*}
				(n + 1)! &= n!(n + 1) \\
				(3n + 4)! &= (3n)!(3n + 4)(3n + 3)(3n + 2)(3n + 1) \\
				(an + b)! &= (an)!(an + b)(an + b - 1)(an + b - 2) \cdots = (an)!\prod_{i = 0}^{b - 1}(an + b - i) = (an)!\prod_{i = 1}^b(an + i)\\
				(0 + 1)! &= 0!(0 + 1) \\
				1! &= 0!(1) \\
				1 &= 0!
			\end{align*}
		\paragraph{Root Test}
			\begin{align*}
				\sum_{n = 1}^\infty\left[\frac{e^{2n}}{n^n}\right] \\ \lim_{n\to\infty}\left(\frac{e^{2n}}{n^n}\right)^{1/n} &= \lim_{n\to\infty}\left(\frac{e^2}{n}\right) = 0 < 1 \therefore\text{converges}
			\end{align*}
	\subsection{Power Series}
		\begin{align*}
			&\sum_{n = 1}^\infty\left[\frac{(x - 2)^n}{n}\right] \\
			\lim_{n\to\infty}\left|\frac{a_{n + 1}}{a_n}\right| &< 1 \\
			\lim_{n\to\infty}\left|\frac{(x-2)^{n + 1}}{n + 1} \times \frac{n}{(x - 2)^n}\right| &< 1 \\
			\lim_{n\to\infty}|(x - 2) \times 1| < 1 \\ 
			|x - 2| &< 1 \\
			x - 2 &< 1 & x - 2 &> -1 \\
			x &< 3 & x &> 1 \\
			1 &< x < 3 \\
			&\sum_{n = 1}^\infty\left[\frac{(-1)^n}{n}\right] \\
		\end{align*}
	\subsection{Taylor and Maclaurin Polynomials}
		\begin{align*}
			f(x) &= \sqrt{x + 1} \\
				&\footnotesize\begin{array}{c|c|c|c|c}
					f(x) = \sqrt{x + 1} & f'(x) = \frac{1}{2}(x + 1)^{-1/2} & f''(x) = \frac{-1}{4}(x + 1)^{-3/2} & f^{(3)}(x) = \frac{3}{8}(x + 1)^{-5/2} & f^{(4)}(x) = \frac{-15}{16}(x + 1)^{-7/2}\\\hline
					f(0) = 1 & f'(1) = \frac{1}{2} & f''(0) = \frac{-1}{4} & f^{(3)}(0) = \frac{3}{8} & f^{(4)}(0) = \frac{-15}{16}
				\end{array} \\
			P_4 &= 1 + \frac{1}{2}x - \frac{\frac{1}{4}x^2}{2!} + \frac{\frac{3}{8}x^3}{3!} - \frac{\frac{15}{16}x^4}{4!} = 1 + \frac{1}{2}x - \frac{1}{8}x^2 + \frac{1}{16}x^3 - \frac{5}{128}x^4
		\end{align*}
		\begin{align*}
			y &= \ln(2 + x) \\ 
			\begin{array}{l|l}
				f(x) = \ln(2 + x) & \ln(1) = 0 \\
				f'(x) = (2 + x)^{-1} & (1)^{-1} = 1 \\
				f''(x) = -1(2 + x)^{-2} & -1(-1)^{2} = -1 \\
				f^{(3)}(x) = 2(2 + x)^{-3} & 2(-1)^{-3} = 2 \\
				f^{(4)}(x) = -6(2 + x)^{-4} & -6(-1)^{-4} = -6
			\end{array} \\
			P_4 &= 0 + (1)(x + 1) + \frac{-1(x + 1)^2}{2!} + \frac{2(x + 1)^3}{3!} + frac{-6(x + 1)^4}{4!} \\
				&= x + 1 - \frac{(x + 1)^2}{2} + \frac{x + 1}{3} - \frac{(x + 1)^4}{4} \\ 
				&= \sum_{i = 1}^4\left[\frac{(-1)^{n + 1}(x + 1)^n}{n}\right] \\
			y &= \sum_{n = 1}^\infty\left[\frac{(-1)^{n + 1}(x + 1)^n}{n}\right]
		\end{align*}
		\subsection{Manipulating Known Maclaurin Polynomials}
			\begin{align*}
				\frac{x^2}{1 + x^2} &= x^2\sum_{n = 0}^\infty\left(-x^2\right)^n = \sum_{n = 0}^\infty\left(-1\right)^nx^{2n + 1} \\
				\tan x &= \frac{\sin x}{\cos x} \approx \frac{P_2(\sin x)}{P_2\cos(x)} = \frac{x - \frac{x^3}{3!} + \frac{x^5}{5!}}{1 - \frac{x^2}{2!} + \frac{x^4}{4!}} = \frac{x - \frac{x^3}{6} + \frac{x^5}{120}}{1 - \frac{x^2}{2} + \frac{x^4}{24}} = x \\
				P_3(\arctan x) &= P_3\left(\int\left[\frac{1}{1 + x^2}\right]dx\right) = \int\left[1 - x^2 + x^4 - x^6\right] = x - \frac{x^3}{3} + \frac{x^5}{5} - \frac{x^7}{7} \\
				P_2\left(e^{-x^2}\arctan{x}\right) &= \left(1 - x^2 + \frac{x^4}{2}\right)\left(x - \frac{x^3}{3} + \frac{x^5}{5}\right) \\
					&= x - \frac{x^3}{3} + \frac{x^5}{5} - x^3 + \frac{x^5}{3} - \frac{x^7}{5} + \frac{x^5}{3} - \frac{x^7}{6} + \frac{x^9}{10} = x - \frac{4x^3}{3} + \frac{31x^5}{30}
			\end{align*}
\end{document}