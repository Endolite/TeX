\documentclass[./MATH137.tex]{subfiles}
\sectionb{Assignment 2}

\begin{document}
	\begin{tasks}
		\task
			\textbf{Claim.} \(L \ge 0\). \\
			\textit{Proof.}
				Assume that \(L < 0\). This means that there is some \(N \in \N\) for which \(n \ge N\) means that for any \(\varepsilon > 0\),
					\[|a_n - L| < \varepsilon\]
					As \(a_n > 0\) and \(L < 0\),
					\[|a_n - L| = a_n - L\]
					Letting \(\varepsilon = L\),
					\begin{align*}
						a_n - L &\le L \\
						a_n & \le 2L < 0
					\end{align*}
					which is a contradiction, so \(L \ge 0\). \(\square\).
			\task
				\textbf{Claim.} \[\lim_{n \to \infty} \frac{2}{a_n + 5} = \frac{2}{L + 5}\]
				\textit{Proof.}
					Let \(a_n \to L \ge 0\); that is, for any \(\varepsilon > 0\), there is some cutoff \(N \in \N\) for which \(n \ge N\) implies
						\[|a_n - L| < \varepsilon\]
						Letting \(n > N\),
						\begin{align*}
							\abs{\frac{2}{a_n + 5} - \frac{2}{L + 5}} &= \abs{\frac{2(L + 5) - 2(a_n + 5)}{(a_n + 5)(L + 5)}} \\
								&\le \frac{2|L - a_0|}{5(L + 5)} \\
								&< \frac{2\varepsilon}{5L + 25}
						\end{align*}
						Defining 
						\[\varepsilon_1 = \frac{2\varepsilon}{5L + 25}\]
						it is clear that
						\[\lim_{n \to \infty} \frac{2}{a_n + 5} = \frac{2}{L + 5} \qquad \square\]
	\end{tasks}
\end{document}
