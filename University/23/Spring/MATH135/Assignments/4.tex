\documentclass[../MATH135.tex]{subfiles}

\begin{document}
	\sectionb{Assignment 4}
		\begin{enumerate}
			\item
				\begin{tasks}
					\task	
						If \(m \ge n\), \((m - n)! = 0\), meaning that the denominator of \(\binom{n}{m}\) would be 0.
					\task
						The bounds of summation were changed so that in the inductive step, the sums from \(P(k)\) and \(P(k + 1)\) could be combined, enabling PI to be applied before changing the bounds again to prove \(P(k + 1)\).
					\task
						The upper bound of the summation for the case with \(a = 0\) was \(m = n\), making \(a^{m - n} = 0^0\).
					\task
						\textbf{Claim.} \[\forall a \in \Z, \forall n \in \N, a \mid \bigl((2 + a)^n - 2^n\bigr)\]
						\textit{Proof.}
							This is a proof by induction on \(n\), where \(P(x)\) is
								\[\forall a \in \Z, a \mid \bigl((2 + a)^n - 2^n\bigr)\]
							\textit{Base Case:} Let \(n = 1\). Then
								\[2 + a - 2 = a\]
								By the definition of divisibility, for all \(a \in \Z\), \(a \mid a\), so \(P(1)\) is true. \\
							\textit{Inductive Step:} Assume that for some \(k \in \N\), \(P(k)\) is true; that is,
								\[\forall a \in \Z, a \mid \bigl((2 + a)^k - 2^k\bigr)\]
								By BT2 and the definition of divisibility, the expression becomes
								\begin{align*}
									(2 + a)^k - 2^k &= \sum_{i = 0}^k\qty[\binom{k}{i}2^{k - i}a^i] - 2^k \\
										&= xa
								\end{align*}
								for some \(x \in \Z\). \\
								The expression for \(P(k + 1)\) is
								\[(2 + a)^{k + 1} - 2^{k + 1} = \sum_{i = 0}^{k + 1}\qty[\binom{k + 1}{i}2^{k + 1 - i}a^i] - 2^{k + 1}\]
								by BT2. By Pascal's identity and the induction hypothesis,
								\begin{align*}
									(2 + a)^{k + 1} - 2^{k + 1} &= \binom{k + 1}{0}2^{k + 1}a^0 + \sum_{i = 1}^{k + 1}\qty[\qty(\binom{k}{i - 1} + \binom{k}{i})2^{k + 1 - i}a^i] - 2^{k + 1} \\
										&= 2^{k + 1} + a\sum_{i = 0}^{k}\qty[\binom{k}{i}2^{k - i}a^i] \\
											&\,\,\,+ 2\sum_{i = 0}^k\qty[\binom{k}{i}2^{k - i}a^i] - \binom{k}{0}2^{k + 1}a^0 + \binom{k}{k + 1}2^0a^{k + 1} - 2^{k + 1} \\
										&= a\qty(xa + 2^k) + 2\qty(xa + 2^k) - 2^{k + 1} + 0  \\
										&= a\qty(xa + 2^k) + 2xa + 2^{k + 1} - 2^{k + 1} \\
										&= a\qty(xa + 2^k + 2x)
								\end{align*}
								As \(xa + 2^k + 2x \in \Z\), \((2 + a)^{k + 1} - 2^{k + 1}\) is by definition divisible by \(a\), making \(P(k + 1)\) true. By the Principle of Mathematical Induction, \(P(n)\) is true for all \(n \in \N\). \(\square\) \\
							
				\end{tasks}
				\textit{Alternate Proof.}
					Let \(a \in \Z\) and \(n \in \N\). By BT2,
						\begin{align*}
							(2 + a)^n - 2^n &= \sum_{i = 0}^n\qty[\binom{n}{i}2^{n - i}a^i] - 2^n \\
								&= 2^n + \sum_{i = 1}^n\qty[\binom{n}{i}2^{n - i}a^i] - 2^n \\
								&=  a\sum_{i = 1}^n\binom{n}{i}2^{n - i}a^i
						\end{align*}
						As \(\sum\limits_{i = 1}^n\binom{n}{i}2^{n - i}a^i \in \Z\), \(a \mid \qty((2 + a)^n - 2^n)\).
			\item
				\textbf{Claim.} \[\forall n \in \Z, \bigl(4 \nmid \qty(n^4 + 3)\bigl) \iff (n \mid 2)\]
				\textit{Proof.}
					Let \(n \in \Z\). \\
				\textit{Case 1.} \(n \nmid 2\). \\
					This means that for some \(k \in \Z\), \(n = 2k + 1\), so
						\begin{align*}
							n^4 + 3 &= (2k + 1)^4 + 3 \\
								&= (2k)^4 + 4(2k)^3 + 6(2k)^2 + 4(2k) + 1 + 3 \\
								&= 16k^2 + 24k^3 + 24k^2 + 8k + 4 \\
								&= 4\qty(4k^2 + 6k^3 + 6k^2 + 2k + 1)
						\end{align*}
						\(4k^2 + 6k^3 + 6k^2 + 2k + 1 \in \Z\), so by the definition of divisibility, \((n \nmid 2) \simp \bigl(4 \mid (n^4 + 3)\bigr)\).
				\textit{Case 2.} \(n \mid 2\). \\
					In this case, for some \(k \in \Z\), \(n = 2k\), so
						\begin{align*}
							n^4 + 3 &= (2k)^4 + 3 \\
								&= 16k^4 + 3 \\
								&= 4(4k^4) + 3
						\end{align*}
					As \(4k^4 \in \Z\) and \(4 \nmid 3\), \((n \mid 2) \simp \bigl(4 \nmid (n^4 + 3)\bigr)\). \\
					This means that \(\bigl(4 \nmid \qty(n^4 + 3)\bigl) \iff (n \mid 2)\). \(\square\)
			\item
				\textbf{Claim.} \(\forall a \in \N, \forall n \in \Z, (a^2 - 10) \ne n^2\). \\
				\textit{Proof.}
					Let \(a \in \N\), and assume that for some \(n \in \Z\), 
						\[a^2 - 10 = n^2\]
						This means that
						\[a^2 - n^2 = 10\]
						This implies that there are two perfect squares separated by distance 10. The sequence of the difference of the perfect squares is given by
						\begin{align*}
							c_i &= (i + 1)^2 - i^2 \\
								&= i^2 + 2i + 1 - i^2 \\
								&= 2i + 1
						\end{align*}
						It is clear that this difference is always odd, meaning that it can never be equal to 10. There is therefore no \(a \in \N\) that is 10 more than a perfect square. \(\square\)
			\item
				\textbf{Claim.} \[\forall (w, x, y, z \in \Z), \bigl((w \ne y) \land (wz - xy \ne 0)\bigr) \simp \exists!r \in \Q, \dfrac{wr + x}{yr + z} = 1\]
				\textit{Proof.}
					Let \(w, x, y, z \in \Z\) with \(w \ne y\) and \(wz \ne xy\). Assume that for some \(r \in \Q\),
					\begin{align*}
						\frac{wr + x}{yr + z} &= 1 \\
						wr + x &= yr + z \\
						wr - yr &= z - x \\
						r &= \frac{z - x}{w - y}
					\end{align*}
					As \(w \ne y\), the denominator is not 0. Substituting this back into the equation,
					\begin{align*}
						1 &= \frac{wr + x}{yr + z} \\
							&= \frac{w\frac{z - x}{w - y} + x}{y\frac{z - x}{w - y} + z} \\
							&= 	\frac{w(z - x) + x(w - y)}{y(z - x) + z(w - y)} \\
							&= \frac{wz - wx + wx - xy}{yz - xy + wz - yz} \\
							&= \frac{wz - xy}{wz - xy} \\
							&= 1
					\end{align*}
					meaning that this solution is not extraneous. (\(wz \ne xy\), meaning that the second to last expression is not indeterminate.) The expression for \(r\) is dependent on the values of the parameters alone, meaning that it is simply the ratio of two fixed differences. There is therefore only one such \(r\) that may exist. \(\square\)
			\item
				\textbf{Claim}
					\[\forall n \in \N, \sum_{i = 1}^n (-1)^ii^2 = \frac{(-1)^nn(n + 1)}{2}\]
				\textit{Proof.}
					This is a proof by induction on \(n\), where \(P(n)\) is the statement
						\[\sum_{i = 1}^n (-1)^ii^2 = \frac{(-1)^nn(n + 1)}{2}\]
					\textit{Base Case.}
						For \(n = 1\),
							\begin{align*}
								\sum_{i = 1}^1 (-1)^ii^2 &= (-1)^11^2 \\
									&= -1 \\
									&= -\frac{2}{2} \\
									&= \frac{(-1)^11(1 + 2)}{2}
							\end{align*}
							so \(P(1)\) is true. \\
					\textit{Inductive Step}
						Assume that for some \(k \in \N\), \(P(k)\) is true; that is,
							\[\sum_{i = 1}^k (-1)^ii^2 = \frac{(-1)^kk(k + 1)}{2}\]
							Adding the next term to the sum yields
							\begin{align*}
								\sum_{i = 1}^{k + 1} (-1)^ii^2 &= (-1)^{k + 1}(k + 1)^2 + \sum_{i = 1}^k(-1)^ii^2 \\
									&= (-1)^{k + 1}(k + 1)^2 + \frac{(-1)^kk(k + 1)}{2} \\
									&= \frac{2(-1)^{k + 1}(k + 1)^2 + (-1)^kk(k + 1)}{2} \\
									&= \frac{(-1)^{k + 1}\bigl(2(k^2 + 2k + 1) - (k^2 + k)\bigr)}{2} \\
									&= \frac{(-1)^{k + 1}(2k^2 + 4k + 2 - k^2 - k)}{2} \\
									&= \frac{(-1)^{k + 1}(k^2 + 3k + 2)}{2} \\
									&= \frac{(-1)^{k + 1}(k + 1)(k + 2)}{2} \\
									&= \frac{(-1)^{k + 1}(k + 1)\bigl((k + 1) + 1)}{2}
							\end{align*}
							so \(P(k + 1)\) is also true. \\
							By the Principle of Mathematical Induction, \(P(n)\) is true for all \(n \in \N\). \(\square\)
			\item
				\textbf{Claim.}
					\[\forall x, y \in \R, (0 < y < x) \simp \forall n \in \N, x^n - y^n \le nx^{n - 1}(x - y)\]
				\textit{Proof.}
					Let \(x, y \in \R\) with \(0 < y < x\). This is a proof by induction on \(n\), where \(P(n)\) is
						\[x^n - y^n \le nx^{n - 1}(x - y)\]
					\textit{Base Case.}
						Let \(n = 1\). Then
							\begin{align*}
								x^1 - y^1 &\le (1)x^{1 - 1}(x - y) \\
								x - y &\le x - y
							\end{align*}
							so \(P(1)\) is true. \\
					\textit{Inductive Step.}
						Assume that for some \(k \in \N\), \(P(k)\) is true; that is,
							\[x^k - y^k \le kx^{k - 1}(x - y)\]
							Multiplying by \(x\) on both sides and expanding
							\[x^{k + 1} - xy^k \le kx^{k + 1} - kx^ky\]
							Adding \(xy^k - y^{k + 1}\) to both sides and recalling that \(0 < y < x\),
							\begin{align*}
								x^{k + 1} - y^{k + 1} &\le kx^{k + 1} - kx^ky + xy^k - y^{k + 1} \\
									&= kx^{k + 1} - kx^ny + xy^k\qty(1 - \frac{y}{x}) \\
									&< kx^{k + 1} - kx^ky + x^{k + 1}\qty(1 - \frac{y}{x}) \\
									&= kx^{k + 1} - kx^ky + x^{k + 1} - x^ky\\
									&= (k + 1)x^k(x - y) \\
							\end{align*}
							making \(P(k + 1)\) true as well. \\
							By the Principle of Mathematical Induction, \(P(n)\) is true for all \(n \in \N\). \(\square\)
		\end{enumerate}
\end{document}
