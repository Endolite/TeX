\documentclass[../MATH135.tex]{subfiles}

\begin{document}
	\sectionb{Assignment 1}
		\begin{enumerate}
			\item
				\begin{tasks}
					\task
						The smallest prime number can either be 2 or not 2, making \enquote{The smallest prime number is 2} a valid statement.
					\task
						The sum of \(\cos^2\theta\) and \(\sin^2\theta\) could be 1 or not 1, so \enquote{\(\cos^2\theta + \sin^2\theta = 1\)} is a valid statement.
					\task
						It is either possible for every integer to be of the form \(2k\) or \(2k + 1\) or there exists at least one exception, so \enquote{Every integer \(x\) is of the form \(2k\) or \(2k + 1\)} is a valid statement.
					\task
						0 can either be even or odd or it could not be either, making \enquote{The number 0 is neither even nor odd} a valid statement.
					\task
						A question is not true or false; therefore, \enquote{Is \(3 > 2\) true?} is not a valid statement.
				\end{tasks}
			\item
				\begin{tasks}(2)
					\task
						\(\forall x \in \Z, x^2 > 0\)
					\task
						\(\forall x \in \R, x^3 \in \R\)				
				\end{tasks}
			\item
				\begin{tasks}(2)
					\task
						\(\exists x \in Z, \forall y \in \R, x + y \ge 3\sqrt{2}\)
					\task
						\(\forall a \in \N, \exists b \in \Q, \forall c \in \Z, a = b - c\)
				\end{tasks}
			\item
				\begin{tasks}
					\task	
						\(P(2, 4) = \exists y \in \Z, 2(2) + 4y = 4 \implies y = 0 \in \Z \implies \text{true}\) \\
						\(P(2, 5) = \exists y \in \Z, 2(2) + 4y = 5 \implies y = 0.25 \notin \Z \implies \text{false}\)
					\task
						There is no condition given for \(n\), meaning that the truth value of \(P(x, n)\) cannot be determined, so \enquote{\(\exists x \in \Z, P(x, n)\)} is an open sentence depending on \(n\). \\
					\task
						As all variables are specified, \enquote{\(\forall n \in \Z, \exists x \in \Z, P(x, n)\)} is a mathematical statement. As the \(2(x + 2y)\) must be even given that \(x\) and \(y\) are integers, this statement is false for all odd values of \(n\), meaning that the statement as a whole is false.
				\end{tasks}
			\item
				\begin{tasks}
					\task
						\begin{align*}
							\left(8^{k^2}\right)(4^k) &= \left(2^{3k^2}\right)\left(2^{2k}\right)
									= 2^{3k^2 + 2k}
									= 2
									\implies 3k^2 + 2k = 1
									\implies 3k^2 + 2k - 1 = 0 \\
								&\implies 0 = (3k - 1)(k + 1)
								\implies k = -1 \in \Z
								\implies \text{true}	
						\end{align*}
					\task
						\[
							x^2 - x + \frac{1}{4} > 0
								\implies \lnot\exists x = \frac{1 \pm \sqrt{1 - 1}}{2} = \frac{1}{2}\in \R
								\implies \text{false}
						\]
					\task
						\[
							\forall x \in \{0, 1, 2, 3\}, \forall y \in \{0, 1, 2, 3\}, 
								(x + y) \in \Z, 
								(x^2 + y^2) \in \Z
								\implies \frac{x + y}{x^2 + y^2} \in \Q
								\implies \text{true}
						\]
					\task
						\[
							4^x + (\ln x)^2 \ge 2x\ln(x^2)
								= 4x\ln x \\
						\]
						As \(4^x\) grows faster than \(x\ln x\), for all \(x \in \N\) and \(4 \ge 0\), the the statement is true.
					\task
						\begin{align*}
							x + 2xy &= 4 \\
							1 + 2y &= \frac{2}{x} \\
							y &= \frac{4}{x} - \frac{1}{2} \\
							x \in \Q &\implies \frac{4}{x} - \frac{1}{2} \in \Q
									\implies \text{true}
						\end{align*}
				\end{tasks}
			\item
				\begin{tasks}(2)
					\task
						\(\forall x \in \{1, 2, 3\}, \forall y \in \{1, 2, 3\}, \dfrac{4680}{x^2 + y^2} \in \Z\)
					\task
						\(\forall x \in \R, \exists y \in \R, x^2 + y^3 = 135\)
					\task
						\(\exists x \in \N, \forall y \in \N, x \le y\)
					\task
						\(\exists x \in \Q, \exists y \in \Q, x^3y^2 = 108\)
				\end{tasks}
		\end{enumerate}
\end{document}