\documentclass[12pt, A4]{report}

\usepackage{amsmath}
\usepackage{amssymb}
\usepackage{bm}
\usepackage{fancyhdr}
\usepackage[hang,flushmargin]{footmisc}
\usepackage[margin=0.75in]{geometry}
\usepackage[utf8]{inputenc}
\usepackage{moresize}
\usepackage{pgf}
\usepackage{tikz}
\usepackage{tikz-3dplot}
\usetikzlibrary{calc}

% Configuration
\author{Arnav Patri}

\pagestyle{fancy}
\fancyhead{}
	\fancyhead[L]{\leftmark}
	\fancyhead[R]{Arnav Patri}
	\renewcommand{\chaptermark}[1]{\markboth{#1}{}}

% Utilities


% Operators

\renewcommand{\d}{\text{d}}

\begin{document}
	\chapter*{Volume: Disk/Washer\markboth{Volume: Disk/Washer}{}}
		\section*{Sources}
			\subsection*{Calculus: Early Transcendentals 9$^{\textbf{th}}$ Edition}
				\begin{enumerate}
					\item
						6.2 Exercise 12
					\item
						6.2 Exercise 15
					\item 
						6.2 Exercise 1
				\end{enumerate}
			\newpage
		\section*{Problems}
			\flushleft{Evaluate the volume of the solid generated by revolving the region bounded by the given equations about the specified line using the disk/washer method.}
			\begin{enumerate}
				\item
					\[
						\begin{aligned}
							y &= 0 &
								y &= \frac{1}{x} \\
							x &= 1 &
								x &= 4
						\end{aligned} \qquad \text{about}\qquad y = 0
					\]
				\item
					\[
						\begin{aligned}
							y &= \frac{x^2}{4} &
								y &= 9 \\
							x &= 0
						\end{aligned} \qquad \text{about}\qquad x = 0
					\]
				\item
					\[
						\begin{aligned}
							y &= 0 & 
								y &= x^2 + 5 \\
							x &= 0 &
								x &= 3
						\end{aligned} \qquad\text{about}\qquad y = 0
					\]
			\end{enumerate}
			\newpage
		\section*{Solutions}
			\begin{enumerate}
				\item
					\begin{align*}
						V &= \pi\int_1^4\left(\frac{1}{x}\right)^2\d x
								 = \pi\left[-\frac{1}{x}\right]_1^4
								 = \pi\left[-\frac{1}{4} -\left(-\frac{1}{1}\right)\right]
								 = \frac{3\pi}{4} \\
					\end{align*}
				\item
					\begin{align*}
						y &= \frac{x^2}{4} \implies x = 2\sqrt{y} \\
						y_1 &= 2\sqrt{0} = 0 \\
						V &= \pi\int_0^9\left(2\sqrt{y}\right)^2\d y = \pi\left[2y^2\right]_0^9 = 2(81)\pi = 162\pi
					\end{align*}
				\item
					\begin{align*}
						V &= \pi\int_0^3\left(x^2 + 5\right)^2\d x 
								= \pi\int_{0}^3\left[x^4 + 10x^2 + 25\right]\d x 
								= \pi\left[\frac{x^5}{5} + \frac{10x^3}{3} + 25x\right]_0^3 \\
							&= \pi\left[\frac{3^5}{5} + \frac{10(3)^3}{3} + 25(3) - (0)\right] 
								= \pi\left[\frac{243}{5} + 90 + 75\right] 
								= \frac{\pi(243 + 825)}{5} 
								= \frac{1068\pi}{5} \\
					\end{align*}
			\end{enumerate}
	\chapter*{Indeterminate Powers (Type 3)\markboth{Indeterminate Powers}{}}
		\section*{Sources}
			\subsection*{Calculus: Early Transcendentals 9$^{\textbf{th}}$ Edition}
				\begin{enumerate}
					\item
						4.4 Exercise 57
					\item
						4.4 Exercise 60
					\item
						4.4 Exercise 61
				\end{enumerate}
			\newpage
		\section*{Problems}
			Evaluate the following limits
			\begin{enumerate}
				\item
					\[\lim_{x\to 0^+}\left[x^{\sqrt{x}}\right]\]
				\item
					\[\lim_{x\to\infty}\left(1 + \frac{a}{x}\right)^{bx}\]
			\end{enumerate}
			\newpage
		\section*{Solutions}
			\begin{enumerate}
				\item
					\begin{align*}
						L &= \lim_{x\to 0^+}\left[x^{\sqrt{x}}\right] 
							&&\implies 0^0 \\
						\ln L &= \lim_{x\to 0^+}\left[\sqrt{x}\ln x\right] 
							&&\implies 0 \times (-\infty) \\
						&= \lim_{x\to 0^+}\left[\frac{\ln x}{x^{-1/2}}\right] 
							&&\implies -\frac{\infty}{\infty} \\
						&= \lim_{x\to 0^+}\left[\frac{\frac{1}{x}}{-\frac{1}{2x^{3/2}}}\right] = \lim_{x\to 0^+}\left[-2\sqrt{x}\right] = 0 \\
						L &= e^0 = 1
					\end{align*}
				\item
					\begin{align*}
						L &= \lim_{x\to\infty}\left(1 + \frac{a}{x}\right)^{bx}
							&&\implies 1^{\infty} \\
						\ln L &= \lim_{x\to\infty}\left[bx\ln\left(1 + \frac{a}{x}\right)\right]
							&&\implies \infty \times 0 \\
						&= \lim_{x\to\infty}\left[\frac{b\ln\left(1 + \frac{a}{x}\right)}{x^{-1}}\right]
							&&\implies \frac{0}{0} \\
						&= \lim_{x\to\infty}\left[\frac{\frac{-bax^{-2}}{1 + \frac{a}{x}}}{-x^{-2}}\right] = \lim_{x\to\infty}\left[\frac{ab}{1 + \frac{a}{x}}\right] = ab \\
						L &= e^{ab}
					\end{align*}
			\end{enumerate}
\end{document}