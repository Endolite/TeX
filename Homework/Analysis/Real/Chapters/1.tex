\documentclass[../Exercises.tex]{subfiles}

\begin{document}
	\setcounter{section}{1}
	\section{Some Preliminaries}
		\subsection{}
			\begin{enumerate}[(a)]
				\item
					Prove that $\sqrt{3}$ is irrational. Does a similar argument work to show $\sqrt{6}$ is irrational? \\
					\subitem \textbf{Solution}
					\subitem
						\emph{Proof}. A rational number can always be expressed as the ratio of two integers $p$ and $q$. The square of a rational number can therefore always be
				\item
					Where does the proof of Theorem 1.1.1 break down if we try to use it to prove $\sqrt{4}$ is irrational?
					\subitem \textbf{Solution}
					\subitem
						
			\end{enumerate}
\end{document}