\documentclass[12pt, A4]{article}

% Packages
	% Basics
		\usepackage{amsmath}
		\usepackage{bm}
		\usepackage{cellspace}
		\usepackage{csquotes}
		\usepackage[shortlabels]{enumitem}
		\usepackage[hang,flushmargin]{footmisc}
		\usepackage[margin=0.75in]{geometry}
		\usepackage{hyperref}
		\usepackage[utf8]{inputenc}
	% Diagrams
		\usepackage{pgfplots}
		\usepackage{tikz}
		\usepackage{tikz-3dplot}
			\usetikzlibrary{arrows.meta, angles, calc, quotes}
	% Formatting
		\usepackage{booktabs}
		\usepackage{tasks}
	% Symbols
		\usepackage{amssymb} % Miscellaneous
		\usepackage{esint} % Integrals
		\usepackage{physics} % Differentials
% Configuration
	\title{Homework Set 1}
	\author{Arnav Patri}
	\hypersetup{
	    colorlinks,
	    citecolor=black,
	    filecolor=black,
	    linkcolor=black,
	    urlcolor=black
	}
	\cellspacetoplimit10pt
	\cellspacebottomlimit10pt
	
% Macros
	% Notation
		% Operators
			\DeclareMathOperator{\divr}{div}
		% Sets
			\newcommand{\R}{\mathbb{R}}
		% Other
			\DeclareMathOperator{\avg}{avg}
			\DeclareMathOperator{\return}{\text{return}}
			\renewcommand{\th}{\text{th}}
			\DeclareMathOperator{\while}{\text{while}}
	% Utilities
		\newcommand{\callout}[2]{\begin{center}\fbox{\begin{minipage}{#1cm}#2\end{minipage}}\end{center}}
		\newcommand{\comment}[1]{}
		\newcommand{\enumset}[1]{\setcounter{enumi}{#1}}
		\newcommand{\subsectionb}[1]{\subsection*{#1}\addcontentsline{toc}{subsection}{#1}}	

\begin{document}
	\maketitle
	\setcounter{section}{3}
	\section{Number Theory and Cryptography}
		\setcounter{subsection}{1}
		\subsection{Integer Representations and Algorithms}
			\subsubsection*{1--11 odd, 21, 23}
				\begin{enumerate}
					\item
						\begin{tasks}(2)
							\task
								\(231 = (1110\, 0111)_2\)
							\task
								\(4532 = (1\, 0001\, 1011\, 0100)_2\)
							\task
								\(97644 = (1\, 0111\, 1101\, 0110\, 1100)_2\)
						\end{tasks}
					\enumset{2}
					\item
						\begin{tasks}(2)
							\task
								\((1\, 1111)_2 = 37\)
							\task
								\((10\, 0000\, 0001)_2 = 513\)
							\task
								\((1\, 0101\, 0101)_2 = 215\)
							\task
								\((110\, 1001\, 0001\, 0000)_2 = 26896\)
						\end{tasks}
					\enumset{4}
					\item
						\begin{tasks}(2)
							\task
								\((572)_8 = 378\)
							\task
								\((1604)_8 = 900\)
							\task
								\((432)_8 = 275\)
							\task
								\((2417)_8 = 1295\)
						\end{tasks}
					\enumset{6}
					\item
						\begin{tasks}(1)
							\task
								\((80\text{E})_{16} = (1000\, 0000\, 1110)_2\)
							\task
								\((135\text{AB})_{16} = (0001\, 0011\, 0101\, 1010\, 1011)_2\)
							\task
								\((\text{ABBA})_{16} = (1010\, 1011\, 1011\, 1010)_2\)
							\task
								\((\text{DEFACED})_{16} = (1101\, 1110\, 1111\, 1010\, 1100\, 1110\, 1101)_2\)
						\end{tasks}
					\enumset{8}
					\item
						\((\text{ABCDEF})_{16} = (1010\, 1011\, 1100\, 1101\, 1110\, 1111)_2\)
					\enumset{10}
					\item
						\((1011\, 0111\, 1011)_2 = (\text{B7B})_{16}\)
					\enumset{20}
					\item
						\begin{tasks}(2)
							\task
								\begin{tabular}{*{9}{c@{\,}}}
									\(\overset{1}{}\) & 1 & 0 & 0 && \(\overset{1}{0}\) & \(\overset{1}{1}\) & \(\overset{1}{1}\) & 1 \\
									+ & 1 & 1 & 1 && 0 & 1 & 1 & 1 \\\hline
									 1 & 0 & 1 & 1 && 1 & 1 & 1 & 0\\
									\end{tabular} \\\\
								\begin{tabular}{*{18}{c@{\,}}}
									& & & & & & & & & & 1 & 0 & 0 && 0 & 1 & 1 & 1 \\
									& & & & & & & & & \(\times\) & 1 & 1 & 1 && 0 & 1 & 1 & 1 \\\hline
									& \(\overset{1}{}\) & \(\overset{1}{}\) && \(\overset{1}{}\)  & \(\overset{1}{}\) & \(\overset{10}{}\) & \(\overset{11}{}\) && \(\overset{11}{}\) & \(\overset{10}{1}\) & \(\overset{10}{0}\) & \(\overset{10}{0}\) && \(\overset{10}{0}\) & \(\overset{1}{1}\) & 1 & 1 \\
									& & && & & & && 1 & 0 & 0 & 0 && 1 & 1 & 1 \\
									& & && & & & 1 && 0 & 0 & 0 & 1 && 1 & 1 \\
									& & && & 1 & 0 & 0 && 0 & 1 & 1 & 1 \\
									& & && 1 & 0 & 0 & 0 && 1 & 1 & 1 \\
									+ & & 1 && 0 & 0 & 0 & 1 && 1 & 1 \\\hline
									& 1 & 0 && 0 & 0 & 0 & 1 && 0 & 0 & 0 & 0 && 0 & 0 & 0 & 1
									\end{tabular} \\
							\task
								\begin{tabular}{*{11}{c@{\,}}}
									\(\overset{1}{}\) && \(\overset{1}{1}\) & \(\overset{1}{1}\) & \(\overset{1}{1}\) & \(\overset{1}{0}\) && \(\overset{1}{1}\) & \(\overset{1}{1}\) & \(\overset{1}{1}\) & 1 \\
									&+ & 1 & 0 & 1 & 1 && 1 & 1 & 0 & 1 \\\hline
									1 && 1 & 0 & 1 & 0 && 1 & 1 & 0 & 0
								\end{tabular} \\\\
								\begin{tabular}{*{19}{c@{\,}}}
									& & & & & & & & & &  1& 1 & 1 & 0 && 1 & 1 & 1 & 1 \\
									& & & & & & & & & \(\times\) & 1 & 0 & 1 & 1 && 1 & 1 & 0 & 1 \\\hline
									\(\overset{1}{}\) & \(\overset{1}{}\) & \(\overset{10}{}\) & \(\overset{11}{}\) && \(\overset{100}{}\) & \(\overset{100}{}\) & \(\overset{100}{}\) & \(\overset{100}{}\) && \(\overset{11}{1}\) & \(\overset{11}{1}\) & \(\overset{10}{1}\) & \(\overset{10}{0}\) && \(\overset{1}{1}\) & 1 & 1 & 1 \\
									& & & & & & & 1 & 1 && 1 & 0 & 1 & 1 && 1 & 1 \\
									& & & & & & 1 & 1 & 1 && 0 & 1 & 1 & 1 && 1 \\
									& & & & & 1 & 1 & 1 & 0 && 1 & 1 & 1 & 1 \\
									& & & 1 && 1 & 1 & 0 & 1 && 1 & 1 & 1 \\
									+ & 1 & 1 & 1 && 0 & 1 & 1 & 1 && 1 \\\hline
									1 & 0 & 1 & 1 && 0 & 0 & 0 & 0 && 0 & 1 & 1 & 1 && 0 & 0 & 1 & 1
								\end{tabular}
							\task
								
						\end{tasks}
					\enumset{22}
					\item
						\begin{tasks}(4)
							\task
								\begin{tabular}{*{4}{c@{\,}}}
									\(\overset{1}{}\) & \(\overset{1}{7}\) & \(\overset{1}{6}\) & 3 \\
									+ & 1 & 4 & 7 \\\hline
									1 & 1 & 3 & 2
								\end{tabular} \\\\
								\begin{tabular}{*6{c@{\,}}}
									& & & 7 & 6 & 3 \\
									& & \(\times\) & 1 & 4 & 7 \\\hline
									& \(\overset{2}{}\) & \(\overset{1}{6}\) & \(\overset{1}{6}\) & 4 & 5 \\
									\(\overset{1}{}\) & 3 & 7 & 1 & 4  \\
									+ & 7 & 6 & 3 \\\hline
									1 & 4 & 4 & 3 & 0 & 5
								\end{tabular}
							\task
								\begin{tabular}{*{5}{c@{\,}}}
									& 6 & 0 & 0 & 1 \\
									+ & & 2 & 7 & 2 \\\hline
									& 6 & 2 & 7 & 3
								\end{tabular} \\\\
								\begin{tabular}{*{8}{c@{\,}}}
									& & & & 6 & 0 & 0 & 1 \\
									& & & \(\times\) & & 2 & 7 & 2 \\\hline
									& \(\overset{1}{}\) & & 1 & 4 & 0 & 0 & 2 \\
									& & 5 & 2 & 0 & 0 & 7 \\
									+ & 1 & 4 & 0 & 0 & 2 \\\hline
									& 2& 1 & 3 & 4 & 2 & 7 & 2
									
								\end{tabular}
							\task
								\begin{tabular}{*{5}{c@{\,}}}
									& \(\overset{1}{1}\) & \(\overset{1}{1}\) & \(\overset{1}{1}\) & 1 \\
									+ & & 7 & 7 & 7 \\\hline
									& 2 & 1 & 1 & 0
								\end{tabular} \\\\
								\begin{tabular}{*{7}{c@{\,}}}
									& & & \(\overset{1}{1}\) & \(\overset{1}{1}\) & \(\overset{1}{1}\) & 1 \\
									& & \(\times\) & & 7 & 7 & 7 \\\hline
									& & & 7 & 7 & 7 & 7 \\
									& & 7 & 7 & 7 & 7  \\
									+ & 7 & 7 & 7 & 7 \\\hline
									1 & 1 & 0 & 7 & 6 & 6 & 7
								\end{tabular}
							\task
								\begin{tabular}{*{5}{c@{\,}}}
									5 & 4 & 3 & 2 & 1 \\
									+ & 3 & 4 & 5 & 6 \\\hline
									5 & 7 & 7 & 7 & 7
								\end{tabular} \\\\
								\begin{tabular}{*{10}{c@{\,}}}
									& & & & & 5 & 4 & 3 & 2 & 1 \\
									& & & & & \(\times\) & 3 & 4 & 5 & 6 \\\hline
									& & \(\overset{1}{}\) & \(\overset{1}{}\) & \(\overset{2}{4}\) & 1 & \(\overset{1}{2}\) & \(\overset{1}{3}\) & 4 & 6 \\
									& & & 3 & 3 & 6 & 0 & 2 & 5 \\
									& & 2 & 6 & 1 & 5 & 0 & 4 \\
									+ & 2 & 0 & 5 & 1 & 6 & 3 \\\hline
									& 2 & 3 & 7 & 3 & 2 & 6 & 2 & 1 & 6
								\end{tabular}
						\end{tasks}
				\end{enumerate}
		\subsection{Primes and Greatest Common Divisors}
			\subsubsection*{1, 3, 5, 15, 17 (19 extra credit)}
				\begin{enumerate}
					\item
						\begin{tasks}(2)
							\task
								\(21 = 7 \times 3 \therefore \) composite
							\task
								\(\sqrt{29} \approx 5.385\)
								\begin{itemize}[leftmargin = -0.25cm]
									\item Odd \(\therefore \not{\mid}\,\,\, 2\) 
									\item \(29 = 10(3) - 1 \therefore \not{\mid}\,\,\, 3\)
									\item \(29 = 6(5) - 1 \therefore \not{\mid}\,\,\, 5 \therefore \text{prime}\)
								\end{itemize}
							\task
								\(\sqrt{71} \approx 8.426\)
									\begin{itemize}[leftmargin = -0.25cm]
										\item Odd \(\therefore \not{\mid}\,\,\, 2\)
										\item \(7 + 1 = 8 = 3(3) - 1 \therefore \not{\mid}\,\,\, 3\)
										\item \(71 = 5(14) + 1 \therefore \not{\mid}\,\,\, 5\)
										\item  \(71 = 7(10) + 1 \therefore \not{\mid}\,\,\, 7 \therefore\) prime
									\end{itemize}
							\task
								\(\sqrt{97} \approx 9.849\)
									\begin{itemize}[leftmargin = -0.25cm]
										\item Odd \(\therefore \not{\mid}\,\,\, 2\)
										\item \(97 = 3(32) + 1 \therefore \not{\mid}\,\,\, 3\)
										\item \(97 = 5(19) + 2 \therefore \not{\mid}\,\,\, 5\)
										\item \(97 = 7(14) - 1 \therefore \not{\mid}\,\,\, 7 \therefore\) prime	
									\end{itemize}
						\end{tasks}
					\setcounter{enumi}{2}
					\item
						\begin{tasks}(4)
							\task
								\(88 = 2^3 \times 11\)
							\task
								\(126 = 2 \times 3^2 \times 7\)
							\task
								\(729 = 3^6\)
							\task
								\(1001 = 7 \times 11 \times 13\)
							\task
								\(1,111 = 11 \times 101\)
							\task
								\(909,090 = 2 \times 3^3 \times 5 \times 13 \times 259\)
						\end{tasks}
					\setcounter{enumi}{4}
					\item
						\(10! = 2 \times 3 \times 4 \times 5 \times 6 \times 7 \times 8 \times 9 \times 10 = 2^8 \times 3^4 \times 5^2 \times 7\)
					\setcounter{enumi}{14}
					\item
						\(30 = 2 \times 3 \times 5 \implies 1, 7, 11, 13, 17, 19, 23, 29\)
					\setcounter{enumi}{16}
					\item
						\begin{tasks}(2)
							\task
								\(11, 15 = 3 \times 5, 19 \therefore\) Yes
							\task
								\(14 = 7 \times 2, 15 = \bm{3} \times 5, 21 = \bm{3} \times 21 \therefore\) No
							\task
								\(12 = 2^2 \times 3, 17, 31, 37 \therefore\) Yes
							\task
								\(7, 8 = 2^3, 9 = 3^2, 11 \therefore\) Yes
						\end{tasks}
				\end{enumerate}
	\setcounter{section}{5}
	\section{Counting}
		\subsection{The Basics of Counting}
			\begin{enumerate}
				\setcounter{enumi}{2}
				\item	
			\end{enumerate}
		\setcounter{subsection}{2}
		\subsection{Permutations and Combinations}
			\begin{enumerate}
				\item
					\(\{a, b, c\}, \{a, c, b\}, \{b, a, c\}, \{b, c, a\}, \{c, a, b\}, \{c, b, a\}\)
				\setcounter{enumi}{2}
				\item
					\(P(6, 6) = \dfrac{6!}{(6 - 6)!} = 720\)
				\setcounter{enumi}{4}
				\item
					\begin{tasks}(2)
						\task
							\(P(6, 3) = \dfrac{6!}{(6 - 3)!} = 120\)
						\task
							\(P(6, 5) = \dfrac{6!}{(6 - 5)!} = 720\)
						\task
							\(P(8, 1) = \dfrac{8!}{(8 - 1)!} = 8\)
						\task
							\(P(8, 5) = \dfrac{8!}{(8 - 5)!} = 336\)
						\task
							\(P(8, 8) = \dfrac{8!}{(8 - 8)!} = 40,320\)
						\task
							\(P(10, 9) = \dfrac{10!}{(10 - 9)!} = 3,628,880\) 
					\end{tasks}
				\setcounter{enumi}{6}
				\item
					\(P(9, 5) = \dfrac{9!}{(9 - 5)!} = 15,120\)
				\setcounter{enumi}{8}
				\item
					\(P(12, 3 = \dfrac{12!}{(12 - 3)!} = 1,320\)
				\setcounter{enumi}{10}
				\item
					\begin{tasks}(2)
						\task
							\(C(10, 4) = \dfrac{10!}{4!(10 - 4)!} = 210\)
						\task
							\(\displaystyle \sum_{i = 0}^4 C(10, i) = \sum_{i = 0}^4 \frac{10!}{i!(10 - i)!} = 386\)
						\task
							\(\displaystyle \sum_{i = 4}^{10} C(10, i) = \sum_{i = 4}^{10} \frac{10!}{i!(10 - i)!} = 848\)
						\task
							\(C(10, 5) = \dfrac{10!}{5!(10 - 5)!} = 252\)
					\end{tasks}
				\setcounter{enumi}{20}
				\item
					\begin{tasks}(3)
						\task
							\(P(5, 5) = \dfrac{5!}{(5 - 5)!} = 120\)	
						\task
							\(P(4, 4) = \dfrac{4!}{(4 - 4)!} = 24\)
						\task
							\(P(5, 5) = \dfrac{5!}{(5 - 5)!} = 120\)
						\task
							\(P(4, 4) = \dfrac{4!}{(4 - 4)!} = 24\)
						\task
							\(P(3, 3) = \dfrac{3!}{(3 - 3)!} = 6\)
						\task
							0, as repetitions are not allowed
					\end{tasks}
				\setcounter{enumi}{28}
				\item
					\begin{tasks}(2)
						\task
							\(C(25, 4) = \dfrac{25!}{4!(25 - 4)!} = 12,650\)
						\task
							\(P(25, 4) = \dfrac{25!}{(25 - 4)!} = 303,600\)
					\end{tasks}
				\setcounter{enumi}{36}
				\item
					\(C(10, 2) = \frac{10!}{2!(10 - 2)!} = 45\)
				\setcounter{enumi}{38}
				\item
					\(\displaystyle \sum_{i = 3}^{7}C(10, i) = \sum_{i = 3}^7 \frac{10!}{i!(10 - i)!} = 912\)
			\end{enumerate}
		\setcounter{subsection}{3}
		\subsection{Binomial Coefficients and Identities}
			\begin{enumerate}
				\item
			\end{enumerate}
		\subsection{Generalized Permutations and Combinations}
			\begin{enumerate}
				\setcounter{enumi}{4}
				\item
					\(C(5 + 3 - 1, 3) = \dfrac{(7)!}{3!(4)!} = 35\)
				\setcounter{enumi}{8}
				\item
					\begin{tasks}(2)
						\task
							\(C(8 + 6 - 1, 6) = \dfrac{13!}{6!(7)!} = 1,716\)	
						\task
							\(C(8 + 12 - 1, 12) = \dfrac{19!}{12!(7)!} = 50,388\)
						\task
							\(C(8 + 24 - 1, 24) = \dfrac{31!}{24!(7)!} = 2,629,575\)
						\task
							\(C(8 + 4 - 1, 4) = \dfrac{11!}{4!(7)!} = 330\)
					\end{tasks}
					\begin{itemize}[leftmargin = 1.04cm]
						\item[e)]
							\(\displaystyle \sum_{i = 0}^{2} C(7 + 9 - i - 1, 9 - i) = \sum_{i = 0}^2 \frac{(15 - i)!}{(9 - i)!(6)!} = 9,724\)	
					\end{itemize}
				\setcounter{enumi}{10}
				\item
					\(C(2 + 8 - 1, 8) = \dfrac{9!}{8!(1)!} = 9\)
				\setcounter{enumi}{32}
				\item
					\(\dfrac{11!}{5!2!2!1!1!} = 83,160\)
				\setcounter{enumi}{34}
				\item
					\(P(3, 1) + [1 + P(3, 2)] + \left[1 + 2\left(\dfrac{3!}{2!1!}\right) + P(3, 3)\right] + \left[2\left(\dfrac{4!}{3!1!}\right) + \dfrac{4!}{2!1!1!}\right] + \left[\dfrac{5!}{3!1!1!}\right] = 63\)
			\end{enumerate}
	\setcounter{section}{4}
	\section{Induction and Recursion}
		\subsection{Mathematical Induction}
			\begin{enumerate}
				\setcounter{enumi}{4}
				\item
					Let
						\[P(n) \implies \sum_{i = 0}^{n}(2i + 1)^2 = \frac{(n + 1)(2n + 1)(2n + 3)}{3}\]
						Let \(n = 0\):
						\begin{align*}
							\sum_{i = 0}^0(2i + 1)^2 &= \frac{(0 + 1)(0 + 1)(0 + 3)}{3} \\
							(0 + 1)^2 &= \frac{3}{3} \\
							1 &= 1 \implies P(0)
						\end{align*}
						Assume that \(P(k)\) is true for an arbitrary fixed integer \(k > 0\):
						\begin{align*}
							P(k) \implies \sum_{i = 0}^k(2i + 1)^2 &= \frac{(k + 1)(2k + 1)(2k + 3)}{3} \\
							\sum_{i = 0}^{k + 1}(2i + 1)^2 &= \frac{(k + 1)(2k + 1)(2(k + 3)}{3} + (2k + 3)^2 \\
								&= (2k + 3)\left(\frac{(k + 1)(2k + 1)}{3} + 2k + 3\right) \\
								&= (2k + 3)\left(\frac{2k^2 + k + 2k + 1 + 6k + 9}{3}\right) \\
								&= \frac{(2k + 3)(2k^2 + 9k + 10)}{3}
										=  \frac{2k + 3)(2k + 5)(k + 2)}{3} \\
								&= \frac{((k + 1) + 2)(2(k + 1) + 1)(2(k + 1) + 3)}{3}
										\implies P(k + 1)
						\end{align*}
						By mathematical induction, \(P(n)\) is true for all integers \(n \ge 0\).
				\setcounter{enumi}{6}
				\item
					Let
						\[P(n) \implies \sum_{i = 0}^n\left[3 \times 5^i\right] = \frac{3\left(5^{n + 1} - 1\right)}{4}\]
						Let \(n = 0\):
						\begin{align*}
							\sum_{i = 0}^0\left[3 \times 5^i\right] = \frac{3\left(5^{0 + 1} - 1\right)}{4} \\
							3 \times 5^0 &= \frac{3(4)}{4} \\
							3 &= 3 \implies P(0)
						\end{align*}
						Assume that \(P(k)\) is true for an arbitrary fixed integer \(k > 0\):
						\begin{align*}
							P(k) \implies \sum_{i = 0}^k\left[3 \times 5^i\right] &= \frac{3\left(5^{k + 1} - 1\right)}{4} \\
							\sum_{i = 0}^{k + 1}\left[3 \times 5^i\right] &= \frac{3\left(5^{k + 1} - 1\right)}{4} + \left(3 \times 5^{k + 1}\right)
									= \frac{3\left(5^{k + 1}\left(1 + 4\right) - 1\right)}{4} \\
								&= \frac{3\left(5^{k + 2} - 1\right)}{4}
										= \frac{3\left(5^{(k + 1) + 1} - 1\right)}{4} 
										\implies P(k + 1)
						\end{align*}
						By mathematical induction, \(P(n)\) is true for all integers \(n \ge 0\).
				\setcounter{enumi}{8}
				\item
					\begin{tasks}
						\task
							\[\sum_{i = 1}^n 2i = 2 \times \frac{n(n + 1)}{2} = n(n + 1)\]
						\task
							Let
								\[P(n) \implies \sum_{i = 1}^n 2i = n(n + 1)\]
								Let \(n = 1\):
								\begin{align*}
									\sum_{i = 1}^1 2i &= 1(1 + 1) \\
									2 &= 2 = 2
								\end{align*}
								Assume that \(P(k)\) is true for an arbitrary fixed integer \(k > 1\):
								\begin{align*}
									P(k) \implies \sum_{i = 1}^k 2i &= k(k + 1) \\
									\sum_{i = 1}^{k + 1} 2i &= k(k + 1) + 2(k + 1) 
											= k^2 + k + 2k + 1
											= k^2 + 3k + 1 \\
										&= (k + 1)(k + 2) \implies P(k + 1)
								\end{align*}
								By mathematical induction, \(P(n)\) is true for all integers \(n \ge 1\).
					\end{tasks}
				\setcounter{enumi}{10}
				\item
					\begin{tasks}
						\task
							\[\def\arraystretch{2}\begin{array}{|c|ccc|}\hline
								n & 1 & 2 & 3 \\\hline
								\displaystyle \sum_{i = 1}^n \frac{1}{2^i} & \frac{1}{2} & \frac{3}{4} & \frac{7}{8} \\\hline
							\end{array}\]
							\[\sum_{i = 1}^n \frac{1}{2^i} = 1 - \frac{1}{2^n}\]
						\task
							Let
								\[P(n) \implies \sum_{i = 1}^n \frac{1}{2^i} = 1 - \frac{1}{2^n}\]
								Let \(n = 1\):
								\begin{align*}
									\sum_{i = 1}^1 \frac{1}{2^i} &= 1 - \frac{1}{2^1} \\
									\frac{1}{2} &= \frac{1}{2}
											\implies P(1)
								\end{align*}
								Assume that \(P(k)\) is true for an arbitrary fixed integer \(k > 1\):
								\begin{align*}
									P(k) \implies \sum_{i = 1}^k \frac{1}{2^i} &= 1 - \frac{1}{2^k} \\
									\sum_{i = 1}^{k + 1} \frac{1}{2^i} &= 1 - \frac{1}{2^k} + \frac{1}{2^{k + 1}}
											= 1 + \frac{1 - 2}{2^{k + 1}}
											= 1 - \frac{1}{2^{k + 1}}
											\implies P(k + 1)
								\end{align*}
								By mathematical induction, \(P(n)\) is true for all integers \(n \ge 1\).
					\end{tasks}
				\setcounter{enumi}{12}
				\item
					Let
						\[P(n) \implies \sum_{i = 1}^n (-1)^{i - 1}i^2 = \frac{(-1)^{n - 1}n(n + 1)}{2}\]
						Let \(n = 1\):
						\begin{align*}
							\sum_{i = 1}^1 (-1)^{i - 1}i^2 &= \frac{(-1)^{1 - 1}1(1 + 1)}{2} \\
							1 &= 1 \implies P(1)
						\end{align*}
						Assume that \(P(k)\) is true for an arbitrary integer \(k > 1\):
						\begin{align*}
							P(k) \implies \sum_{i = 1}^k (-1)^{i - 1}i^2 &= \frac{(-1)^{k - 1}k(k + 1)}{2} \\
							\sum_{i = 1}^{k + 1} (-1)^{i - 1}i^2 &= \frac{(-1)^{k - 1}k(k + 1)}{2} + (-1)^{k + 1 - 1}(k + 1)^2 \\
								&= \frac{(-1)(-1)^{k}(k^2 + k) + 2(-1)^k(k^2 + 2k + 1)}{2} \\
								&= \frac{(-1)^k(2k^2 + 4k + 2 - k^2 - k)}{2}
										= \frac{(-1)^k(k^2 + 3k + 2)}{2} \\
								&= \frac{(-1)^{k}(k + 1)(k + 1)}{2}
										= \frac{(-1)^k(k + 1)((k + 1) + 1)}{2}
										\implies P(k + 1)
						\end{align*}
						By mathematical induction, \(P(n)\) is true for all integers \(n \ge 1\).
				\setcounter{enumi}{14}
				\item
					Let
						\[P(n) \implies \sum_{i = 1}^n i(i + 1) = \frac{n(n + 1)(n + 2)}{3}\]
						Let \(n = 1\)
						\begin{align*}
							\sum_{i = 1}^1 i(i + 1) = \frac{1(1 + 1)(1 + 2)}{3} \\
							1(2) &= \frac{1(2)(3)}{3} \\
							2 &= 2 \implies P(1)
						\end{align*}
						Assume that \(P(k)\) is true for an arbitrary fixed integer \(k > 1\):
						\begin{align*}
							P(k) \implies \sum_{i = 1}^k i(i + 1) &= \frac{k(k + 1)(k + 2)}{3} \\
							\sum_{i = 1}^{k + 1} i(i + 1) &= \frac{k(k + 1)(k + 2)}{3} + (k + 1)(k + 2)
									= \frac{(k + 3)(k + 1)(k + 2)}{3} \\
								&= \frac{(k + 1)((k + 1) + 1)((k + 1) + 2)}{3} \implies P(k + 1)
						\end{align*}
						By mathematical induction, \(P(n)\) is true for all integers \(n \ge 1\).
				\setcounter{enumi}{16}
				\item
					Let
						\[P(n) \implies \sum_{j = 1}^n j^4 = \frac{n(n + 1)(2n + 1)(3n^2 + 3n - 1)}{30}\]
						Let \(n = 1\):
						\begin{align*}
							\sum_{j = 1}^1 j^4 &= \frac{1(1 + 1)(2 + 1)(3 + 3 - 1)}{30} \\
							1 &= \frac{1(2)(3)(5)}{30}
									= \frac{30}{30} 
									= 1
									\implies P(1)
						\end{align*}
						Assume that \(P(k)\) is true for an arbitrary integer \(k > 1\):
						\begin{align*}
							P(k) \implies \sum_{i = 1}^k j^4 &= \frac{k(k + 1)(2k + 1)(3k^2 + 3k - 1)}{30} \\
							\sum_{j = 1}^{k + 1} j^4 &= \frac{k(k + 1)(2k + 1)(3k^2 + 3k - 1)}{30} + (k + 1)^4 \\
								&= \frac{(2k^3 + 3k^2 + k)(3k^2 + 3k - 1)}{30} + k^4 + 4k^3 + 6k^2 + 4k + 1 \\
								&= \frac{6k^5 + 6k^4 - 2k^3 + 9k^4 + 9k^3 - 3k^2 + 3k^3 + 3k^2 - k}{30} \\
								&\qquad + k^4 + 4k^3 + 6k^2 + 4k + 1 \\
								&= \frac{6k^5 + 45k^4 + 130k^3 + 180^2 + 119k + 30}{30} \\
								&= \frac{(k + 1)((k + 1) + 1)(2(k + 1) + 1)(3(k + 1)^2 + 3(k + 1) - 1)}{30} \\
								&\implies P(k + 1)
						\end{align*}
						By mathematical induction, \(P(n)\) is true for al integers \(n \ge 1\).
			\end{enumerate}
\end{document}
