\documentclass[12pt, A4]{article}

% Packages
	% Basics
		\usepackage{amsmath}
		\usepackage{bm}
		\usepackage{cellspace}
		\usepackage{csquotes}
		\usepackage[shortlabels]{enumitem}
		\usepackage[hang,flushmargin]{footmisc}
		\usepackage[margin=0.75in]{geometry}
		\usepackage{hyperref}
		\usepackage[utf8]{inputenc}
	% Diagrams
		\usepackage{pgfplots}
		\usepackage{tikz}
		\usepackage{tikz-3dplot}
			\usetikzlibrary{arrows.meta, angles, calc, quotes}
	% Formatting
		\usepackage{booktabs}
		\usepackage{tasks}
	% Symbols
		\usepackage{amssymb} % Miscellaneous
		\usepackage{esint} % Integrals
		\usepackage{physics} % Differentials
% Configuration
	\title{Homework Set 1}
	\author{Arnav Patri}
	\hypersetup{
	    colorlinks,
	    citecolor=black,
	    filecolor=black,
	    linkcolor=black,
	    urlcolor=black
	}
	\cellspacetoplimit10pt
	\cellspacebottomlimit10pt
	
% Macros
	% Notation
		% Operators
			\DeclareMathOperator{\divr}{div}
		% Sets
			\newcommand{\R}{\mathbb{R}}
		% Other
			\DeclareMathOperator{\avg}{avg}
			\renewcommand{\mod}{\text{mod}}
			\DeclareMathOperator{\return}{\text{return}}
			\renewcommand{\th}{\text{th}}
			\DeclareMathOperator{\while}{\text{while}}
	% Utilities
		\newcommand{\callout}[2]{\begin{center}\fbox{\begin{minipage}{#1cm}#2\end{minipage}}\end{center}}
		\newcommand{\comment}[1]{}
		\newcommand{\enumset}[1]{\setcounter{enumi}{#1}}
		\newcommand{\subsectionb}[1]{\subsection*{#1}\addcontentsline{toc}{subsection}{#1}}	

\begin{document}
	\maketitle
	\setcounter{section}{3}
	\section{Number Theory and Cryptography}
		\setcounter{subsection}{1}
		\subsection{Integer Representations and Algorithms}
			\subsubsection*{1--11 odd, 21, 23}
				\begin{enumerate}
					\item
						\begin{tasks}(2)
							\task
								\(231 = (1110\, 0111)_2\)
							\task
								\(4532 = (1\, 0001\, 1011\, 0100)_2\)
							\task
								\(97644 = (1\, 0111\, 1101\, 0110\, 1100)_2\)
						\end{tasks}
					\enumset{2}
					\item
						\begin{tasks}(2)
							\task
								\((1\, 1111)_2 = 37\)
							\task
								\((10\, 0000\, 0001)_2 = 513\)
							\task
								\((1\, 0101\, 0101)_2 = 215\)
							\task
								\((110\, 1001\, 0001\, 0000)_2 = 26896\)
						\end{tasks}
					\enumset{4}
					\item
						\begin{tasks}(2)
							\task
								\((572)_8 = 378\)
							\task
								\((1604)_8 = 900\)
							\task
								\((432)_8 = 275\)
							\task
								\((2417)_8 = 1295\)
						\end{tasks}
					\enumset{6}
					\item
						\begin{tasks}(1)
							\task
								\((80\text{E})_{16} = (1000\, 0000\, 1110)_2\)
							\task
								\((135\text{AB})_{16} = (0001\, 0011\, 0101\, 1010\, 1011)_2\)
							\task
								\((\text{ABBA})_{16} = (1010\, 1011\, 1011\, 1010)_2\)
							\task
								\((\text{DEFACED})_{16} = (1101\, 1110\, 1111\, 1010\, 1100\, 1110\, 1101)_2\)
						\end{tasks}
					\enumset{8}
					\item
						\((\text{ABCDEF})_{16} = (1010\, 1011\, 1100\, 1101\, 1110\, 1111)_2\)
					\enumset{10}
					\item
						\((1011\, 0111\, 1011)_2 = (\text{B7B})_{16}\)
					\enumset{20}
					\item
						\begin{tasks}(2)
							\task
								\begin{tabular}{*{9}{c@{\,}}}
									\(\overset{1}{}\) & 1 & 0 & 0 && \(\overset{1}{0}\) & \(\overset{1}{1}\) & \(\overset{1}{1}\) & 1 \\
									+ & 1 & 1 & 1 && 0 & 1 & 1 & 1 \\\hline
									 1 & 0 & 1 & 1 && 1 & 1 & 1 & 0\\
									\end{tabular} \\\\
								\begin{tabular}{*{18}{c@{\,}}}
									& & & & & & & & & & 1 & 0 & 0 && 0 & 1 & 1 & 1 \\
									& & & & & & & & & \(\times\) & 1 & 1 & 1 && 0 & 1 & 1 & 1 \\\hline
									& \(\overset{1}{}\) & \(\overset{1}{}\) && \(\overset{1}{}\)  & \(\overset{1}{}\) & \(\overset{10}{}\) & \(\overset{11}{}\) && \(\overset{11}{}\) & \(\overset{10}{1}\) & \(\overset{11}{0}\) & \(\overset{10}{0}\) && \(\overset{10}{0}\) & \(\overset{1}{1}\) & 1 & 1 \\
									& & && & & & && 1 & 0 & 0 & 0 && 1 & 1 & 1 \\
									& & && & & & 1 && 0 & 0 & 0 & 1 && 1 & 1 \\
									& & && & 1 & 0 & 0 && 0 & 1 & 1 & 1 \\
									& & && 1 & 0 & 0 & 0 && 1 & 1 & 1 \\
									+ & & 1 && 0 & 0 & 0 & 1 && 1 & 1 \\\hline
									& 1 & 0 && 0 & 0 & 0 & 1 && 0 & 0 & 0 & 0 && 0 & 0 & 0 & 1
									\end{tabular} \\
							\task
								\begin{tabular}{*{10}{c@{\,}}}
									\(\overset{1}{}\) & \(\overset{1}{1}\) & \(\overset{1}{1}\) & 1 & 0 && \(\overset{1}{1}\) & 1 & \(\overset{1}{1}\) & 1 \\
									+ & 1 & 0 & 1 & 1 && 1 & 1 & 0 & 1 \\\hline
									1 & 1 & 0 & 0 & 1 && 1 & 0 & 1 & 0
								\end{tabular} \\\\
								\begin{tabular}{*{10}{c@{\,}}}
									\(\overset{1}{}\) & \(\overset{1}{1}\) & \(\overset{1}{1}\) & 1 & 0 && \(\overset{1}{1}\) & 1 & \(\overset{1}{1}\) & 1 \\
									+ & 1 & 0 & 1 & 1 && 1 & 1 & 0 & 1 \\\hline
									1 & 1 & 0 & 0 & 1 && 1 & 0 & 1 & 0
								\end{tabular}
							\task
								
						\end{tasks}
					\enumset{22}
					\item
						\begin{tasks}(4)
							\task
								\begin{tabular}{*{4}{c@{\,}}}
									\(\overset{1}{}\) & \(\overset{1}{7}\) & \(\overset{1}{6}\) & 3 \\
									+ & 1 & 4 & 7 \\\hline
									1 & 1 & 3 & 2
								\end{tabular} \\\\
								\begin{tabular}{*6{c@{\,}}}
									& & & 7 & 6 & 3 \\
									& & \(\times\) & 1 & 4 & 7 \\\hline
									& \(\overset{2}{}\) & \(\overset{1}{6}\) & \(\overset{1}{6}\) & 4 & 5 \\
									\(\overset{1}{}\) & 3 & 7 & 1 & 4  \\
									+ & 7 & 6 & 3 \\\hline
									1 & 4 & 4 & 3 & 0 & 5
								\end{tabular}
							\task
								\begin{tabular}{*{5}{c@{\,}}}
									& 6 & 0 & 0 & 1 \\
									+ & & 2 & 7 & 2 \\\hline
									& 6 & 2 & 7 & 3
								\end{tabular} \\\\
								\begin{tabular}{*{8}{c@{\,}}}
									& & & & 6 & 0 & 0 & 1 \\
									& & & \(\times\) & & 2 & 7 & 2 \\\hline
									& \(\overset{1}{}\) & & 1 & 4 & 0 & 0 & 2 \\
									& & 5 & 2 & 0 & 0 & 7 \\
									+ & 1 & 4 & 0 & 0 & 2 \\\hline
									& 2& 1 & 3 & 4 & 2 & 7 & 2
									
								\end{tabular}
							\task
								\begin{tabular}{*{5}{c@{\,}}}
									& \(\overset{1}{1}\) & \(\overset{1}{1}\) & \(\overset{1}{1}\) & 1 \\
									+ & & 7 & 7 & 7 \\\hline
									& 2 & 1 & 1 & 0
								\end{tabular} \\\\
								\begin{tabular}{*{7}{c@{\,}}}
									& & & \(\overset{1}{1}\) & \(\overset{1}{1}\) & \(\overset{1}{1}\) & 1 \\
									& & \(\times\) & & 7 & 7 & 7 \\\hline
									& & & 7 & 7 & 7 & 7 \\
									& & 7 & 7 & 7 & 7  \\
									+ & 7 & 7 & 7 & 7 \\\hline
									1 & 1 & 0 & 7 & 6 & 6 & 7
								\end{tabular}
							\task
								\begin{tabular}{*{5}{c@{\,}}}
									5 & 4 & 3 & 2 & 1 \\
									+ & 3 & 4 & 5 & 6 \\\hline
									5 & 7 & 7 & 7 & 7
								\end{tabular} \\\\
								\begin{tabular}{*{10}{c@{\,}}}
									& & & & & 5 & 4 & 3 & 2 & 1 \\
									& & & & & \(\times\) & 3 & 4 & 5 & 6 \\\hline
									& & \(\overset{1}{}\) & \(\overset{1}{}\) & \(\overset{2}{4}\) & 1 & \(\overset{1}{2}\) & \(\overset{1}{3}\) & 4 & 6 \\
									& & & 3 & 3 & 6 & 0 & 2 & 5 \\
									& & 2 & 6 & 1 & 5 & 0 & 4 \\
									+ & 2 & 0 & 5 & 1 & 6 & 3 \\\hline
									& 2 & 3 & 7 & 3 & 2 & 6 & 2 & 1 & 6
								\end{tabular}
						\end{tasks}
				\end{enumerate}
		\subsection{Primes and Greatest Common Divisors}
			\subsubsection*{1, 3, 5, 15, 17, (19 extra credit)}
\end{document}
