\documentclass[12pt]{article}

% Packages
	% Basics
		\usepackage{amsmath}
    	\usepackage{bm}
		\usepackage[shortlabels]{enumitem}
		\usepackage[margin = 0.75 in]{geometry}
	% Notation
		\usepackage{amssymb}
		\usepackage{mathrsfs}
		\usepackage{physics}
    	\usepackage{siunitx}
    % Formatting
    	\usepackage{tasks}
% Macros
	\newcommand{\en}{\text{e}}
	\newcommand{\Exp}{\mathbb{E}}
  	\newcommand{\subt}[2]{#1_{\text{#2}}}
  	\newcommand{\Z}{\mathbb{Z}}
% Configuration
	\title{Homework Set 3}
	\author{Arnav Patri}

\begin{document}
	\maketitle
	\setcounter{section}{6}
	\section{Discrete Probability}
		\begin{enumerate}
			\item	
				Boys and girls being equally likely implies that the probability of each is \(1/2 = 0.5\) (assuming them to be the only 2 possible outcomes). The number of trials (the number of children) and the probability of each child being a boy are then fixed at 3 and 0.5 respectively, meaning that this situation follows a binomial setting with probability of success \(p = 0.5\) and number of trials \(n = 3\). As such,
				\begin{align*}
					\mu_X &= np 
							= 0.5 \times 3 
							= 1.5 \\
					\sigma_X &= \sqrt{np(1 - p)} 
							= \sqrt{0.5(3)(1 - 0.5)} 
							= \sqrt{0.75} 
							= 0.5\sqrt{3} 
							\approx 0.866
				\end{align*}
				The mean is 1.5 boys while the standard deviation is about 0.866 boys.
			\item
				\(X\) is a discrete random variable that follows the given probability distribution
				\[\begin{array}{|c|*{5}{c}|}\hline
					x & 0 & 1 & 2 & 3 & 4 \\\hline
					P(x) & 0.3 & 0.4 & 0.2 & 0.06 & 0.04 \\\hline
				\end{array}\]
				As such,
				\begin{align*}
					\mu_X &= \sum x_iP(x_i)
							= 1.14 \\
					\sigma_X &= \sqrt{\sum(x_i - \mu_X)^2P(x_i)}
							\approx 1.039
				\end{align*}
				The mean is 1.14 toppings while the standard deviation is about 1.039 toppings.
			\item
				One ticked being randomly selected from \(1,000\) means that the probability of any one ticket being selected is \(1/1,000 = 0.001\). This describes a discrete random variable \(X\) with probability distribution
					\[\begin{array}{|c|*{2}{c}|}\hline
						x & 0 & 387 \\\hline
						P(x) & 1 - 0.001 = 0.999 & 0.001 \\\hline
					\end{array}\]
					The expected value of this random variable is then
					\[
						\Exp[X] = \sum x_iP(x_i)
							= 0.387
					\]
					\(X\) represents the money gained from the drawing rather than the profit, though. The amount of money spent on the ticket is a constant 2, so 2 can simply be subtracted from this result to yield the expected profit of entering the drawing, which is \(-\$1.613\)
			\item
				The number of defective computers is a discrete random variable \(X\) with probability distribution
					\[\begin{array}{|c|*{5}{c}|}\hline
						x & 0 & 1 & 2 & 3 & 4 \\\hline
						P(x) & 0.4305 & 0.4039 & 0.1421 & 0.0222 & 0.0013 \\\hline
					\end{array}\]
					As such,
					\[
						\mu_X = \sum x_iP(x_i)
							= 0.7599
					\]
					The mean number of defective computers in a batch of 4 is 0.7599.
			\item
				The number of defective computers is a discrete random variable \(X\) with probability distribution
					\[\begin{array}{|c|*{5}{c}|}\hline
						x & 0 & 1 & 2 & 3 & 4 \\\hline
						P(x) & 0.5220 & 0.3685 & 0.0975 & 0.0115 & 0.0005\\\hline
					\end{array}\]
					As such,
					\[
						\sigma_X = \sqrt{\sum(x_i - \mu_X)^2P(x_i)}
							\approx 0.714
					\]
			\item
				The number of trials (days) and the probability of success (rain) are fixed at 3 and 0.4 respectively. This situation therefore follows a binomial setting with probability of success \(p = 0.4\) and number of trials \(n = 3\). Let \(X\) be a random variable following this binomial distribution. The only possible values of \(X\) are 0, 1, 2, and 3. As such, the complement of \(X = 0\) is equivalent to \(X \ge 1\). Therefore,
					\[
						P(X \ge 1) = 1 - P(X = 0)
							= 1 - \binom{3}{0}0.4^0(1 - 0.4)^{3 - 0}
							= 1 - 0.216
							= 0.784 
					\]
					The probability of it raining on at least 1 of the 3 days is 0.784.
			\item
				The probability of a given student owning a credit card given that they are a freshman is the relative frequency of owning one among freshman. This is simply the proportion of freshman that carry credit cards:
					\begin{align*}
						P(\text{carries a credit card} \,|\, \text{freshman}) &= \frac{\text{\# of freshman credit card carriers}}{\text{\# of freshman}} \\
							&= \frac{40}{60}
								= \frac{2}{3}
								\approx 0.667
					\end{align*}
					The probability of a randomly selected freshman carrying a cred card is about 0.667.
			\item
				The probability of a given student being a freshman given that they carry a credit card is simply the relative frequency of being a freshman among those that own one. This is the proportion of credit card carriers that are freshman:
					\begin{align*}
						P(\text{freshman} \,|\, \text{carries a credit card}) &= \frac{\text{\# of freshman credit card carriers}}{\text{\# of credit card carriers}} \\
							&= \frac{49}{61}
								\approx 0.803
					\end{align*}
					The probability of a randomly selected credit card owner being a freshman is about 0.803.
			\item
				
		\end{enumerate}
\end{document}

