\documentclass[12pt]{article}

% Packages
	% Basics
		\usepackage{amsmath}
    	\usepackage{bm}
		\usepackage[shortlabels]{enumitem}
		\usepackage[margin = 0.75 in]{geometry}
	% Notation
		\usepackage{amssymb}
		\usepackage{mathrsfs}
		\usepackage{physics}
    	\usepackage{siunitx}
    % Formatting
    	\usepackage{tasks}
% Macros
	\newcommand{\en}{\text{e}}
	\newcommand{\enumset}[1]{\setcounter{enumi}{#1}}
	\newcommand{\Exp}{\mathbb{E}}
  	\newcommand{\subt}[2]{#1_{\text{#2}}}
  	\newcommand{\Z}{\mathbb{Z}}
% Configuration
	\title{Homework Set 3}
	\author{Arnav Patri}

\begin{document}
	\maketitle
	\setcounter{section}{6}
	\section{Discrete Probability}
		\begin{enumerate}
			\item	
				Boys and girls being equally likely implies that the probability of each is \(1/2 = 0.5\) (assuming them to be the only 2 possible outcomes). The number of trials (the number of children) and the probability of each child being a boy are then fixed at 3 and 0.5 respectively, meaning that this situation follows a binomial setting with probability of success \(p = 0.5\) and number of trials \(n = 3\). As such,
				\begin{align*}
					\mu_X &= np 
							= 0.5 \times 3 
							= 1.5 \\
					\sigma_X &= \sqrt{np(1 - p)} 
							= \sqrt{0.5(3)(1 - 0.5)} 
							= \sqrt{0.75} 
							= 0.5\sqrt{3} 
							\approx 0.866
				\end{align*}
				The mean is 1.5 boys while the standard deviation is about 0.866 boys.
			\item
				\(X\) is a discrete random variable that follows the given probability distribution
				\[\begin{array}{|c|*{5}{c}|}\hline
					x & 0 & 1 & 2 & 3 & 4 \\\hline
					P(x) & 0.3 & 0.4 & 0.2 & 0.06 & 0.04 \\\hline
				\end{array}\]
				As such,
				\begin{align*}
					\mu_X &= \sum x_iP(x_i)
							= 1.14 \\
					\sigma_X &= \sqrt{\sum(x_i - \mu_X)^2P(x_i)}
							\approx 1.039
				\end{align*}
				The mean is 1.14 toppings while the standard deviation is about 1.039 toppings.
			\item
				One ticked being randomly selected from \(1,000\) means that the probability of any one ticket being selected is \(1/1,000 = 0.001\). This describes a discrete random variable \(X\) with probability distribution
					\[\begin{array}{|c|*{2}{c}|}\hline
						x & 0 & 387 \\\hline
						P(x) & 1 - 0.001 = 0.999 & 0.001 \\\hline
					\end{array}\]
					The expected value of this random variable is then
					\[
						\Exp[X] = \sum x_iP(x_i)
							= 0.387
					\]
					\(X\) represents the money gained from the drawing rather than the profit, though. The amount of money spent on the ticket is a constant 2, so 2 can simply be subtracted from this result to yield the expected profit of entering the drawing, which is \(-\$1.613\)
			\item
				The number of defective computers is a discrete random variable \(X\) with probability distribution
					\[\begin{array}{|c|*{5}{c}|}\hline
						x & 0 & 1 & 2 & 3 & 4 \\\hline
						P(x) & 0.4305 & 0.4039 & 0.1421 & 0.0222 & 0.0013 \\\hline
					\end{array}\]
					As such,
					\[
						\mu_X = \sum x_iP(x_i)
							= 0.7599
					\]
					The mean number of defective computers in a batch of 4 is 0.7599.
			\item
				The number of defective computers is a discrete random variable \(X\) with probability distribution
					\[\begin{array}{|c|*{5}{c}|}\hline
						x & 0 & 1 & 2 & 3 & 4 \\\hline
						P(x) & 0.5220 & 0.3685 & 0.0975 & 0.0115 & 0.0005\\\hline
					\end{array}\]
					As such,
					\[
						\sigma_X = \sqrt{\sum(x_i - \mu_X)^2P(x_i)}
							\approx 0.714
					\]
			\item
				The number of trials (days) and the probability of success (rain) are fixed at 3 and 0.4 respectively. This situation therefore follows a binomial setting with probability of success \(p = 0.4\) and number of trials \(n = 3\). Let \(X\) be a random variable following this binomial distribution. The only possible values of \(X\) are 0, 1, 2, and 3. As such, the complement of \(X = 0\) is equivalent to \(X \ge 1\). Therefore,
					\[
						P(X \ge 1) = 1 - P(X = 0)
							= 1 - \binom{3}{0}0.4^0(1 - 0.4)^{3 - 0}
							= 1 - 0.216
							= 0.784 
					\]
					The probability of it raining on at least 1 of the 3 days is 0.784.
			\item
				The probability of a given student owning a credit card given that they are a freshman is the relative frequency of owning one among freshman. This is simply the proportion of freshman that carry credit cards:
					\begin{align*}
						P(\text{carries a credit card} \,|\, \text{freshman}) &= \frac{\text{\# of freshman credit card carriers}}{\text{\# of freshman}} \\
							&= \frac{40}{60}
								= \frac{2}{3}
								\approx 0.667
					\end{align*}
					The probability of a randomly selected freshman carrying a credit card is about 0.667.
			\item
				The probability of a given student being a freshman given that they carry a credit card is simply the relative frequency of being a freshman among those that own one. This is the proportion of credit card carriers that are freshman:
					\begin{align*}
						P(\text{freshman} \,|\, \text{carries a credit card}) &= \frac{\text{\# of freshman credit card carriers}}{\text{\# of credit card carriers}} \\
							&= \frac{49}{61}
								\approx 0.803
					\end{align*}
					The probability of a randomly selected credit card owner being a freshman is about 0.803.
			\item
				
		\end{enumerate}
	\setcounter{section}{0}
	\section{The Foundations: Logic and Proofs}
		\subsection{Propositional Logic}
			\begin{enumerate}
				\item
					\begin{tasks}(3)
						\task
							True proposition
						\task
							False proposition
						\task
							True proposition
						\task
							False proposition
						\task
							Not a proposition
						\task
							Not a proposition
					\end{tasks}
				\enumset{2}
				\item
					\begin{tasks}(2)
						\task
							Linda is not younger than Sanjay.
						\task
							Mei does not make more money than Isabella.
						\task
							Moshe is not taller than Monica.
						\task
							The moon is not made of green cheese.
						\task
							\(2^n < 100\).	
					\end{tasks}
				\enumset{4}
				\item
					\begin{tasks}(2)
						\task
							Mei does not have an MP3 player.
						\task
							There is pollution in New Jersey.
						\task
							\(2 + 1 \ne 3\).
						\task
							The summer in Maine is neither hot nor sunny.	
					\end{tasks}
				\enumset{10}
				\item
					\begin{tasks}(2)
						\task
							Sharks have not been spotted near the shore.
						\task
							Swimming at New Jersey shore is allowed and sharks have been spotted near the shore.
						\task
							Swimming at the New Jersey shore is not allowed or sharks have been spotted at the shore.
						\task
							Swimming at the New Jersey shore being allowed implies that sharks have not been spotted near the shore.
						\task
							Sharks not being spotted near the shore implies that Swimming at the New Jersey shore is allowed.
						\task
							Swimming at the New Jersey shore not being allowed implies that sharks have not been spotted near the shore.
						\task
							Swimming at the New Jersey shore is allowed if and only if sharks have not been spotted near the shore.
						\task
							Swimming at the New Jersey shore is not allowed or swimming at the New Jersey shore is allowed and sharks have not been spotted near the shore.
					\end{tasks}
				\enumset{12}
				\item
					\begin{tasks}(3)
						\task
							\(p \land q\)
						\task
							\(p \land \lnot q\)
						\task
							\(\lnot p \land \lnot q\)
						\task
							\(q \lor p\)
						\task
							\(p \to q\)
						\task
							\(p \leftrightarrow q\)
					\end{tasks}
				\enumset{14}
				\item
					\begin{tasks}(3)
						\task
							\(\lnot p\)
						\task
							\(p \land \lnot q\)
						\task
							\(q \leftarrow p\)
						\task
							\(\lnot p \to \lnot q\)
						\task
							\(p \to q\)
						\task
							\(q \to p\)
					\end{tasks}
				\enumset{18}
				\item
					\begin{tasks}(4)
						\task
							False
						\task
							False
						\task
							False
						\task
							False	
					\end{tasks}
				\newpage
				\enumset{24}
				\item
					\begin{tasks}(2)
						\task
							If the wind blows from the northeast, then it snows.
						\task
							If it stays warm for a week, then the apple trees will bloom.
						\task
							If the Pistons win the championships, then they beat the Lakers.
						\task
							If one has gotten to the top of Long's peak, they must have walked eight miles.
						\task
							If you are famous, then you can get tenure as a professor.
						\task
							If you drive more than 400 miles, then you will need to buy gasoline.
						\task
							If you bought your CD player less than 90 days ago, your guarantee is good.
						\task
							If the water is not too cold, Jan will go swimming.
						\task
							If people believe in science, we will have a future.
					\end{tasks}
				\enumset{26}
				\item
					\begin{tasks}(2)
						\task
							You can buy an ice cream cone if and only if it is hot outside.
						\task
							You can win the contest if and only if you have the only winning ticket.
						\task
							You can get promoted if and only if you have connections.
						\task
							Your mind will decay if and only if you watch television.
						\task
							The trains run late if and only if I take it.
					\end{tasks}
				\enumset{30}
				\item
					\begin{tasks}(4)
						\task
							2
						\task
							16
						\task
							64
						\task
							16	
					\end{tasks}
				\enumset{32}
				\item
					\begin{tasks}
						\task
							\[\begin{array}{|*{3}{c|}}\hline
								p & \lnot p & p \land \lnot p \\\hline
								0 & 1 & 0 \\
								1 & 0 & 0 \\\hline
							\end{array}\]
						\task
							\[\begin{array}{|*{3}{c|}}\hline
								p & \lnot p & p \lor \lnot p \\\hline
								0 & 1 & 1 \\
								1 & 0 & 1 \\\hline
							\end{array}\]
						\task
							\[\begin{array}{|*{5}{c|}}\hline
								p & q & \lnot q & p \lor \lnot q & (p \lor \lnot q) \to q \\\hline
								0 & 0 & 1 & 1 & 0 \\
								0 & 1 & 0 & 0 & 1 \\
								1 & 0 & 1 & 1 & 0 \\
								1 & 1 & 0 & 1 & 1 \\\hline
							\end{array}\]
						\task
							\[\begin{array}{|*{5}{c|}}\hline
								p & q & p \lor q & p \land q & (p \lor q) \to (p \land q) \\\hline
								0 & 0 & 0 & 0 & 1 \\
								0 & 1 & 1 & 0 & 0 \\
								1 & 0 & 1 & 0 & 0 \\
								1 & 1 & 1 & 1 & 1 \\\hline
							\end{array}\]
						\task
							\[\begin{array}{|*{7}{c|}}\hline
								p & q & p \to q & \lnot p & \lnot q & \lnot q \to \lnot p & (p \to q) \leftrightarrow (\lnot q \to \lnot p) \\\hline
								0 & 0 & 1 & 1 & 1 & 1 & 1 \\
								0 & 1 & 1 & 1 & 0 & 1 & 1 \\
								1 & 0 & 0 & 0 & 1 & 0 & 1 \\
								1 & 1 & 1 & 0 & 0 & 1 & 1 \\\hline
							\end{array}\]
						\task
							\[\begin{array}{|*{5}{c|}}\hline
								p & q & p \to q & q \to p & (p \to q ) \to (q \to p) \\\hline
								0 & 0 & 1 & 1 & 1 \\
								0 & 1 & 1 & 0 & 0 \\
								1 & 0 & 0 & 1 & 1 \\
								1 & 1 & 1 & 1 & 1 \\\hline
							\end{array}\]
					\end{tasks}
					\enumset{34}
					\item
						\begin{tasks}
							\task
									
						\end{tasks}
			\end{enumerate}
	\section{Fall Test 4}
		\begin{enumerate}
			\enumset{8}
			\item
				\[\begin{array}{|*{9}{c|}}\hline
					p & q & r& \lnot q & \lnot r & r \to \lnot q & \lnot(r \to \lnot q) & p \land \lnot r & \lnot(r \to \lnot p) \lor (p \land \lnot r) \\\hline
					0 & 0 & 0 & 1 & 1 & 1 & 0 & 0 & 0 \\
					0 & 0 & 1 & 1 & 0 & 1 & 0 & 0 & 0 \\
					0 & 1 & 0 & 0 & 1 & 1 & 0 & 0 & 0 \\
					0 & 1 & 1 & 0 & 0 & 0 & 1 & 0 & 1 \\
					1 & 0 & 0 & 1 & 1 & 1 & 0 & 1 & 1 \\
					1 & 0 & 1 & 1 & 0 & 1 & 0 & 0 & 0 \\
					1 & 1 & 0 & 0 & 1 & 1 & 0 & 1 & 1 \\
					1 & 1 & 1 & 0 & 0 & 0 & 1 & 0 & 1 \\\hline
				\end{array}\]
			\item
				\begin{tasks}(2)
					\task
						\(q \land \lnot p\)	
					\task
						\(\lnot(\lnot q \lor p)\)
				\end{tasks}
			\item
				\(p \land r \land \lnot q\)
			\item
				\[\begin{array}{|*{8}{c|}}\hline
					p & q & r & q \to r & p \to (q \to r)  & q \land r & p \to (q \land r)\\\hline
					0 & 0 & 0 & 1 & 1 & 0 & 1 \\
					0 & 0 & 1 & 1 & 1 & 0 & 1 \\
					0 & 1 & 0 & 0 & 1 & 0 & 1 \\
					0 & 1 & 1 & 1 & 1 & 1 & 1 \\
					1 & 0 & 0 & 1 & 1 & 0 & 0 \\
					1 & 0 & 1 & 1 & 1 & 0 & 0 \\
					1 & 1 & 0 & 0 & 0 & 0 & 0 \\
					1 & 1 & 1 & 1 & 1 & 1 & 1 \\\hline
				\end{array}\]
				The 5th and and 7th columns are not equivalent, so \(p \to (q \to r) \not\equiv p \to (q \land r)\).
			\item
				\begin{align}
					p \to (q \to r) &\equiv \lnot p \lor (q \to r) \tag{definition of \(\to\)} \\
						&\equiv \lnot p \lor (\lnot q \lor r) &\tag{definition of \(\to\)} \\
						&\equiv \lnot p \lor \lnot q \lor r & \tag{associativity of \(\lor\)} \\
						&\equiv (p \to \lnot q) \lor r \tag{}
				\end{align}

		\end{enumerate}

\end{document}
