\documentclass[A4, 12pt]{article}

% Packages
	% Basics
		\usepackage{amsmath}
		\usepackage[shortlabels]{enumitem}
		\usepackage[margin = 0.75in]{geometry}
	% Notation
		\usepackage{esint} % Integrals
		\usepackage{physics} % Differentials/Vectors
		
\title{Discussion 2: Autonomous or Non-Autonomous DE}
\author{Arnav Patri}

\begin{document}
	\maketitle
		\begin{enumerate}[1)]
			\item
				\[
					\dv{y}{x} + xy = 3 
						\implies \dv{y}{x} = 3 - xy = f(x, y)
				\]
				As \(\dv*{y}{x}\) can not be expressed as a function of just \(y\), it this DE is \textbf{non-autonomous}.
			\item
				\[
					\dv{y}{x} + y = 3x
						\implies \dv{y}{x} = 3x - y = f(x, y)
				\]
				\(\dv*{y}{x}\) is dependent on both \(x\) and \(y\), meaning that this DE is \textbf{non-autonomous} as well.
			\item
				\[
					x\dv{y}{x} - y = 0
						\implies \dv{y}{x} = \frac{y}{x} = f(x, y)
				\]
				Because \(\dv*{y}{x}\) is a ratio of \(y\) and \(x\), it is dependent on both, so this DE is \textbf{non-autonomous}.
			\item
				\[
					\dv{y}{x} + 5y = 0
						\implies \dv{y}{x} = -5y = g(y)
				\]
				Since \(\dv*{y}{x}\) can be expressed as a function of \(y\), this DE is \textbf{autonomous}.
		\end{enumerate}
\end{document}
