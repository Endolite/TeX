\documentclass[../AP_Statistics.tex]{subfiles}

\begin{document}
	\chapter{Sampling Distributions}
		A \textbfi{parameter} is a numerical value that describes a \emphi{population}. A parameter representing a numerical value is a \textbfi{mean}, denoted $\mu$, while one representing a proportion is a \textbfi{proportion}, denoted $p$. An arbitrary parameter can be denoted $\theta$. \\
		A \textbfi{statistic} is a numerical value that describes a \emphi{sample}. A sample mean is denoted $\bar{x}$ while a sample proportion is denoted $\hat{p}$. An arbitrary statistic can be denoted $\hat{\theta}$. \\
		
		\section{Sampling Distributions of Proportions}
			\subsection*{Sampling Distributions of Differences in Proportions}
		\section{Sampling Distributions of Means}
			\subsection*{Sampling Distributions of Differences in Means}
\end{document}