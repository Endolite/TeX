\documentclass{subfiles}

\begin{document}
	In analyzing complex physical systems, it is often convenient to separate the important parts from the less important parts. This approach unfortunately cannot be used in analyzing the motion of electrons in atoms with more than one electron. The problem of mutual significant interactions between 3 or more bodies is called a \textit{many-body problem}. For such problems, exact, closed solutions cannot be found for the Schr\"odinger equation.
		\section{The Pauli Exclusion Principle}
			It would be intuitive for all \(Z\) electrons to eventually cascade down to the lowest energy level, the \(1s\) state. If this were the case, it would be expected that atoms with \(Z \pm 1\) display similar properties. Indeed, some properties of atoms, such as the energies of emitted X rays, do vary smoothly with \(Z\). Other properties violate this notion, though, such as reactivity. \\
			The \textit{Pauli exclusion principle} states that \textit{no two electrons in a single atom can share a set of quantum numbers}. The filling of the orbitals in this way accurately models the properties of the elements.
		\section{Electronic States in Many-Electron Atoms}
			The set of orbits of a specific principle quantum number are the same average distance from the nucleus and is called an atomic \textit{shell}. The shells are designated by letters:
				\[\begin{array}{*{6}{l}}
					n & 1 & 2 & 3 & 4 & 5 \\
					\text{Shell} & K & L & M & N & O
				\end{array}\]
				The levels with a specified value for both \(n\) and \(\ell\) are \textit{subshells}. The Pauli principle dictates that the maximum allowed electrons in a given subshell is \(2(2\ell + 1)\). This is because \(m_\ell\) can take on \(2\ell + 1\) values and \(m_s\) can take 2 values. \\
				\subsectionb{The Periodic Table}
					The periodic table is an arrangement of the elements in order of ascending \(Z\). They are arranged in such a way that the columns, called \textit{groups}, contain elements with similar properties. \\
					The rules that are followed in the filling of electronic subshells are as follows:
						\begin{enumerate}
							\item 
								The capacity of each subshell is \(2(2\ell + 1)\).
							\item
								Electrons occupy the lowest available energy state.
						\end{enumerate}
						An element's electron configuration is indicated by the subshells in fill order, each of which is denoted by the shell's number followed by the orbital's letter with the superscript of the number of electrons in that shell. \\
						Each row or \textit{period} of the periodic table begins with the next \(s\) orbital. \\
						The rightmost row of the periodic table is comprised of \textit{noble} or \textit{inert} gases. An element's electron configuration can be denoted by the symbol of the previous noble gas in brackets followed by the subsequent subshells.
\end{document}
