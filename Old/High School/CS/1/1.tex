\documentclass[12pt, A4]{article}

\usepackage{amsmath}
\usepackage{amssymb}
\usepackage{arydshln}
\usepackage{csquotes}
\usepackage[hang,flushmargin]{footmisc} 
\usepackage[margin=0.5in]{geometry}
\usepackage[utf8]{inputenc}
\usepackage{moresize}
\usepackage{tikz}
\usetikzlibrary{calc}

\setcounter{secnumdepth}{0}

\newcommand{\callout}[2]{\begin{center}\fbox{\begin{minipage}{#1cm}#2\end{minipage}}\end{center}}

\title{Computer Science 1}
\author{Arnav Patri}

\begin{document}
	\maketitle
	\tableofcontents
	
	\section{Binary}
		\subsection{Conversions}
			\textbf{Binary} is a base-2 numbering system, just as \textbf{decimal} is base-10. This means that binary makes use of only two unique symbols to denote digits, being 1's and 0's. \\
			A base-$n$ system can be read left-to-right. Each digit is multiplied by $n$ to the power of 1 less than the number of digits from the right. This applies to binary as well. \\
			The \textbf{one's complement} of a binary number is the result of \enquote{flipping} each of its digits, turning 0's to 1's and vice versa.
			\callout{17}{
				The one's complements of four-bit numbers are as follows:
				\[\begin{array}{|c|c|c|c|c|c|c|c|}\hline
					0000 & 0001 & 0010 & 0100 & 0101 & 0110 & 0111 & 1000 \\\hdashline
					1111 & 1110 & 1101 & 1011 & 1010 & 1001 & 1000 & 0111 \\\hline
					1001 & 1010 & 1011 & 1100 & 1101 & 1110 & 1111
				\end{array}\]
			}
\end{document}