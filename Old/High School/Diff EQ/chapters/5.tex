\documentclass[./Differential Equations.tex]{subfiles}

\begin{document}
	\section{Linear Models: Initial-Value Problems}
		\subsection{Spring/Mass Systems: Free Undampened Motion}
			\subsubsectionb{Hooke's Law}
				Suppose a rigid body of is attached to a flexible spring. The spring force \(F_s\) is proportional to the displacement \(s\) of the body from its equilibrium position and is in the direction of equilibrium; that is
					\[F_s = -ks\]
			\subsubsectionb{Newton's Second Law}
				Newton's second law of motion states that
					\[\subt{F}{net} = ma\]
					where \(m\) is mass and \(a\) is acceleration or \(\dv*[2]{x}{t}\). \\
				If the mass on a spring vibrates without regard to any external forces, having \textbf{free motion}, and the mass is hanging vertically from the spring, then Newton's second law gives
					\[
						m\dv[2]{x}{t} = \overset{\subt{F}{net}}{\overbrace{-(x + s) + mg}}
							= -kx + \overset{0}{\overbrace{mg - ks}} 
							= -kx
					\]
					where \(s\) is the equilibrium position (where \(ks = mg\)) and \(x\) is the displacement from equilibrium.
			\subsubsectionb{DE of Free Undampened Motion}
				Dividing by \(m\) yields the second-order DE
					\[\dv[2]{x}{t} + \omega^2x = 0\]
					where \(\omega^2 = k/m\). This equation is said to describe \textbf{simple harmonic motion (SHM)} or \textbf{free undampened motion}.
			\subsubsectionb{Equation of Motion}
				The auxiliary equation of the SHM DE is
					\[m^2 + \omega^2 = 0\]
					so the solutions are
					\[m = \pm i\omega\]
					making the general solution
					\[x(t) = C_1\cos(\omega t) + C_2\sin(\omega t)\]
				The \textbf{period} of motion \(\bm{T}\) (in \(\SI{}{s}\)), the amount of time it takes for a full oscillation to occur, is
					\[T = \frac{2\pi}{\omega}\]
					The \textbf{frequency} of motion \(\bm{f}\) (in \(\SI{}{s^{-1}}\) or \(\SI{}{hz}\)) is
					\[f = \frac{1}{T} = \frac{\omega}{2\pi}\]
					The number
					\[\omega = \sqrt{\frac{k}{x}}\]
					(in \(\SI{}{rad/s}\)) and \(f\) are both sometimes referred to as the \textbf{natural frequency} of the system. \\
					The equation derived by solving for the constants of the general solution is the \textbf{equation of motion} of the system.
			\subsubsectionb{Alternative Forms of \(\bm{x(t)}\)}
				When \(C_1, C_2 \ne 0\), the \textbf{actual amplitude \(\bm{A}\)} of free vibrations is not immediately obvious, so it is often convenient to convert the equation of SHM to the simpler form
					\[x(t) = A\sin(\omega t + \varphi)\]
					where
					\[A = \sqrt{C_1^2 + C_2^2}\]
					and \(\varphi\) is a \textbf{phase angle} defined by
					\[
						\left.\begin{aligned}
							\sin\varphi &= \frac{C_1}{A} \\
							\cos\varphi &= \frac{C_2}{A}
						\end{aligned}\right\}
						\tan\varphi = \frac{C_1}{C_2}
					\]
				A cosine function is sometimes preferred, making the solution
					\[x(t) = A\cos(\omega t + \varphi)\]
					where \(\varphi\) is defined by
					\[
						\left.\begin{aligned}
							\sin\varphi &= \frac{C_2}{A} \\
							\cos\varphi &= \frac{C_1}{A}
						\end{aligned}\right\}
						\tan\varphi = \frac{C_2}{C_1}
					\]
			\subsubsectionb{Double Spring Systems}
				The \textbf{effective spring constant \(\bm{\subt{k}{eff}}\)} of a system with two \textit{parallel} springs with spring constants \(k_1\) and \(k_2\) is 
					\[\subt{k}{eff} = k_1 + k_2\]
				That of a system with two \textit{series} springs is
					\[\subt{k}{eff} = \frac{k_1k_2}{k_1 + k_2}\]
			\subsubsectionb{Systems with Variable Spring Constants}
				In reality, it is reasonable to expect the spring constant to decay over time. One model for the \textbf{aging spring} replaces the spring constant \(k\) with the decreasing function
					\[K(t) = k\en^{-\alpha t}\]
					where \(k\) and \(\alpha\) are positive constants. The linear DE
					\[mx'' + k\en^{-\alpha t}x = 0\]
					cannot be solved with the methods discussed thus far. \\
				When a spring/mass system is subject to a rapidly decreasing temperature, \(k\) may be replaced with \(K(t) = kt\) where \(k\) is a positive constant, a function that increases with time. The resulting model
				\[mx'' + ktx = 0\]
				is a form of \textbf{Airy's differential equation}.
		\subsection{Spring/Mass Systems: Free Dampened Motion}
			\subsubsection{DE of Free Dampened Motion}
				Damping forces are proportional to a power of the instantaneous velocity. It is assumed here that it is a constant multiple of \(\dv*{x}{t}\). When no external forces are present, it follows from Newton's second law that
					\[m\dv[2]{x}{t} = -kx - \beta\dv{x}{t}\]
					where \(\beta\) is a positive \textit{damping constant}. Dividing by \(m\) yields
					\[
						0 = \dv[2]{x}{t} + \frac{\beta}{m}\dv{x}{t} + \frac{k}{m}x 
							= \dv[2]{x}{t} + 2\lambda\dv{x}{t} + \omega^2x
					\]
					where
					\[2\lambda = \frac{\beta}{m}\]
					\(2\lambda\) is used for convenience, as the auxiliary equation then becomes
					\[0 = m^2 + 2\lambda m + \omega^2\]
					which has roots
					\[m = -\lambda \pm \sqrt{\lambda^2 - \omega^2}\]
					The discriminant is then \(\lambda^2 - \omega^2\), providing 3 possible cases. \\
					As each solution contains the \textit{damping factor} \(\en^{-\lambda t}\) and \(\lambda\) is positive, the displacements become negligible as \(t\) increases.
			\subsubsectionb{Case I: \(\bm{\lambda^2 - \omega^2 > 0}\)}
				The system is said to be \textbf{overdamped} when \(\lambda^2 > \omega^2\), as \(\beta\) is large compared to \(k\). The corresponding solution is
					\[x(t) = \en^{-\lambda t}\left(C_1\en^{t\sqrt{\lambda^2 - \omega^2}} + C_2\en^{-t\sqrt{\lambda^2 - \omega^2}}\right)\]
					This equation describes smooth nonoscillatory motion.
			\subsubsectionb{Case II: \(\bm{\lambda^2 - \omega^2 = 0}\)}
				The system is said to be \textbf{critically dampened} when \(\lambda^2 = \omega^2\), as any slight decrease in the damping force would yield oscillatory motion. The general solution is
					\[x(t) = \en^{-\lambda t}\left(C_1 + C_2t\right)\]
					The motion described by this equation is quite similar to that of an overdampened system. Note that the mass may pass through the equilibrium position at most once.
			\subsubsectionb{Case III: \(\bm{\lambda^2 - \omega^2 < 0}\)}
				The system is said to be \textbf{underdamped} when \(\lambda^2 < \omega^2\), as \(\beta\) is small compared to \(k\). The roots are now complex, being
					\[m = -\lambda \pm i\sqrt{\omega^2 - \lambda^2}\]
					The general solution is
					\[x(t) = \en^{-\lambda t}\left(C_1\cos\left(t\sqrt{\omega^2 - \lambda^2}\right) + C_2\sin\left(t\sqrt{\omega^2 - \lambda^2}\right)\right)\]
					The motion described by this equation is oscillatory, but due to the \(\en^{-\lambda t}\) term, the amplitude approaches 0 as \(t\) increases.
			\subsubsectionb{Alternative Form of \(\bm{x(t)}\)}
				Any solution
					\[x(t) = \en^{-\lambda t}\left(C_1\cos\left(t\sqrt{\omega^2 - \lambda^2}\right) + C_2\sin\left(t\sqrt{\omega^2 - \lambda^2}\right)\right)\]
					can be rewritten as
					\[x(t) = A\en^{-\lambda t}\sin\left(t\sqrt{\omega^2 - \lambda^2} + \varphi\right)\]
					where
					\[
						A = \sqrt{C_1^2 + C_2^2} \qquad \text{and} \qquad
						\tan\varphi = \frac{C_1}{C_2}
					\]
					\(A\en^{-\lambda t}\) is sometimes referred to as the \textbf{damped amplitude} of vibrations. \\
				As the solution is not periodic, the number
					\[\frac{2\pi}{\sqrt{\omega^2 - \lambda^2}}\]	
					is called the \textbf{quasi period} and
					\[\frac{\sqrt{\omega^2 - \lambda^2}}{2\pi}\]
					the \textbf{quasi frequency}. \\
					The quasi period is the interval between two successive maxima.
		\subsection{Spring/Mass Systems: Driven Motion}
			\subsubsectionb{DE of Driven Motion with Damping}
				Consider an external force \(f(t)\). Including it in the formulation of Newton's second law yields the DE of \textbf{driven/forced motion}:
					\[m\dv[2]{x}{t} = -kx - \beta\dv{x}{t}\]
					Dividing by \(m\) yields
					\[\dv[2]{x}{t} + 2\lambda\dv{x}{t} + kx = F(t)\]
					where 
					\[F(t) = \frac{f(t)}{m}\]
			\subsubsectionb{Transient and Steady-State Terms}
				When \(F\) is a periodic function, such as
					\[F(t) = F_0\sin(\gamma t) \qquad \text{or} \qquad F(t) = F_0\cos(\gamma t)\]
	\section{Linear Models: Boundary-Value Problems}
		\subsectionb{Deflection of a Beam}
			Many structures employ girders or beams to keep stable. Such beams distort or deflect under their own weight or under the influence of some external force. This deflection \(y(x)\) is governed by a relatively simple \(4^{\th}\)-order DE. \\
			Assume that a beam of length \(L\) is homogenous with uniform cross-sections along its length. Without any load (including the beam's own weight, a curve that joins the centroids of each cross section in a straight line is called the \textbf{axis of symmetry}. If a load is then applied in a vertical plane containing the axis of symmetry, the beam undergoes a distortion. The curve that now connects all centroids is the \textbf{deflection/elastic curve}. Let the axis of symmetry be the \(x\)-axis and that the deflection \(y(x)\) is measured relative to it pointing downwards. The bending moment \(M(x)\) at a point \(x\) along the beam can be found from the load per unit length \(w(x)\) by
				\[\dv[2]{M}{x} = w(x)\]
				It is also proportional to the curvature \(\kappa\) as
				\[\dv[2]{M}{x} = EI\kappa\]
				where \(E\) and \(I\) are constants, being Young's modulus of the beam's material and the moment of inertia of a cross section (about an axis referred to as the neutral axis) respectively. The product is called the \textbf{flexural rigidity} of the beam. \\
				Curvature is given by
				\[\kappa = \frac{y''}{\left(1 + (y')^2\right)^{3/2}}\]
			When the deflection is small, the slope \(y' \approx 0\), so \(\left(1 + (y')^2\right)^{3/2} \approx 1\). Letting \(\kappa \approx y''\), \(M = EIy''\), so
				\[\dv[2]{M}{x} + EI\dv[2]{x}y'' = EI\dv[4]{y}{x}\]
				It can then be seen that 
				\[EI\dv[4]{y}{x} = w(x)\]
				The boundary conditions are dependent on how the beam is supported. A cantilever beam is \textbf{embedded} or \textbf{clamped} at one end and \textbf{free} at the other. For a cantilever, \(y(0) = y'(0) = y''(L) = y'''(L) = 0\) \\
			The function \(F(x) = \dv*{M}{x} = EI\dv*[3]{y}{x}\) is called the shear force. If the end of a beam is \textbf{simply/pin/fulcrum supported} or \textbf{hinged}, then \(y = y' = 0\) at that end. In summary,
				\[\begin{array}{|c|cc|}\hline
					\text{Ends} & \text{Boundary} & \text{Conditions} \\\hline 
					\text{Embedded} & y = 0, & y' = 0 \\
					\text{Free} & y'' = 0, & y''' = 0 \\
					\text{Simply Supported or HInged} &  y = 0, & y'' = 0 \\\hline
				\end{array}\]
		\subsectionb{Eigenvalues and Eigenfunctions}
			Many problems require the solving of a 2-point BVP involving a linear DE containing a parameter \(\lambda\). \textit{Nontrivial (nonzero)} solutions are sought. \\
			Consider the BVP
				\[
					y'' + \lambda y = 0, \quad
					y(0) = 0, \quad
					y(L) = 0
				\]
			This presents 3 cases:
			\subsubsectionb{Case I: \(\bm{\lambda = 0}\)}
				For \(\lambda = 0\), the solution of \(y'' = 0\) is \(y = C_1x + C_2\). The conditions imply that \(C_1 = C_2 = 0\), so the only solution is the trivial solution \(y = 0\).
			\subsubsectionb{Case II: \(\bm{\lambda < 0}\)}
				For \(\lambda < 0\), it is convenient to make the substitution \(\lambda = \alpha^2\) (where \(\alpha \in \R^+\)). The roots of the auxiliary equation \(m^2 - \alpha^2 = 0\) are then \(m = \pm \alpha\). As the interval is finite, the general solution of
					\[y'' - \alpha^2y = 0\]
					is written as
					\[y = C_1\cosh(\alpha x) + C_2\sinh(\alpha x)\]
					making \(y(0)\)
					\[y(0) = C_1\cosh(0) + C_2\sinh(0) = C_1\]
					\(y(0) = 0\) then implies that \(C_1 = 0\), so \(y\) becomes
					\[y = C_2\sinh(\alpha x)\]
					The second condition, \(y(L) = 0\), means that
					\[y(L) = C_2\sinh(\alpha L) = 0\]
					which, for \(\alpha \ne 0\) necessitates that \(C_2 = 0\). The only solution for this BVP is then the trivial solution \(y = 0\).
			\subsubsectionb{Case III: \(\bm{\lambda > 0}\)}
				For \(\lambda > 0\), it is also convenient to write \(\lambda = -\alpha^2\) (where \(\alpha \in \R^+\)). The roots of the auxiliary equation \(m^2 + \alpha^2 = 0\) are then \(m = \pm i\alpha\), so the general solution of
					\[y'' + \alpha^2y = 0\]
					is
					\[y = C_1\cos(\alpha x) + C_2\sin(\alpha x)\]
					\(y(0)\) again implies that \(C_1  0\) and the condition \(y(L) = 0\) or
					\[C_2\sin(\alpha L) = 0\]
					is satisfied by \(C_2 = 0\). This again yields the trivial solution \(y = 0\). If it is required that \(C_2 \ne 0\), though, then \(\sin(\alpha L) = 0\) is satisfied so long as \(\alpha L\) is an integer multiple of \(\pi\):
					\[
						\alpha L = n\pi \quad \text{or} \quad
						\alpha = \frac{n\pi}{L} \quad \text{or} \quad
						\lambda_n = \alpha_n^2 = \left(\frac{n\pi}{L}\right)^2, \quad
						n \in \Z^+
					\]
					For any real nonzero \(C_2\),
					\[y_n(x) = C_2\sin\left(\frac{n\pi x}{L}\right)\]
					is a solution for \(n \in \Z^+\). As the DE is homogenous, any constant multiple of a solution is itself also a solution, so \(C_2\) can be set to 1. In other words, for each number in the sequence
					\[\lambda_n = \frac{n^2\pi^2}{L^2} \quad \text{for} \quad n \in \Z^+\]
					the corresponding function
					\[y_n = \sin\left(\frac{n\pi}{L}\right)\]
					is a nontrivial solution of the problem
					\[
						y'' + \lambda_ny = 0, \quad
						y(0) = 0, \quad
						y(L) = 0
					\]
					The numbers \(\lambda_n\) are known as \textbf{eigenvalues}. The nontrivial solutions are called \textbf{eigenfunctions}.
\end{document}