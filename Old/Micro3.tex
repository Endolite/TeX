\documentclass[12pt]{article}
\usepackage[margin = 1 in]{geometry}          
\usepackage{graphicx}
\usepackage{amsthm, amsmath, amssymb}
\usepackage{setspace}\onehalfspacing
\usepackage[loose,nice]{units} %replace "nice" by "ugly" for units in upright fractions
\usepackage{enumitem}

\title{AP Microeconomics Lesson 3.4}



\begin{document}

\begin{flushleft}
 Which of the following can be done to avoid diminishing marginal returns to labor?
 \end{flushleft}

\begin{enumerate}[label = \alph*)]
	\item Reduce the plant size and quantity of equipment
	\item Reduce the quantity of resources available
	\item Increase the number of workers
	\item Increase the quantity of capital
	\item Increase the amount of labor used
\end{enumerate}


\begin{flushleft}
For a large-scale bakery, which of the following is most likely to be an input that is adjustable in the long-run but not in the short-run?
\end{flushleft}

\begin{enumerate}[label = (\alph*)]
	\item Workers
	\item Flour
	\item Electricity
	\item The Building
	\item Sesame Seeds
\end{enumerate}


\begin{center}
\vspace{1cm}
\begin{tabular}{ |c|c| }
	\hline
	\textbf{Units of Labor} & \textbf{Output} \\
	\hline
	0 & 0 \\
	5 & 1175 \\
	10 & 1250 \\
	15 & 1300 \\
	20 & 1350 \\
	25 & 1380 \\
	30 & 1400 \\
	\hline
\end{tabular}
\end{center}

\begin{flushleft}
Using the above table, what is the marginal product of labor (\emph{MP\textsubscript{L}}) of the 25\textsuperscript{th} unit of labor?\end{flushleft}

\begin{enumerate}[label = (\alph*)]
	\item 6 Units of Output
	\item 4 Units of Output
	\item 55.2 Units of Output
	\item 25 Units of Output
	\item It cannot be determined with the data provided
\end{enumerate}


\begin{flushleft}
What is true of marginal product when total product is maximized?
\end{flushleft}

\begin{enumerate}[label = (\alph*)]
	\item It is increasing
	\item It is equal to 0
	\item It equal to 1
	\item It negative
	\item It is positive
\end{enumerate}


\begin{center}
\begin{tabular}{ |c|c| }
	\hline
	\textbf{Units of Labor} & \textbf{Total Product of Labor (\emph{TP\textsubscript{L}})} \\
	\hline
	0 & 0 \\
	1 & 2 \\
	2 & 6 \\
	3 & 12 \\
	4 & 17 \\
	5 & 21 \\
	6 & 24 \\
	7 & 26 \\
	8 & 25 \\
	9 & 23 \\
	\hline
\end{tabular}
\end{center}

\begin{flushleft}
Using the above table, at what quantity of labor does diminishing returns to labor take effect?
\end{flushleft}

\begin{enumerate}[label = (\alph*)]
	\item At the 5\textsuperscript{th} unit of labor
	\item At the 4\textsuperscript{th} unit of labor
	\item At the 1\textsuperscript{st} unit of labor
	\item At the 8\textsuperscript{th} unit of labor
	\item At the 9\textsuperscript{th} unit of labor
\end{enumerate}


\begin{flushleft}
\vspace{1.325cm}
Why is economic profit always less than accounting profit?
\end{flushleft}

\begin{enumerate}[label = (\alph*)]
	\item Economic profit considers more costs than accounting profit.
	\item Economic profit includes more utility than accounting profit.
	\item Economic profit includes explicit but not implicit costs while accounting profit includes both.
	\item Economic profit includes more revenues than accounting profit.
	\item Economic profit includes fewer revenues than accounting profit.
\end{enumerate}


\begin{flushleft}
Suppose a firm produces and sells 10,000 greeting cards at \$5 each in a competitive market and that it has explicit costs of \$4,000 and an opportunity cost of capital of \$1,000. If the firm's owner closed the business for good and take another job, they could be earning \$31,000 annually. What is this firm's economic profit?
\end{flushleft}

\begin{enumerate}[label = (\alph*)]
	\item \$15,000
	\item \$14,000
	\item \$45,000
	\item \$12,000
	\item \$46,000
\end{enumerate}


\begin{flushleft}
What must be true of a firm earning normal profits?
\end{flushleft}

\begin{enumerate}[label = (\alph*)]
	\item Accounting profits are negative
	\item Accounting profits are positive
	\item Economic profits are equal to zero
	\item Economic profits are negative
	\item Economic profits are positive
\end{enumerate}


\begin{flushleft}
\vspace{1.9cm}
Suppose that there is an individual that is a writer and programmer and enjoys both professions equally. If they can earn \$80,000 as a writer or \$70,000 as a programmer and has no other costs associated with either profession. What is their economic profit if they choose to be a programmer rather than a painter?
\end{flushleft}

\begin{enumerate}[label = (\alph*)]
	\item \$0
	\item \$80,000
	\item \$70,000
	\item -\$10,000
	\item -\$150,000
\end{enumerate}

\begin{flushleft}
A firm produces 200 pies for \$20 each. The explicit coset of producing the pies is \$2,000, the opportunity cost of the firm's owner is \$1,000, and the building that the firm is in could be rented out for \$700. What is this firm's economic profit?
\end{flushleft}

\begin{enumerate}[label = (\alph*)]
	\item \$1,300
	\item \$300
	\item \$2,000
	\item \$4,000
	\item \$1,000
\end{enumerate}
\end{document}